%\documentclass[11pt]{article}
\documentclass[paper=landscape]{scrartcl}
	\usepackage[paperwidth=40cm,paperheight=265cm,margin=1in]{geometry}

    \usepackage[breakable]{tcolorbox}
    \usepackage{parskip} % Stop auto-indenting (to mimic markdown behaviour)
    
    \usepackage{iftex}
    \ifPDFTeX
    	\usepackage[T1]{fontenc}
    	\usepackage{mathpazo}
    \else
    	\usepackage{fontspec}
    \fi

    % Basic figure setup, for now with no caption control since it's done
    % automatically by Pandoc (which extracts ![](path) syntax from Markdown).
    \usepackage{graphicx}
    % Maintain compatibility with old templates. Remove in nbconvert 6.0
    \let\Oldincludegraphics\includegraphics
    % Ensure that by default, figures have no caption (until we provide a
    % proper Figure object with a Caption API and a way to capture that
    % in the conversion process - todo).
    \usepackage{caption}
    \DeclareCaptionFormat{nocaption}{}
    \captionsetup{format=nocaption,aboveskip=0pt,belowskip=0pt}

    \usepackage[Export]{adjustbox} % Used to constrain images to a maximum size
    \adjustboxset{max size={0.9\linewidth}{0.9\paperheight}}
    \usepackage{float}
    \floatplacement{figure}{H} % forces figures to be placed at the correct location
    \usepackage{xcolor} % Allow colors to be defined
    \usepackage{enumerate} % Needed for markdown enumerations to work
    \usepackage{geometry} % Used to adjust the document margins
    \usepackage{amsmath} % Equations
    \usepackage{amssymb} % Equations
    \usepackage{textcomp} % defines textquotesingle
    % Hack from http://tex.stackexchange.com/a/47451/13684:
    \AtBeginDocument{%
        \def\PYZsq{\textquotesingle}% Upright quotes in Pygmentized code
    }
    \usepackage{upquote} % Upright quotes for verbatim code
    \usepackage{eurosym} % defines \euro
    \usepackage[mathletters]{ucs} % Extended unicode (utf-8) support
    \usepackage{fancyvrb} % verbatim replacement that allows latex
    \usepackage{grffile} % extends the file name processing of package graphics 
                         % to support a larger range
    \makeatletter % fix for grffile with XeLaTeX
    \def\Gread@@xetex#1{%
      \IfFileExists{"\Gin@base".bb}%
      {\Gread@eps{\Gin@base.bb}}%
      {\Gread@@xetex@aux#1}%
    }
    \makeatother

    % The hyperref package gives us a pdf with properly built
    % internal navigation ('pdf bookmarks' for the table of contents,
    % internal cross-reference links, web links for URLs, etc.)
    \usepackage{hyperref}
    % The default LaTeX title has an obnoxious amount of whitespace. By default,
    % titling removes some of it. It also provides customization options.
    \usepackage{titling}
    \usepackage{longtable} % longtable support required by pandoc >1.10
    \usepackage{booktabs}  % table support for pandoc > 1.12.2
    \usepackage[inline]{enumitem} % IRkernel/repr support (it uses the enumerate* environment)
    \usepackage[normalem]{ulem} % ulem is needed to support strikethroughs (\sout)
                                % normalem makes italics be italics, not underlines
    \usepackage{mathrsfs}
    

    
    % Colors for the hyperref package
    \definecolor{urlcolor}{rgb}{0,.145,.698}
    \definecolor{linkcolor}{rgb}{.71,0.21,0.01}
    \definecolor{citecolor}{rgb}{.12,.54,.11}

    % ANSI colors
    \definecolor{ansi-black}{HTML}{3E424D}
    \definecolor{ansi-black-intense}{HTML}{282C36}
    \definecolor{ansi-red}{HTML}{E75C58}
    \definecolor{ansi-red-intense}{HTML}{B22B31}
    \definecolor{ansi-green}{HTML}{00A250}
    \definecolor{ansi-green-intense}{HTML}{007427}
    \definecolor{ansi-yellow}{HTML}{DDB62B}
    \definecolor{ansi-yellow-intense}{HTML}{B27D12}
    \definecolor{ansi-blue}{HTML}{208FFB}
    \definecolor{ansi-blue-intense}{HTML}{0065CA}
    \definecolor{ansi-magenta}{HTML}{D160C4}
    \definecolor{ansi-magenta-intense}{HTML}{A03196}
    \definecolor{ansi-cyan}{HTML}{60C6C8}
    \definecolor{ansi-cyan-intense}{HTML}{258F8F}
    \definecolor{ansi-white}{HTML}{C5C1B4}
    \definecolor{ansi-white-intense}{HTML}{A1A6B2}
    \definecolor{ansi-default-inverse-fg}{HTML}{FFFFFF}
    \definecolor{ansi-default-inverse-bg}{HTML}{000000}

    % commands and environments needed by pandoc snippets
    % extracted from the output of `pandoc -s`
    \providecommand{\tightlist}{%
      \setlength{\itemsep}{0pt}\setlength{\parskip}{0pt}}
    \DefineVerbatimEnvironment{Highlighting}{Verbatim}{commandchars=\\\{\}}
    % Add ',fontsize=\small' for more characters per line
    \newenvironment{Shaded}{}{}
    \newcommand{\KeywordTok}[1]{\textcolor[rgb]{0.00,0.44,0.13}{\textbf{{#1}}}}
    \newcommand{\DataTypeTok}[1]{\textcolor[rgb]{0.56,0.13,0.00}{{#1}}}
    \newcommand{\DecValTok}[1]{\textcolor[rgb]{0.25,0.63,0.44}{{#1}}}
    \newcommand{\BaseNTok}[1]{\textcolor[rgb]{0.25,0.63,0.44}{{#1}}}
    \newcommand{\FloatTok}[1]{\textcolor[rgb]{0.25,0.63,0.44}{{#1}}}
    \newcommand{\CharTok}[1]{\textcolor[rgb]{0.25,0.44,0.63}{{#1}}}
    \newcommand{\StringTok}[1]{\textcolor[rgb]{0.25,0.44,0.63}{{#1}}}
    \newcommand{\CommentTok}[1]{\textcolor[rgb]{0.38,0.63,0.69}{\textit{{#1}}}}
    \newcommand{\OtherTok}[1]{\textcolor[rgb]{0.00,0.44,0.13}{{#1}}}
    \newcommand{\AlertTok}[1]{\textcolor[rgb]{1.00,0.00,0.00}{\textbf{{#1}}}}
    \newcommand{\FunctionTok}[1]{\textcolor[rgb]{0.02,0.16,0.49}{{#1}}}
    \newcommand{\RegionMarkerTok}[1]{{#1}}
    \newcommand{\ErrorTok}[1]{\textcolor[rgb]{1.00,0.00,0.00}{\textbf{{#1}}}}
    \newcommand{\NormalTok}[1]{{#1}}
    
    % Additional commands for more recent versions of Pandoc
    \newcommand{\ConstantTok}[1]{\textcolor[rgb]{0.53,0.00,0.00}{{#1}}}
    \newcommand{\SpecialCharTok}[1]{\textcolor[rgb]{0.25,0.44,0.63}{{#1}}}
    \newcommand{\VerbatimStringTok}[1]{\textcolor[rgb]{0.25,0.44,0.63}{{#1}}}
    \newcommand{\SpecialStringTok}[1]{\textcolor[rgb]{0.73,0.40,0.53}{{#1}}}
    \newcommand{\ImportTok}[1]{{#1}}
    \newcommand{\DocumentationTok}[1]{\textcolor[rgb]{0.73,0.13,0.13}{\textit{{#1}}}}
    \newcommand{\AnnotationTok}[1]{\textcolor[rgb]{0.38,0.63,0.69}{\textbf{\textit{{#1}}}}}
    \newcommand{\CommentVarTok}[1]{\textcolor[rgb]{0.38,0.63,0.69}{\textbf{\textit{{#1}}}}}
    \newcommand{\VariableTok}[1]{\textcolor[rgb]{0.10,0.09,0.49}{{#1}}}
    \newcommand{\ControlFlowTok}[1]{\textcolor[rgb]{0.00,0.44,0.13}{\textbf{{#1}}}}
    \newcommand{\OperatorTok}[1]{\textcolor[rgb]{0.40,0.40,0.40}{{#1}}}
    \newcommand{\BuiltInTok}[1]{{#1}}
    \newcommand{\ExtensionTok}[1]{{#1}}
    \newcommand{\PreprocessorTok}[1]{\textcolor[rgb]{0.74,0.48,0.00}{{#1}}}
    \newcommand{\AttributeTok}[1]{\textcolor[rgb]{0.49,0.56,0.16}{{#1}}}
    \newcommand{\InformationTok}[1]{\textcolor[rgb]{0.38,0.63,0.69}{\textbf{\textit{{#1}}}}}
    \newcommand{\WarningTok}[1]{\textcolor[rgb]{0.38,0.63,0.69}{\textbf{\textit{{#1}}}}}
    
    
    % Define a nice break command that doesn't care if a line doesn't already
    % exist.
    \def\br{\hspace*{\fill} \\* }
    % Math Jax compatibility definitions
    \def\gt{>}
    \def\lt{<}
    \let\Oldtex\TeX
    \let\Oldlatex\LaTeX
    \renewcommand{\TeX}{\textrm{\Oldtex}}
    \renewcommand{\LaTeX}{\textrm{\Oldlatex}}
    % Document parameters
    % Document title
    \title{derivate}
    
    
    
    
    
% Pygments definitions
\makeatletter
\def\PY@reset{\let\PY@it=\relax \let\PY@bf=\relax%
    \let\PY@ul=\relax \let\PY@tc=\relax%
    \let\PY@bc=\relax \let\PY@ff=\relax}
\def\PY@tok#1{\csname PY@tok@#1\endcsname}
\def\PY@toks#1+{\ifx\relax#1\empty\else%
    \PY@tok{#1}\expandafter\PY@toks\fi}
\def\PY@do#1{\PY@bc{\PY@tc{\PY@ul{%
    \PY@it{\PY@bf{\PY@ff{#1}}}}}}}
\def\PY#1#2{\PY@reset\PY@toks#1+\relax+\PY@do{#2}}

\expandafter\def\csname PY@tok@w\endcsname{\def\PY@tc##1{\textcolor[rgb]{0.73,0.73,0.73}{##1}}}
\expandafter\def\csname PY@tok@c\endcsname{\let\PY@it=\textit\def\PY@tc##1{\textcolor[rgb]{0.25,0.50,0.50}{##1}}}
\expandafter\def\csname PY@tok@cp\endcsname{\def\PY@tc##1{\textcolor[rgb]{0.74,0.48,0.00}{##1}}}
\expandafter\def\csname PY@tok@k\endcsname{\let\PY@bf=\textbf\def\PY@tc##1{\textcolor[rgb]{0.00,0.50,0.00}{##1}}}
\expandafter\def\csname PY@tok@kp\endcsname{\def\PY@tc##1{\textcolor[rgb]{0.00,0.50,0.00}{##1}}}
\expandafter\def\csname PY@tok@kt\endcsname{\def\PY@tc##1{\textcolor[rgb]{0.69,0.00,0.25}{##1}}}
\expandafter\def\csname PY@tok@o\endcsname{\def\PY@tc##1{\textcolor[rgb]{0.40,0.40,0.40}{##1}}}
\expandafter\def\csname PY@tok@ow\endcsname{\let\PY@bf=\textbf\def\PY@tc##1{\textcolor[rgb]{0.67,0.13,1.00}{##1}}}
\expandafter\def\csname PY@tok@nb\endcsname{\def\PY@tc##1{\textcolor[rgb]{0.00,0.50,0.00}{##1}}}
\expandafter\def\csname PY@tok@nf\endcsname{\def\PY@tc##1{\textcolor[rgb]{0.00,0.00,1.00}{##1}}}
\expandafter\def\csname PY@tok@nc\endcsname{\let\PY@bf=\textbf\def\PY@tc##1{\textcolor[rgb]{0.00,0.00,1.00}{##1}}}
\expandafter\def\csname PY@tok@nn\endcsname{\let\PY@bf=\textbf\def\PY@tc##1{\textcolor[rgb]{0.00,0.00,1.00}{##1}}}
\expandafter\def\csname PY@tok@ne\endcsname{\let\PY@bf=\textbf\def\PY@tc##1{\textcolor[rgb]{0.82,0.25,0.23}{##1}}}
\expandafter\def\csname PY@tok@nv\endcsname{\def\PY@tc##1{\textcolor[rgb]{0.10,0.09,0.49}{##1}}}
\expandafter\def\csname PY@tok@no\endcsname{\def\PY@tc##1{\textcolor[rgb]{0.53,0.00,0.00}{##1}}}
\expandafter\def\csname PY@tok@nl\endcsname{\def\PY@tc##1{\textcolor[rgb]{0.63,0.63,0.00}{##1}}}
\expandafter\def\csname PY@tok@ni\endcsname{\let\PY@bf=\textbf\def\PY@tc##1{\textcolor[rgb]{0.60,0.60,0.60}{##1}}}
\expandafter\def\csname PY@tok@na\endcsname{\def\PY@tc##1{\textcolor[rgb]{0.49,0.56,0.16}{##1}}}
\expandafter\def\csname PY@tok@nt\endcsname{\let\PY@bf=\textbf\def\PY@tc##1{\textcolor[rgb]{0.00,0.50,0.00}{##1}}}
\expandafter\def\csname PY@tok@nd\endcsname{\def\PY@tc##1{\textcolor[rgb]{0.67,0.13,1.00}{##1}}}
\expandafter\def\csname PY@tok@s\endcsname{\def\PY@tc##1{\textcolor[rgb]{0.73,0.13,0.13}{##1}}}
\expandafter\def\csname PY@tok@sd\endcsname{\let\PY@it=\textit\def\PY@tc##1{\textcolor[rgb]{0.73,0.13,0.13}{##1}}}
\expandafter\def\csname PY@tok@si\endcsname{\let\PY@bf=\textbf\def\PY@tc##1{\textcolor[rgb]{0.73,0.40,0.53}{##1}}}
\expandafter\def\csname PY@tok@se\endcsname{\let\PY@bf=\textbf\def\PY@tc##1{\textcolor[rgb]{0.73,0.40,0.13}{##1}}}
\expandafter\def\csname PY@tok@sr\endcsname{\def\PY@tc##1{\textcolor[rgb]{0.73,0.40,0.53}{##1}}}
\expandafter\def\csname PY@tok@ss\endcsname{\def\PY@tc##1{\textcolor[rgb]{0.10,0.09,0.49}{##1}}}
\expandafter\def\csname PY@tok@sx\endcsname{\def\PY@tc##1{\textcolor[rgb]{0.00,0.50,0.00}{##1}}}
\expandafter\def\csname PY@tok@m\endcsname{\def\PY@tc##1{\textcolor[rgb]{0.40,0.40,0.40}{##1}}}
\expandafter\def\csname PY@tok@gh\endcsname{\let\PY@bf=\textbf\def\PY@tc##1{\textcolor[rgb]{0.00,0.00,0.50}{##1}}}
\expandafter\def\csname PY@tok@gu\endcsname{\let\PY@bf=\textbf\def\PY@tc##1{\textcolor[rgb]{0.50,0.00,0.50}{##1}}}
\expandafter\def\csname PY@tok@gd\endcsname{\def\PY@tc##1{\textcolor[rgb]{0.63,0.00,0.00}{##1}}}
\expandafter\def\csname PY@tok@gi\endcsname{\def\PY@tc##1{\textcolor[rgb]{0.00,0.63,0.00}{##1}}}
\expandafter\def\csname PY@tok@gr\endcsname{\def\PY@tc##1{\textcolor[rgb]{1.00,0.00,0.00}{##1}}}
\expandafter\def\csname PY@tok@ge\endcsname{\let\PY@it=\textit}
\expandafter\def\csname PY@tok@gs\endcsname{\let\PY@bf=\textbf}
\expandafter\def\csname PY@tok@gp\endcsname{\let\PY@bf=\textbf\def\PY@tc##1{\textcolor[rgb]{0.00,0.00,0.50}{##1}}}
\expandafter\def\csname PY@tok@go\endcsname{\def\PY@tc##1{\textcolor[rgb]{0.53,0.53,0.53}{##1}}}
\expandafter\def\csname PY@tok@gt\endcsname{\def\PY@tc##1{\textcolor[rgb]{0.00,0.27,0.87}{##1}}}
\expandafter\def\csname PY@tok@err\endcsname{\def\PY@bc##1{\setlength{\fboxsep}{0pt}\fcolorbox[rgb]{1.00,0.00,0.00}{1,1,1}{\strut ##1}}}
\expandafter\def\csname PY@tok@kc\endcsname{\let\PY@bf=\textbf\def\PY@tc##1{\textcolor[rgb]{0.00,0.50,0.00}{##1}}}
\expandafter\def\csname PY@tok@kd\endcsname{\let\PY@bf=\textbf\def\PY@tc##1{\textcolor[rgb]{0.00,0.50,0.00}{##1}}}
\expandafter\def\csname PY@tok@kn\endcsname{\let\PY@bf=\textbf\def\PY@tc##1{\textcolor[rgb]{0.00,0.50,0.00}{##1}}}
\expandafter\def\csname PY@tok@kr\endcsname{\let\PY@bf=\textbf\def\PY@tc##1{\textcolor[rgb]{0.00,0.50,0.00}{##1}}}
\expandafter\def\csname PY@tok@bp\endcsname{\def\PY@tc##1{\textcolor[rgb]{0.00,0.50,0.00}{##1}}}
\expandafter\def\csname PY@tok@fm\endcsname{\def\PY@tc##1{\textcolor[rgb]{0.00,0.00,1.00}{##1}}}
\expandafter\def\csname PY@tok@vc\endcsname{\def\PY@tc##1{\textcolor[rgb]{0.10,0.09,0.49}{##1}}}
\expandafter\def\csname PY@tok@vg\endcsname{\def\PY@tc##1{\textcolor[rgb]{0.10,0.09,0.49}{##1}}}
\expandafter\def\csname PY@tok@vi\endcsname{\def\PY@tc##1{\textcolor[rgb]{0.10,0.09,0.49}{##1}}}
\expandafter\def\csname PY@tok@vm\endcsname{\def\PY@tc##1{\textcolor[rgb]{0.10,0.09,0.49}{##1}}}
\expandafter\def\csname PY@tok@sa\endcsname{\def\PY@tc##1{\textcolor[rgb]{0.73,0.13,0.13}{##1}}}
\expandafter\def\csname PY@tok@sb\endcsname{\def\PY@tc##1{\textcolor[rgb]{0.73,0.13,0.13}{##1}}}
\expandafter\def\csname PY@tok@sc\endcsname{\def\PY@tc##1{\textcolor[rgb]{0.73,0.13,0.13}{##1}}}
\expandafter\def\csname PY@tok@dl\endcsname{\def\PY@tc##1{\textcolor[rgb]{0.73,0.13,0.13}{##1}}}
\expandafter\def\csname PY@tok@s2\endcsname{\def\PY@tc##1{\textcolor[rgb]{0.73,0.13,0.13}{##1}}}
\expandafter\def\csname PY@tok@sh\endcsname{\def\PY@tc##1{\textcolor[rgb]{0.73,0.13,0.13}{##1}}}
\expandafter\def\csname PY@tok@s1\endcsname{\def\PY@tc##1{\textcolor[rgb]{0.73,0.13,0.13}{##1}}}
\expandafter\def\csname PY@tok@mb\endcsname{\def\PY@tc##1{\textcolor[rgb]{0.40,0.40,0.40}{##1}}}
\expandafter\def\csname PY@tok@mf\endcsname{\def\PY@tc##1{\textcolor[rgb]{0.40,0.40,0.40}{##1}}}
\expandafter\def\csname PY@tok@mh\endcsname{\def\PY@tc##1{\textcolor[rgb]{0.40,0.40,0.40}{##1}}}
\expandafter\def\csname PY@tok@mi\endcsname{\def\PY@tc##1{\textcolor[rgb]{0.40,0.40,0.40}{##1}}}
\expandafter\def\csname PY@tok@il\endcsname{\def\PY@tc##1{\textcolor[rgb]{0.40,0.40,0.40}{##1}}}
\expandafter\def\csname PY@tok@mo\endcsname{\def\PY@tc##1{\textcolor[rgb]{0.40,0.40,0.40}{##1}}}
\expandafter\def\csname PY@tok@ch\endcsname{\let\PY@it=\textit\def\PY@tc##1{\textcolor[rgb]{0.25,0.50,0.50}{##1}}}
\expandafter\def\csname PY@tok@cm\endcsname{\let\PY@it=\textit\def\PY@tc##1{\textcolor[rgb]{0.25,0.50,0.50}{##1}}}
\expandafter\def\csname PY@tok@cpf\endcsname{\let\PY@it=\textit\def\PY@tc##1{\textcolor[rgb]{0.25,0.50,0.50}{##1}}}
\expandafter\def\csname PY@tok@c1\endcsname{\let\PY@it=\textit\def\PY@tc##1{\textcolor[rgb]{0.25,0.50,0.50}{##1}}}
\expandafter\def\csname PY@tok@cs\endcsname{\let\PY@it=\textit\def\PY@tc##1{\textcolor[rgb]{0.25,0.50,0.50}{##1}}}

\def\PYZbs{\char`\\}
\def\PYZus{\char`\_}
\def\PYZob{\char`\{}
\def\PYZcb{\char`\}}
\def\PYZca{\char`\^}
\def\PYZam{\char`\&}
\def\PYZlt{\char`\<}
\def\PYZgt{\char`\>}
\def\PYZsh{\char`\#}
\def\PYZpc{\char`\%}
\def\PYZdl{\char`\$}
\def\PYZhy{\char`\-}
\def\PYZsq{\char`\'}
\def\PYZdq{\char`\"}
\def\PYZti{\char`\~}
% for compatibility with earlier versions
\def\PYZat{@}
\def\PYZlb{[}
\def\PYZrb{]}
\makeatother


    % For linebreaks inside Verbatim environment from package fancyvrb. 
    \makeatletter
        \newbox\Wrappedcontinuationbox 
        \newbox\Wrappedvisiblespacebox 
        \newcommand*\Wrappedvisiblespace {\textcolor{red}{\textvisiblespace}} 
        \newcommand*\Wrappedcontinuationsymbol {\textcolor{red}{\llap{\tiny$\m@th\hookrightarrow$}}} 
        \newcommand*\Wrappedcontinuationindent {3ex } 
        \newcommand*\Wrappedafterbreak {\kern\Wrappedcontinuationindent\copy\Wrappedcontinuationbox} 
        % Take advantage of the already applied Pygments mark-up to insert 
        % potential linebreaks for TeX processing. 
        %        {, <, #, %, $, ' and ": go to next line. 
        %        _, }, ^, &, >, - and ~: stay at end of broken line. 
        % Use of \textquotesingle for straight quote. 
        \newcommand*\Wrappedbreaksatspecials {% 
            \def\PYGZus{\discretionary{\char`\_}{\Wrappedafterbreak}{\char`\_}}% 
            \def\PYGZob{\discretionary{}{\Wrappedafterbreak\char`\{}{\char`\{}}% 
            \def\PYGZcb{\discretionary{\char`\}}{\Wrappedafterbreak}{\char`\}}}% 
            \def\PYGZca{\discretionary{\char`\^}{\Wrappedafterbreak}{\char`\^}}% 
            \def\PYGZam{\discretionary{\char`\&}{\Wrappedafterbreak}{\char`\&}}% 
            \def\PYGZlt{\discretionary{}{\Wrappedafterbreak\char`\<}{\char`\<}}% 
            \def\PYGZgt{\discretionary{\char`\>}{\Wrappedafterbreak}{\char`\>}}% 
            \def\PYGZsh{\discretionary{}{\Wrappedafterbreak\char`\#}{\char`\#}}% 
            \def\PYGZpc{\discretionary{}{\Wrappedafterbreak\char`\%}{\char`\%}}% 
            \def\PYGZdl{\discretionary{}{\Wrappedafterbreak\char`\$}{\char`\$}}% 
            \def\PYGZhy{\discretionary{\char`\-}{\Wrappedafterbreak}{\char`\-}}% 
            \def\PYGZsq{\discretionary{}{\Wrappedafterbreak\textquotesingle}{\textquotesingle}}% 
            \def\PYGZdq{\discretionary{}{\Wrappedafterbreak\char`\"}{\char`\"}}% 
            \def\PYGZti{\discretionary{\char`\~}{\Wrappedafterbreak}{\char`\~}}% 
        } 
        % Some characters . , ; ? ! / are not pygmentized. 
        % This macro makes them "active" and they will insert potential linebreaks 
        \newcommand*\Wrappedbreaksatpunct {% 
            \lccode`\~`\.\lowercase{\def~}{\discretionary{\hbox{\char`\.}}{\Wrappedafterbreak}{\hbox{\char`\.}}}% 
            \lccode`\~`\,\lowercase{\def~}{\discretionary{\hbox{\char`\,}}{\Wrappedafterbreak}{\hbox{\char`\,}}}% 
            \lccode`\~`\;\lowercase{\def~}{\discretionary{\hbox{\char`\;}}{\Wrappedafterbreak}{\hbox{\char`\;}}}% 
            \lccode`\~`\:\lowercase{\def~}{\discretionary{\hbox{\char`\:}}{\Wrappedafterbreak}{\hbox{\char`\:}}}% 
            \lccode`\~`\?\lowercase{\def~}{\discretionary{\hbox{\char`\?}}{\Wrappedafterbreak}{\hbox{\char`\?}}}% 
            \lccode`\~`\!\lowercase{\def~}{\discretionary{\hbox{\char`\!}}{\Wrappedafterbreak}{\hbox{\char`\!}}}% 
            \lccode`\~`\/\lowercase{\def~}{\discretionary{\hbox{\char`\/}}{\Wrappedafterbreak}{\hbox{\char`\/}}}% 
            \catcode`\.\active
            \catcode`\,\active 
            \catcode`\;\active
            \catcode`\:\active
            \catcode`\?\active
            \catcode`\!\active
            \catcode`\/\active 
            \lccode`\~`\~ 	
        }
    \makeatother

    \let\OriginalVerbatim=\Verbatim
    \makeatletter
    \renewcommand{\Verbatim}[1][1]{%
        %\parskip\z@skip
        \sbox\Wrappedcontinuationbox {\Wrappedcontinuationsymbol}%
        \sbox\Wrappedvisiblespacebox {\FV@SetupFont\Wrappedvisiblespace}%
        \def\FancyVerbFormatLine ##1{\hsize\linewidth
            \vtop{\raggedright\hyphenpenalty\z@\exhyphenpenalty\z@
                \doublehyphendemerits\z@\finalhyphendemerits\z@
                \strut ##1\strut}%
        }%
        % If the linebreak is at a space, the latter will be displayed as visible
        % space at end of first line, and a continuation symbol starts next line.
        % Stretch/shrink are however usually zero for typewriter font.
        \def\FV@Space {%
            \nobreak\hskip\z@ plus\fontdimen3\font minus\fontdimen4\font
            \discretionary{\copy\Wrappedvisiblespacebox}{\Wrappedafterbreak}
            {\kern\fontdimen2\font}%
        }%
        
        % Allow breaks at special characters using \PYG... macros.
        \Wrappedbreaksatspecials
        % Breaks at punctuation characters . , ; ? ! and / need catcode=\active 	
        \OriginalVerbatim[#1,codes*=\Wrappedbreaksatpunct]%
    }
    \makeatother

    % Exact colors from NB
    \definecolor{incolor}{HTML}{303F9F}
    \definecolor{outcolor}{HTML}{D84315}
    \definecolor{cellborder}{HTML}{CFCFCF}
    \definecolor{cellbackground}{HTML}{F7F7F7}
    
    % prompt
    \makeatletter
    \newcommand{\boxspacing}{\kern\kvtcb@left@rule\kern\kvtcb@boxsep}
    \makeatother
    \newcommand{\prompt}[4]{
        \ttfamily\llap{{\color{#2}[#3]:\hspace{3pt}#4}}\vspace{-\baselineskip}
    }
    

    
    % Prevent overflowing lines due to hard-to-break entities
    \sloppy 
    % Setup hyperref package
    \hypersetup{
      breaklinks=true,  % so long urls are correctly broken across lines
      colorlinks=true,
      urlcolor=urlcolor,
      linkcolor=linkcolor,
      citecolor=citecolor,
      }
    % Slightly bigger margins than the latex defaults
    
    \geometry{verbose,tmargin=1in,bmargin=1in,lmargin=1in,rmargin=1in}
    
    

\begin{document}
    
    \maketitle
    
    

    
    \hypertarget{python-for-finance-im-mastermodul-termingeschuxe4fte-und-finanzderivate---teil-1}{%
\section{Python for Finance im Mastermodul ``Termingeschäfte und
Finanzderivate'' - Teil
1}\label{python-for-finance-im-mastermodul-termingeschuxe4fte-und-finanzderivate---teil-1}}

\begin{center}\rule{0.5\linewidth}{\linethickness}\end{center}

Dieses Jupyter Notebook basiert auf den Beispielen aus ``Python for
Finance - Second Edition'' von Yuxing Yan:
https://www.packtpub.com/big-data-and-business-intelligence/python-finance-second-edition
Sämtliche Beispiele sind in leicht abgewandeltet Form zu finden unter:
https://github.com/PacktPublishing/Python-for-Finance-Second-Edition

\hypertarget{urheberrechtsinformationen}{%
\subsubsection{Urheberrechtsinformationen:}\label{urheberrechtsinformationen}}

MIT License

Copyright (c) 2017 Packt

Permission is hereby granted, free of charge, to any person obtaining a
copy of this software and associated documentation files (the
``Software''), to deal in the Software without restriction, including
without limitation the rights to use, copy, modify, merge, publish,
distribute, sublicense, and/or sell copies of the Software, and to
permit persons to whom the Software is furnished to do so, subject to
the following conditions:

The above copyright notice and this permission notice shall be included
in all copies or substantial portions of the Software.

THE SOFTWARE IS PROVIDED ``AS IS'', WITHOUT WARRANTY OF ANY KIND,
EXPRESS OR IMPLIED, INCLUDING BUT NOT LIMITED TO THE WARRANTIES OF
MERCHANTABILITY, FITNESS FOR A PARTICULAR PURPOSE AND NONINFRINGEMENT.
IN NO EVENT SHALL THE AUTHORS OR COPYRIGHT HOLDERS BE LIABLE FOR ANY
CLAIM, DAMAGES OR OTHER LIABILITY, WHETHER IN AN ACTION OF CONTRACT,
TORT OR OTHERWISE, ARISING FROM, OUT OF OR IN CONNECTION WITH THE
SOFTWARE OR THE USE OR OTHER DEALINGS IN THE SOFTWARE. \_\_\_ Version
1.0.1 Weitere Projekte sind zu finden unter: https://github.com/trh0ly

    \hypertarget{grundlegende-einstellungen}{%
\subsection{Grundlegende
Einstellungen:}\label{grundlegende-einstellungen}}

Zunächst müssen die notwendigen Pakete (auch Module) importiert werden,
damit auf diese zugegriffen werden kann.

    \begin{tcolorbox}[breakable, size=fbox, boxrule=1pt, pad at break*=1mm,colback=cellbackground, colframe=cellborder]
\prompt{In}{incolor}{1}{\boxspacing}
\begin{Verbatim}[commandchars=\\\{\}]
\PY{k+kn}{import} \PY{n+nn}{numpy} \PY{k}{as} \PY{n+nn}{np} \PY{c+c1}{\PYZsh{} Programmbibliothek die eine einfache Handhabung von Vektoren, Matrizen oder generell großen mehrdimensionalen Arrays ermöglicht}
\PY{k+kn}{import} \PY{n+nn}{matplotlib}\PY{n+nn}{.}\PY{n+nn}{pyplot} \PY{k}{as} \PY{n+nn}{plt} \PY{c+c1}{\PYZsh{} Programmbibliothek die es erlaubt mathematische Darstellungen aller Art anzufertigen}
\PY{k+kn}{import} \PY{n+nn}{math} \PY{c+c1}{\PYZsh{} Dieses Modul wird verwendet um Skalardaten zu berechnen, z. B. trigonometrische Berechnungen.}
\PY{k+kn}{import} \PY{n+nn}{pylab} \PY{k}{as} \PY{n+nn}{pl} \PY{c+c1}{\PYZsh{} Pylab kombiniert die Packete PyPlot und Numpy}
\PY{k+kn}{import} \PY{n+nn}{pandas} \PY{k}{as} \PY{n+nn}{pd} \PY{c+c1}{\PYZsh{} Programmbibliothek die Hilfsmittel für die Verwaltung von Daten und deren Analyse anbietet}
\PY{k+kn}{import} \PY{n+nn}{scipy} \PY{k}{as} \PY{n+nn}{sp} \PY{c+c1}{\PYZsh{} SciPy ist ein Python\PYZhy{}basiertes Ökosystem für Open\PYZhy{}Source\PYZhy{}Software für Mathematik, Naturwissenschaften und Ingenieurwissenschaften}
\PY{k+kn}{from} \PY{n+nn}{scipy} \PY{k}{import} \PY{n}{stats}
\PY{k+kn}{import} \PY{n+nn}{datetime} \PY{c+c1}{\PYZsh{} Das datetime\PYZhy{}Modul stellt Klassen bereit, mit denen Datums\PYZhy{} und Uhrzeitangaben auf einfache und komplexe Weise bearbeitet werden können}
\PY{k+kn}{import} \PY{n+nn}{operator} \PY{c+c1}{\PYZsh{} Programmbibliothek, welche die Ausgaben übersichtlicher gestaltet}
\PY{k+kn}{import} \PY{n+nn}{yfinance} \PY{k}{as} \PY{n+nn}{yf} \PY{c+c1}{\PYZsh{} Dieses Modul ermöglicht den Download von zuverlässigen (historischen) Marktdaten von Yahoo Finance }
\PY{k+kn}{from} \PY{n+nn}{alternative\PYZus{}plot} \PY{k}{import} \PY{n}{alternative\PYZus{}smile} \PY{k}{as} \PY{n}{asplt}
\PY{k+kn}{import} \PY{n+nn}{sys}
\end{Verbatim}
\end{tcolorbox}

    
    \begin{verbatim}
<Figure size 640x480 with 1 Axes>
    \end{verbatim}

    
    Anschließend werden Einstellungen definiert, die die Formatierung der
Ausgaben betreffen. Hierfür wird das Modul \texttt{operator} genutzt.
Außerdem wird die Breite des im Folgenden genutzten DataFrames erhöht
und die Größe der Grafiken modifiziert, welche später angezeigt werden
sollen.

    \begin{tcolorbox}[breakable, size=fbox, boxrule=1pt, pad at break*=1mm,colback=cellbackground, colframe=cellborder]
\prompt{In}{incolor}{2}{\boxspacing}
\begin{Verbatim}[commandchars=\\\{\}]
\PY{o}{\PYZpc{}\PYZpc{}javascript}
\PY{n+nx}{IPython}\PY{p}{.}\PY{n+nx}{OutputArea}\PY{p}{.}\PY{n+nx}{auto\PYZus{}scroll\PYZus{}threshold} \PY{o}{=} \PY{l+m+mi}{9999}\PY{p}{;}
\end{Verbatim}
\end{tcolorbox}

    
    \begin{verbatim}
<IPython.core.display.Javascript object>
    \end{verbatim}

    
    \begin{tcolorbox}[breakable, size=fbox, boxrule=1pt, pad at break*=1mm,colback=cellbackground, colframe=cellborder]
\prompt{In}{incolor}{3}{\boxspacing}
\begin{Verbatim}[commandchars=\\\{\}]
\PY{k+kn}{from} \PY{n+nn}{IPython}\PY{n+nn}{.}\PY{n+nn}{core}\PY{n+nn}{.}\PY{n+nn}{display} \PY{k}{import} \PY{n}{display}\PY{p}{,} \PY{n}{HTML}
\PY{n}{display}\PY{p}{(}\PY{n}{HTML}\PY{p}{(}\PY{l+s+s2}{\PYZdq{}}\PY{l+s+s2}{\PYZlt{}style\PYZgt{}.container }\PY{l+s+s2}{\PYZob{}}\PY{l+s+s2}{ width:100}\PY{l+s+s2}{\PYZpc{}}\PY{l+s+s2}{ !important; \PYZcb{}\PYZlt{}/style\PYZgt{}}\PY{l+s+s2}{\PYZdq{}}\PY{p}{)}\PY{p}{)}

\PY{c+c1}{\PYZsh{}\PYZsh{}\PYZsh{}\PYZsh{}\PYZsh{}\PYZsh{}\PYZsh{}\PYZsh{}\PYZsh{}\PYZsh{}\PYZsh{}\PYZsh{}\PYZsh{}\PYZsh{}\PYZsh{}\PYZsh{}\PYZsh{}\PYZsh{}\PYZsh{}\PYZsh{}\PYZsh{}\PYZsh{}\PYZsh{}\PYZsh{}\PYZsh{}\PYZsh{}\PYZsh{}\PYZsh{}\PYZsh{}\PYZsh{}\PYZsh{}\PYZsh{}\PYZsh{}\PYZsh{}\PYZsh{}\PYZsh{}\PYZsh{}\PYZsh{}\PYZsh{}\PYZsh{}\PYZsh{}\PYZsh{}\PYZsh{}\PYZsh{}\PYZsh{}\PYZsh{}\PYZsh{}\PYZsh{}\PYZsh{}\PYZsh{}\PYZsh{}\PYZsh{}\PYZsh{}\PYZsh{}\PYZsh{}\PYZsh{}\PYZsh{}\PYZsh{}\PYZsh{}\PYZsh{}\PYZsh{}\PYZsh{}\PYZsh{}\PYZsh{}\PYZsh{}\PYZsh{}\PYZsh{}\PYZsh{}\PYZsh{}\PYZsh{}\PYZsh{}\PYZsh{}\PYZsh{}\PYZsh{}}
\PY{c+c1}{\PYZsh{}\PYZhy{}\PYZhy{}\PYZhy{}\PYZhy{}\PYZhy{}\PYZhy{}\PYZhy{}\PYZhy{}\PYZhy{}\PYZhy{}\PYZhy{}\PYZhy{}\PYZhy{}\PYZhy{}\PYZhy{}\PYZhy{}\PYZhy{}\PYZhy{}\PYZhy{}\PYZhy{}\PYZhy{}\PYZhy{}\PYZhy{}\PYZhy{}\PYZhy{}\PYZhy{}\PYZhy{}\PYZhy{}\PYZhy{}\PYZhy{}\PYZhy{}\PYZhy{}\PYZhy{}\PYZhy{}\PYZhy{}\PYZhy{}\PYZhy{}\PYZhy{}\PYZhy{}\PYZhy{}\PYZhy{}\PYZhy{}\PYZhy{}\PYZhy{}\PYZhy{}\PYZhy{}\PYZhy{}\PYZhy{}\PYZhy{}\PYZhy{}\PYZhy{}\PYZhy{}\PYZhy{}\PYZhy{}\PYZhy{}\PYZhy{}\PYZhy{}\PYZhy{}\PYZhy{}\PYZhy{}\PYZhy{}\PYZhy{}\PYZhy{}\PYZhy{}\PYZhy{}\PYZhy{}\PYZhy{}\PYZhy{}\PYZhy{}\PYZhy{}\PYZhy{}\PYZhy{}\PYZhy{}}

\PY{n}{SCREEN\PYZus{}WIDTH} \PY{o}{=} \PY{l+m+mi}{125} \PY{c+c1}{\PYZsh{} Breite Outputbox}
\PY{n}{SIZE} \PY{o}{=} \PY{p}{[}\PY{l+m+mi}{15}\PY{p}{,}\PY{l+m+mi}{10}\PY{p}{]} \PY{c+c1}{\PYZsh{} Größe Grafiken / Textgröße}

\PY{c+c1}{\PYZsh{}\PYZhy{}\PYZhy{}\PYZhy{}\PYZhy{}\PYZhy{}\PYZhy{}\PYZhy{}\PYZhy{}\PYZhy{}\PYZhy{}\PYZhy{}\PYZhy{}\PYZhy{}\PYZhy{}\PYZhy{}\PYZhy{}\PYZhy{}\PYZhy{}\PYZhy{}\PYZhy{}\PYZhy{}\PYZhy{}\PYZhy{}\PYZhy{}\PYZhy{}\PYZhy{}\PYZhy{}\PYZhy{}\PYZhy{}\PYZhy{}\PYZhy{}\PYZhy{}\PYZhy{}\PYZhy{}\PYZhy{}\PYZhy{}\PYZhy{}\PYZhy{}\PYZhy{}\PYZhy{}\PYZhy{}\PYZhy{}\PYZhy{}\PYZhy{}\PYZhy{}\PYZhy{}\PYZhy{}\PYZhy{}\PYZhy{}\PYZhy{}\PYZhy{}\PYZhy{}\PYZhy{}\PYZhy{}\PYZhy{}\PYZhy{}\PYZhy{}\PYZhy{}\PYZhy{}\PYZhy{}\PYZhy{}\PYZhy{}\PYZhy{}\PYZhy{}\PYZhy{}\PYZhy{}\PYZhy{}\PYZhy{}\PYZhy{}\PYZhy{}\PYZhy{}\PYZhy{}\PYZhy{}}
\PY{c+c1}{\PYZsh{}\PYZsh{}\PYZsh{}\PYZsh{}\PYZsh{}\PYZsh{}\PYZsh{}\PYZsh{}\PYZsh{}\PYZsh{}\PYZsh{}\PYZsh{}\PYZsh{}\PYZsh{}\PYZsh{}\PYZsh{}\PYZsh{}\PYZsh{}\PYZsh{}\PYZsh{}\PYZsh{}\PYZsh{}\PYZsh{}\PYZsh{}\PYZsh{}\PYZsh{}\PYZsh{}\PYZsh{}\PYZsh{}\PYZsh{}\PYZsh{}\PYZsh{}\PYZsh{}\PYZsh{}\PYZsh{}\PYZsh{}\PYZsh{}\PYZsh{}\PYZsh{}\PYZsh{}\PYZsh{}\PYZsh{}\PYZsh{}\PYZsh{}\PYZsh{}\PYZsh{}\PYZsh{}\PYZsh{}\PYZsh{}\PYZsh{}\PYZsh{}\PYZsh{}\PYZsh{}\PYZsh{}\PYZsh{}\PYZsh{}\PYZsh{}\PYZsh{}\PYZsh{}\PYZsh{}\PYZsh{}\PYZsh{}\PYZsh{}\PYZsh{}\PYZsh{}\PYZsh{}\PYZsh{}\PYZsh{}\PYZsh{}\PYZsh{}\PYZsh{}\PYZsh{}\PYZsh{}\PYZsh{}}

\PY{n}{pd}\PY{o}{.}\PY{n}{set\PYZus{}option}\PY{p}{(}\PY{l+s+s1}{\PYZsq{}}\PY{l+s+s1}{display.width}\PY{l+s+s1}{\PYZsq{}}\PY{p}{,} \PY{n}{SCREEN\PYZus{}WIDTH}\PY{p}{)}
\PY{n}{centered} \PY{o}{=} \PY{n}{operator}\PY{o}{.}\PY{n}{methodcaller}\PY{p}{(}\PY{l+s+s1}{\PYZsq{}}\PY{l+s+s1}{center}\PY{l+s+s1}{\PYZsq{}}\PY{p}{,} \PY{n}{SCREEN\PYZus{}WIDTH}\PY{p}{)}
\PY{n}{plt}\PY{o}{.}\PY{n}{rcParams}\PY{p}{[}\PY{l+s+s2}{\PYZdq{}}\PY{l+s+s2}{figure.figsize}\PY{l+s+s2}{\PYZdq{}}\PY{p}{]} \PY{o}{=} \PY{n}{SIZE}\PY{p}{[}\PY{l+m+mi}{0}\PY{p}{]}\PY{p}{,}\PY{n}{SIZE}\PY{p}{[}\PY{l+m+mi}{1}\PY{p}{]} 
\PY{n}{plt}\PY{o}{.}\PY{n}{rcParams}\PY{o}{.}\PY{n}{update}\PY{p}{(}\PY{p}{\PYZob{}}\PY{l+s+s1}{\PYZsq{}}\PY{l+s+s1}{font.size}\PY{l+s+s1}{\PYZsq{}}\PY{p}{:} \PY{n}{SIZE}\PY{p}{[}\PY{l+m+mi}{0}\PY{p}{]}\PY{p}{\PYZcb{}}\PY{p}{)}
\end{Verbatim}
\end{tcolorbox}

    
    \begin{verbatim}
<IPython.core.display.HTML object>
    \end{verbatim}

    
    \hypertarget{aufgabe-1---payoff--und-gewinn-verlustfunktionen-fuxfcr-call--und-put-optionen}{%
\subsection{Aufgabe 1 - Payoff- und Gewinn-/Verlustfunktionen für Call-
und
Put-Optionen}\label{aufgabe-1---payoff--und-gewinn-verlustfunktionen-fuxfcr-call--und-put-optionen}}

Variablenverzeichnis: - s = Preis der Aktie zum Fälligkeitsdatum - b =
Ausübungspreis (Basispreis) - c = Options-Prämie für einen Call - p =
Options-Prämie für einen Put

\hypertarget{bestimmung-des-payoffs-einer-call-option}{%
\subsubsection{1.1) Bestimmung des Payoffs einer
Call-Option}\label{bestimmung-des-payoffs-einer-call-option}}

Annahme: Der Ausübungspreis (Basispreis) \texttt{b} beträgt 30 GE 1.
Fall: Am Zeitpunkt \texttt{t} (Fälligkeit) beträgt der Preis der Aktie
\texttt{s}=25 GE -\textgreater{} Option wird nicht ausgeübt, da die
Aktie am Markt günstiger zu haben ist 1. Fall: Am Zeitpunkt \texttt{t}
(Fälligkeit) beträgt der Preis der Aktie \texttt{s}=40 GE
-\textgreater{} Option wird ausgeübt, da die Aktie am Markt teurer ist

    \begin{tcolorbox}[breakable, size=fbox, boxrule=1pt, pad at break*=1mm,colback=cellbackground, colframe=cellborder]
\prompt{In}{incolor}{4}{\boxspacing}
\begin{Verbatim}[commandchars=\\\{\}]
\PY{c+c1}{\PYZsh{} Definition der Funktion \PYZdq{}payoff\PYZus{}call\PYZdq{}, welche den Payoff einer Call\PYZhy{}Option betimmt}
\PY{k}{def} \PY{n+nf}{payoff\PYZus{}call}\PY{p}{(}\PY{n}{s}\PY{p}{,} \PY{n}{b}\PY{p}{)}\PY{p}{:}    
    \PY{k}{return} \PY{p}{(}\PY{n}{s} \PY{o}{\PYZhy{}} \PY{n}{b} \PY{o}{+} \PY{n+nb}{abs}\PY{p}{(}\PY{n}{s} \PY{o}{\PYZhy{}} \PY{n}{b}\PY{p}{)}\PY{p}{)} \PY{o}{/} \PY{l+m+mi}{2}

\PY{c+c1}{\PYZsh{} Definition der Funktion \PYZdq{}payoff\PYZus{}put\PYZdq{}, welche den Payoff einer Put\PYZhy{}Option betimmt}
\PY{k}{def} \PY{n+nf}{payoff\PYZus{}put}\PY{p}{(}\PY{n}{s}\PY{p}{,} \PY{n}{b}\PY{p}{)}\PY{p}{:}    
    \PY{k}{return} \PY{p}{(}\PY{n}{b} \PY{o}{\PYZhy{}} \PY{n}{s} \PY{o}{\PYZhy{}} \PY{n+nb}{abs}\PY{p}{(}\PY{n}{s} \PY{o}{\PYZhy{}} \PY{n}{b}\PY{p}{)}\PY{p}{)} \PY{o}{/} \PY{l+m+mi}{2}

\PY{c+c1}{\PYZsh{} Anwendung der Funktionen mit den Parametern s=25; s=40 und b=30 und Ausgabe des Ergebnisses}
\PY{n+nb}{print}\PY{p}{(}\PY{l+s+s1}{\PYZsq{}}\PY{l+s+s1}{\PYZsh{}}\PY{l+s+s1}{\PYZsq{}} \PY{o}{+} \PY{n}{SCREEN\PYZus{}WIDTH} \PY{o}{*} \PY{l+s+s1}{\PYZsq{}}\PY{l+s+s1}{\PYZhy{}}\PY{l+s+s1}{\PYZsq{}} \PY{o}{+} \PY{l+s+s1}{\PYZsq{}}\PY{l+s+s1}{\PYZsh{}}\PY{l+s+s1}{\PYZsq{}}\PY{p}{)}
\PY{n+nb}{print}\PY{p}{(}\PY{l+s+s1}{\PYZsq{}}\PY{l+s+s1}{|}\PY{l+s+s1}{\PYZsq{}} \PY{o}{+} \PY{n}{centered}\PY{p}{(}\PY{l+s+s1}{\PYZsq{}}\PY{l+s+s1}{Der Payoff einer Call\PYZhy{}Option mit den Parametern s=25 und b=30 beträgt: }\PY{l+s+si}{\PYZpc{}d}\PY{l+s+s1}{\PYZsq{}} \PY{o}{\PYZpc{}}\PY{k}{payoff\PYZus{}call}(25,30)) + \PYZsq{}| \PYZsq{})
\PY{n+nb}{print}\PY{p}{(}\PY{l+s+s1}{\PYZsq{}}\PY{l+s+s1}{|}\PY{l+s+s1}{\PYZsq{}} \PY{o}{+} \PY{n}{centered}\PY{p}{(}\PY{l+s+s1}{\PYZsq{}}\PY{l+s+s1}{Der Payoff einer Call\PYZhy{}Option mit den Parametern s=40 und b=30 beträgt: }\PY{l+s+si}{\PYZpc{}d}\PY{l+s+s1}{\PYZsq{}} \PY{o}{\PYZpc{}}\PY{k}{payoff\PYZus{}call}(40,30)) + \PYZsq{}| \PYZsq{})
\PY{n+nb}{print}\PY{p}{(}\PY{l+s+s1}{\PYZsq{}}\PY{l+s+s1}{\PYZsh{}}\PY{l+s+s1}{\PYZsq{}} \PY{o}{+} \PY{n}{SCREEN\PYZus{}WIDTH} \PY{o}{*} \PY{l+s+s1}{\PYZsq{}}\PY{l+s+s1}{\PYZhy{}}\PY{l+s+s1}{\PYZsq{}} \PY{o}{+} \PY{l+s+s1}{\PYZsq{}}\PY{l+s+s1}{\PYZsh{}}\PY{l+s+s1}{\PYZsq{}}\PY{p}{)}
\PY{n+nb}{print}\PY{p}{(}\PY{l+s+s1}{\PYZsq{}}\PY{l+s+s1}{|}\PY{l+s+s1}{\PYZsq{}} \PY{o}{+} \PY{n}{centered}\PY{p}{(}\PY{l+s+s1}{\PYZsq{}}\PY{l+s+s1}{Der Payoff einer Put\PYZhy{}Option mit den Parametern s=25 und b=30 beträgt: }\PY{l+s+si}{\PYZpc{}d}\PY{l+s+s1}{\PYZsq{}} \PY{o}{\PYZpc{}}\PY{k}{payoff\PYZus{}put}(25,30)) + \PYZsq{}| \PYZsq{})
\PY{n+nb}{print}\PY{p}{(}\PY{l+s+s1}{\PYZsq{}}\PY{l+s+s1}{|}\PY{l+s+s1}{\PYZsq{}} \PY{o}{+} \PY{n}{centered}\PY{p}{(}\PY{l+s+s1}{\PYZsq{}}\PY{l+s+s1}{Der Payoff einer Put\PYZhy{}Option mit den Parametern s=40 und b=30 beträgt: }\PY{l+s+si}{\PYZpc{}d}\PY{l+s+s1}{\PYZsq{}} \PY{o}{\PYZpc{}}\PY{k}{payoff\PYZus{}put}(40,30)) + \PYZsq{}| \PYZsq{})
\PY{n+nb}{print}\PY{p}{(}\PY{l+s+s1}{\PYZsq{}}\PY{l+s+s1}{\PYZsh{}}\PY{l+s+s1}{\PYZsq{}} \PY{o}{+} \PY{n}{SCREEN\PYZus{}WIDTH} \PY{o}{*} \PY{l+s+s1}{\PYZsq{}}\PY{l+s+s1}{\PYZhy{}}\PY{l+s+s1}{\PYZsq{}} \PY{o}{+} \PY{l+s+s1}{\PYZsq{}}\PY{l+s+s1}{\PYZsh{}}\PY{l+s+s1}{\PYZsq{}}\PY{p}{)}
\end{Verbatim}
\end{tcolorbox}

    \begin{Verbatim}[commandchars=\\\{\}]
\#-------------------------------------------------------------------------------
----------------------------------------------\#
|                           Der Payoff einer Call-Option mit den Parametern s=25
und b=30 beträgt: 0                          |
|                          Der Payoff einer Call-Option mit den Parametern s=40
und b=30 beträgt: 10                          |
\#-------------------------------------------------------------------------------
----------------------------------------------\#
|                           Der Payoff einer Put-Option mit den Parametern s=25
und b=30 beträgt: 0                           |
|                          Der Payoff einer Put-Option mit den Parametern s=40
und b=30 beträgt: -10                          |
\#-------------------------------------------------------------------------------
----------------------------------------------\#
    \end{Verbatim}

    \hypertarget{bestimmung-des-payoffs-einer-call-option-array}{%
\subsubsection{1.2) Bestimmung des Payoffs einer Call-Option
(Array)}\label{bestimmung-des-payoffs-einer-call-option-array}}

Der Aktienkurs \texttt{s} zum Zeitpunkt \texttt{t} kann auch als ein
Array definiert sein, sodass statt einem Wert nun mehrere zurückgegeben
werden.

    \begin{tcolorbox}[breakable, size=fbox, boxrule=1pt, pad at break*=1mm,colback=cellbackground, colframe=cellborder]
\prompt{In}{incolor}{5}{\boxspacing}
\begin{Verbatim}[commandchars=\\\{\}]
\PY{n}{b} \PY{o}{=} \PY{l+m+mi}{20} \PY{c+c1}{\PYZsh{} Ausübungspreis}
\PY{n}{s} \PY{o}{=} \PY{n}{np}\PY{o}{.}\PY{n}{arange}\PY{p}{(}\PY{l+m+mi}{10}\PY{p}{,}\PY{l+m+mi}{50}\PY{p}{,}\PY{l+m+mi}{10}\PY{p}{)} \PY{c+c1}{\PYZsh{} Array mit möglichen Preisen einer Aktie}

\PY{c+c1}{\PYZsh{} Anwendung der Funktion mit den Parametern s=(10,20,30,40) und b=20 und Ausgabe des Ergebnisses}
\PY{n+nb}{print}\PY{p}{(}\PY{l+s+s1}{\PYZsq{}}\PY{l+s+s1}{\PYZsh{}}\PY{l+s+s1}{\PYZsq{}} \PY{o}{+} \PY{n}{SCREEN\PYZus{}WIDTH} \PY{o}{*} \PY{l+s+s1}{\PYZsq{}}\PY{l+s+s1}{\PYZhy{}}\PY{l+s+s1}{\PYZsq{}} \PY{o}{+} \PY{l+s+s1}{\PYZsq{}}\PY{l+s+s1}{\PYZsh{}}\PY{l+s+s1}{\PYZsq{}}\PY{p}{)}
\PY{n+nb}{print}\PY{p}{(}\PY{l+s+s1}{\PYZsq{}}\PY{l+s+s1}{|}\PY{l+s+s1}{\PYZsq{}} \PY{o}{+} \PY{n}{centered}\PY{p}{(}\PY{l+s+s1}{\PYZsq{}}\PY{l+s+s1}{Die Payoffs der Call\PYZhy{}Option für s=}\PY{l+s+s1}{\PYZsq{}} \PY{o}{+} \PY{n+nb}{str}\PY{p}{(}\PY{n}{s}\PY{p}{)} \PY{o}{+} \PY{l+s+s1}{\PYZsq{}}\PY{l+s+s1}{ und b=20 betragen: }\PY{l+s+s1}{\PYZsq{}} \PY{o}{+} \PY{n+nb}{str}\PY{p}{(}\PY{n}{payoff\PYZus{}call}\PY{p}{(}\PY{n}{s}\PY{p}{,}\PY{n}{b}\PY{p}{)}\PY{p}{)}\PY{p}{)} \PY{o}{+} \PY{l+s+s1}{\PYZsq{}}\PY{l+s+s1}{| }\PY{l+s+s1}{\PYZsq{}}\PY{p}{)}
\PY{n+nb}{print}\PY{p}{(}\PY{l+s+s1}{\PYZsq{}}\PY{l+s+s1}{\PYZsh{}}\PY{l+s+s1}{\PYZsq{}} \PY{o}{+} \PY{n}{SCREEN\PYZus{}WIDTH} \PY{o}{*} \PY{l+s+s1}{\PYZsq{}}\PY{l+s+s1}{\PYZhy{}}\PY{l+s+s1}{\PYZsq{}} \PY{o}{+} \PY{l+s+s1}{\PYZsq{}}\PY{l+s+s1}{\PYZsh{}}\PY{l+s+s1}{\PYZsq{}}\PY{p}{)}
\PY{n+nb}{print}\PY{p}{(}\PY{l+s+s1}{\PYZsq{}}\PY{l+s+s1}{|}\PY{l+s+s1}{\PYZsq{}} \PY{o}{+} \PY{n}{centered}\PY{p}{(}\PY{l+s+s1}{\PYZsq{}}\PY{l+s+s1}{Die Payoffs der Put\PYZhy{}Option für s=}\PY{l+s+s1}{\PYZsq{}} \PY{o}{+} \PY{n+nb}{str}\PY{p}{(}\PY{n}{s}\PY{p}{)} \PY{o}{+} \PY{l+s+s1}{\PYZsq{}}\PY{l+s+s1}{ und b=20 betragen: }\PY{l+s+s1}{\PYZsq{}} \PY{o}{+} \PY{n+nb}{str}\PY{p}{(}\PY{n}{payoff\PYZus{}put}\PY{p}{(}\PY{n}{s}\PY{p}{,}\PY{n}{b}\PY{p}{)}\PY{p}{)}\PY{p}{)} \PY{o}{+} \PY{l+s+s1}{\PYZsq{}}\PY{l+s+s1}{| }\PY{l+s+s1}{\PYZsq{}}\PY{p}{)}
\PY{n+nb}{print}\PY{p}{(}\PY{l+s+s1}{\PYZsq{}}\PY{l+s+s1}{\PYZsh{}}\PY{l+s+s1}{\PYZsq{}} \PY{o}{+} \PY{n}{SCREEN\PYZus{}WIDTH} \PY{o}{*} \PY{l+s+s1}{\PYZsq{}}\PY{l+s+s1}{\PYZhy{}}\PY{l+s+s1}{\PYZsq{}} \PY{o}{+} \PY{l+s+s1}{\PYZsq{}}\PY{l+s+s1}{\PYZsh{}}\PY{l+s+s1}{\PYZsq{}}\PY{p}{)}
\end{Verbatim}
\end{tcolorbox}

    \begin{Verbatim}[commandchars=\\\{\}]
\#-------------------------------------------------------------------------------
----------------------------------------------\#
|                     Die Payoffs der Call-Option für s=[10 20 30 40] und b=20
betragen: [ 0.  0. 10. 20.]                    |
\#-------------------------------------------------------------------------------
----------------------------------------------\#
|                   Die Payoffs der Put-Option für s=[10 20 30 40] und b=20
betragen: [  0.   0. -10. -20.]                   |
\#-------------------------------------------------------------------------------
----------------------------------------------\#
    \end{Verbatim}

    \hypertarget{graphische-veranschaulichung-des-payoffs-einer-call-option}{%
\subsubsection{1.3) Graphische Veranschaulichung des Payoffs einer
Call-Option}\label{graphische-veranschaulichung-des-payoffs-einer-call-option}}

Auf Grundlage von 1.2) kann die Payoff-Funktion der Call-Option
graphisch veranschaulicht werden.

    \begin{tcolorbox}[breakable, size=fbox, boxrule=1pt, pad at break*=1mm,colback=cellbackground, colframe=cellborder]
\prompt{In}{incolor}{6}{\boxspacing}
\begin{Verbatim}[commandchars=\\\{\}]
\PY{n}{b} \PY{o}{=} \PY{l+m+mi}{20} \PY{c+c1}{\PYZsh{} Ausübungspreis}
\PY{n}{s} \PY{o}{=} \PY{n}{np}\PY{o}{.}\PY{n}{arange}\PY{p}{(}\PY{l+m+mi}{10}\PY{p}{,}\PY{l+m+mi}{50}\PY{p}{,}\PY{l+m+mi}{10}\PY{p}{)} \PY{c+c1}{\PYZsh{} Array mit möglichen Preisen einer Aktie}

\PY{n}{payoff} \PY{o}{=} \PY{p}{(}\PY{n}{s} \PY{o}{\PYZhy{}} \PY{n}{b} \PY{o}{+} \PY{n+nb}{abs}\PY{p}{(}\PY{n}{s} \PY{o}{\PYZhy{}} \PY{n}{b}\PY{p}{)}\PY{p}{)} \PY{o}{/} \PY{l+m+mi}{2} \PY{c+c1}{\PYZsh{} Bestimmung des Payoffs für jedes Element im Array}

\PY{n}{plt}\PY{o}{.}\PY{n}{ylim}\PY{p}{(}\PY{o}{\PYZhy{}}\PY{l+m+mi}{10}\PY{p}{,} \PY{l+m+mi}{25}\PY{p}{)} \PY{c+c1}{\PYZsh{} Grenzen der Y\PYZhy{}Achse in der Grafik}
\PY{n}{plt}\PY{o}{.}\PY{n}{axhline}\PY{p}{(}\PY{l+m+mi}{0}\PY{p}{,} \PY{n}{color}\PY{o}{=}\PY{l+s+s1}{\PYZsq{}}\PY{l+s+s1}{black}\PY{l+s+s1}{\PYZsq{}}\PY{p}{)} \PY{c+c1}{\PYZsh{} Plotten der X\PYZhy{}Achse}
\PY{n}{plt}\PY{o}{.}\PY{n}{plot}\PY{p}{(}\PY{n}{s}\PY{p}{,} \PY{n}{payoff}\PY{p}{)} \PY{c+c1}{\PYZsh{} Plotten der Payoff\PYZhy{}Funktion (Call)}
\PY{n}{plt}\PY{o}{.}\PY{n}{title}\PY{p}{(}\PY{l+s+s1}{\PYZsq{}}\PY{l+s+s1}{Payoff eines Calls mit Strike b=}\PY{l+s+si}{\PYZpc{}d}\PY{l+s+s1}{\PYZsq{}} \PY{o}{\PYZpc{}}\PY{k}{b}) \PYZsh{} Titel der Grafik
\PY{n}{plt}\PY{o}{.}\PY{n}{xlabel}\PY{p}{(}\PY{l+s+s1}{\PYZsq{}}\PY{l+s+s1}{Preis der Aktie}\PY{l+s+s1}{\PYZsq{}}\PY{p}{)} \PY{c+c1}{\PYZsh{} Bezeichung der X\PYZhy{}Achse}
\PY{n}{plt}\PY{o}{.}\PY{n}{ylabel}\PY{p}{(}\PY{l+s+s1}{\PYZsq{}}\PY{l+s+s1}{Payoff des Calls}\PY{l+s+s1}{\PYZsq{}}\PY{p}{)} \PY{c+c1}{\PYZsh{} Bezeichnung der Y\PYZhy{}Achse}
\PY{n}{plt}\PY{o}{.}\PY{n}{grid}\PY{p}{(}\PY{p}{)} \PY{c+c1}{\PYZsh{} Gitternetz}
\PY{n}{plt}\PY{o}{.}\PY{n}{show}\PY{p}{(}\PY{p}{)} \PY{c+c1}{\PYZsh{} Funktion zum anzeigen der Grafik}
\end{Verbatim}
\end{tcolorbox}

    \begin{center}
    \adjustimage{max size={0.9\linewidth}{0.9\paperheight}}{output_11_0.png}
    \end{center}
    { \hspace*{\fill} \\}
    
    \hypertarget{graphische-veranschaulichung-der-gewinn-verlustfunktion-eines-kuxe4ufersverkuxe4uferes-bei-einer-call-option}{%
\subsubsection{1.4) Graphische Veranschaulichung der
Gewinn-/Verlustfunktion eines Käufers/Verkäuferes bei einer
Call-Option}\label{graphische-veranschaulichung-der-gewinn-verlustfunktion-eines-kuxe4ufersverkuxe4uferes-bei-einer-call-option}}

Die Payoff-Funktionen von Käufer und Verkäufer einer Call-Option
verlaufen entgegengesetzt. Ist \texttt{b} bspw. 45 und \texttt{s}=50,
ergibt sich ein Payoff i.H.v. 5 für den Käufer der Call-Option. Im
selben Umfang erleidet der Verkäufer der Call-Option einen Verlust.
Anschließend ist noch die Optionsprämie \texttt{c} i.H.v. 2.5 GE zu
verrechnen.

    \begin{tcolorbox}[breakable, size=fbox, boxrule=1pt, pad at break*=1mm,colback=cellbackground, colframe=cellborder]
\prompt{In}{incolor}{7}{\boxspacing}
\begin{Verbatim}[commandchars=\\\{\}]
\PY{n}{s} \PY{o}{=} \PY{n}{np}\PY{o}{.}\PY{n}{arange}\PY{p}{(}\PY{l+m+mi}{30}\PY{p}{,}\PY{l+m+mi}{70}\PY{p}{,}\PY{l+m+mi}{5}\PY{p}{)} \PY{c+c1}{\PYZsh{} Array mit möglichen Preisen einer Aktie}
\PY{n}{b} \PY{o}{=} \PY{l+m+mi}{45} \PY{c+c1}{\PYZsh{} Ausübungspreis}
\PY{n}{c} \PY{o}{=} \PY{l+m+mf}{2.5} \PY{c+c1}{\PYZsh{} Options\PYZhy{}Prämie für einen Call}

\PY{n+nb}{print}\PY{p}{(}\PY{l+s+s1}{\PYZsq{}}\PY{l+s+s1}{\PYZsh{}}\PY{l+s+s1}{\PYZsq{}} \PY{o}{+} \PY{n}{SCREEN\PYZus{}WIDTH} \PY{o}{*} \PY{l+s+s1}{\PYZsq{}}\PY{l+s+s1}{\PYZhy{}}\PY{l+s+s1}{\PYZsq{}} \PY{o}{+} \PY{l+s+s1}{\PYZsq{}}\PY{l+s+s1}{\PYZsh{}}\PY{l+s+s1}{\PYZsq{}}\PY{p}{)}
\PY{n+nb}{print}\PY{p}{(}\PY{l+s+s1}{\PYZsq{}}\PY{l+s+s1}{|}\PY{l+s+s1}{\PYZsq{}} \PY{o}{+} \PY{n}{centered}\PY{p}{(}\PY{l+s+s1}{\PYZsq{}}\PY{l+s+s1}{Die Netto\PYZhy{}Payoffs für s=}\PY{l+s+s1}{\PYZsq{}} \PY{o}{+} \PY{n+nb}{str}\PY{p}{(}\PY{n}{s}\PY{p}{)} \PY{o}{+} \PY{l+s+s1}{\PYZsq{}}\PY{l+s+s1}{ und b=}\PY{l+s+si}{\PYZpc{}d}\PY{l+s+s1}{\PYZsq{}} \PY{o}{\PYZpc{}}\PY{k}{b} + \PYZsq{} betragen: \PYZsq{} + str(payoff\PYZus{}call(s,b) \PYZhy{} c) + \PYZsq{} (Long\PYZhy{}Call)\PYZsq{}) + \PYZsq{}| \PYZsq{})
\PY{n+nb}{print}\PY{p}{(}\PY{l+s+s1}{\PYZsq{}}\PY{l+s+s1}{|}\PY{l+s+s1}{\PYZsq{}} \PY{o}{+} \PY{n}{centered}\PY{p}{(}\PY{l+s+s1}{\PYZsq{}}\PY{l+s+s1}{Bzw. }\PY{l+s+s1}{\PYZsq{}} \PY{o}{+} \PY{n+nb}{str}\PY{p}{(}\PY{n}{payoff\PYZus{}put}\PY{p}{(}\PY{n}{s}\PY{p}{,}\PY{n}{b}\PY{p}{)} \PY{o}{+} \PY{n}{c}\PY{p}{)} \PY{o}{+} \PY{l+s+s1}{\PYZsq{}}\PY{l+s+s1}{ für den Short\PYZhy{}Call}\PY{l+s+s1}{\PYZsq{}}\PY{p}{)} \PY{o}{+} \PY{l+s+s1}{\PYZsq{}}\PY{l+s+s1}{| }\PY{l+s+s1}{\PYZsq{}}\PY{p}{)}
\PY{n+nb}{print}\PY{p}{(}\PY{l+s+s1}{\PYZsq{}}\PY{l+s+s1}{\PYZsh{}}\PY{l+s+s1}{\PYZsq{}} \PY{o}{+} \PY{n}{SCREEN\PYZus{}WIDTH} \PY{o}{*} \PY{l+s+s1}{\PYZsq{}}\PY{l+s+s1}{\PYZhy{}}\PY{l+s+s1}{\PYZsq{}} \PY{o}{+} \PY{l+s+s1}{\PYZsq{}}\PY{l+s+s1}{\PYZsh{}}\PY{l+s+s1}{\PYZsq{}}\PY{p}{)}

\PY{n}{y} \PY{o}{=} \PY{p}{(}\PY{n}{s} \PY{o}{\PYZhy{}} \PY{n}{b} \PY{o}{+} \PY{n+nb}{abs}\PY{p}{(}\PY{n}{s} \PY{o}{\PYZhy{}} \PY{n}{b}\PY{p}{)}\PY{p}{)} \PY{o}{/} \PY{l+m+mi}{2} \PY{o}{\PYZhy{}} \PY{n}{c} \PY{c+c1}{\PYZsh{} Bestimmung der Gewinn\PYZhy{}/Verlustfunktion für jedes Element im Array}

\PY{n}{plt}\PY{o}{.}\PY{n}{ylim}\PY{p}{(}\PY{o}{\PYZhy{}}\PY{l+m+mi}{30}\PY{p}{,}\PY{l+m+mi}{35}\PY{p}{)} \PY{c+c1}{\PYZsh{} Grenzen der Y\PYZhy{}Achse in der Grafik}
\PY{n}{plt}\PY{o}{.}\PY{n}{plot}\PY{p}{(}\PY{n}{s}\PY{p}{,}\PY{n}{y}\PY{p}{)} \PY{c+c1}{\PYZsh{} Plotten der Gewinn\PYZhy{}/Verlustfunktion für den Käufer der Call\PYZhy{}Option}
\PY{n}{plt}\PY{o}{.}\PY{n}{axhline}\PY{p}{(}\PY{l+m+mi}{0}\PY{p}{,} \PY{n}{color}\PY{o}{=}\PY{l+s+s1}{\PYZsq{}}\PY{l+s+s1}{black}\PY{l+s+s1}{\PYZsq{}}\PY{p}{)} \PY{c+c1}{\PYZsh{} Plotten der X\PYZhy{}Achse}
\PY{n}{plt}\PY{o}{.}\PY{n}{plot}\PY{p}{(}\PY{n}{s}\PY{p}{,}\PY{o}{\PYZhy{}}\PY{n}{y}\PY{p}{)} \PY{c+c1}{\PYZsh{} Plotten der Gewinn\PYZhy{}/Verlustfunktion für den Verkäufer der Call\PYZhy{}Option}
\PY{n}{plt}\PY{o}{.}\PY{n}{plot}\PY{p}{(}\PY{p}{[}\PY{n}{b}\PY{p}{,}\PY{n}{b}\PY{p}{]}\PY{p}{,}\PY{p}{[}\PY{o}{\PYZhy{}}\PY{l+m+mi}{30}\PY{p}{,}\PY{l+m+mi}{3}\PY{p}{]}\PY{p}{,} \PY{l+s+s1}{\PYZsq{}}\PY{l+s+s1}{:}\PY{l+s+s1}{\PYZsq{}}\PY{p}{)} \PY{c+c1}{\PYZsh{} Plotten der Geraden zur Markierung des Ausübungspreises}
\PY{n}{plt}\PY{o}{.}\PY{n}{plot}\PY{p}{(}\PY{p}{[}\PY{n}{b}\PY{o}{+}\PY{n}{c}\PY{p}{,}\PY{n}{b}\PY{o}{+}\PY{n}{c}\PY{p}{]}\PY{p}{,}\PY{p}{[}\PY{o}{\PYZhy{}}\PY{l+m+mi}{30}\PY{p}{,}\PY{l+m+mi}{0}\PY{p}{]}\PY{p}{,} \PY{l+s+s1}{\PYZsq{}}\PY{l+s+s1}{:}\PY{l+s+s1}{\PYZsq{}}\PY{p}{)} 
\PY{n}{plt}\PY{o}{.}\PY{n}{title}\PY{p}{(}\PY{l+s+s1}{\PYZsq{}}\PY{l+s+s1}{Gewinn\PYZhy{}/Verlustfunktion für eine Call\PYZhy{}Option}\PY{l+s+s1}{\PYZsq{}}\PY{p}{)} \PY{c+c1}{\PYZsh{} Titel der Grafik}
\PY{n}{plt}\PY{o}{.}\PY{n}{xlabel}\PY{p}{(}\PY{l+s+s1}{\PYZsq{}}\PY{l+s+s1}{Preis der Aktie}\PY{l+s+s1}{\PYZsq{}}\PY{p}{)} \PY{c+c1}{\PYZsh{} Bezeichung der X\PYZhy{}Achse}
\PY{n}{plt}\PY{o}{.}\PY{n}{ylabel}\PY{p}{(}\PY{l+s+s1}{\PYZsq{}}\PY{l+s+s1}{Gewinn/Verlust}\PY{l+s+s1}{\PYZsq{}}\PY{p}{)} \PY{c+c1}{\PYZsh{} Bezeichnung der Y\PYZhy{}Achse}
\PY{n}{plt}\PY{o}{.}\PY{n}{grid}\PY{p}{(}\PY{p}{)} \PY{c+c1}{\PYZsh{} Gitternetz}
\PY{n}{plt}\PY{o}{.}\PY{n}{annotate}\PY{p}{(}\PY{l+s+s1}{\PYZsq{}}\PY{l+s+s1}{Käufer Call\PYZhy{}Option}\PY{l+s+s1}{\PYZsq{}}\PY{p}{,} \PY{n}{xy} \PY{o}{=} \PY{p}{(}\PY{l+m+mf}{52.5}\PY{p}{,}\PY{l+m+mf}{6.5}\PY{p}{)}\PY{p}{,} \PY{n}{xytext}\PY{o}{=}\PY{p}{(}\PY{l+m+mf}{52.5}\PY{p}{,}\PY{l+m+mi}{14}\PY{p}{)}\PY{p}{,} \PY{n}{arrowprops}\PY{o}{=}\PY{n+nb}{dict}\PY{p}{(}\PY{n}{facecolor}\PY{o}{=}\PY{l+s+s1}{\PYZsq{}}\PY{l+s+s1}{blue}\PY{l+s+s1}{\PYZsq{}}\PY{p}{,} \PY{n}{shrink}\PY{o}{=}\PY{l+m+mf}{0.05}\PY{p}{)}\PY{p}{)} \PY{c+c1}{\PYZsh{} Beschriftung und Markierung der Funktion mit einem Pfeil}
\PY{n}{plt}\PY{o}{.}\PY{n}{annotate}\PY{p}{(}\PY{l+s+s1}{\PYZsq{}}\PY{l+s+s1}{Verkäufer Call\PYZhy{}Option}\PY{l+s+s1}{\PYZsq{}}\PY{p}{,} \PY{n}{xy} \PY{o}{=} \PY{p}{(}\PY{l+m+mf}{52.5}\PY{p}{,}\PY{o}{\PYZhy{}}\PY{l+m+mf}{6.5}\PY{p}{)}\PY{p}{,} \PY{n}{xytext}\PY{o}{=}\PY{p}{(}\PY{l+m+mf}{52.5}\PY{p}{,}\PY{o}{\PYZhy{}}\PY{l+m+mi}{14}\PY{p}{)}\PY{p}{,} \PY{n}{arrowprops}\PY{o}{=}\PY{n+nb}{dict}\PY{p}{(}\PY{n}{facecolor}\PY{o}{=}\PY{l+s+s1}{\PYZsq{}}\PY{l+s+s1}{orange}\PY{l+s+s1}{\PYZsq{}}\PY{p}{,} \PY{n}{shrink}\PY{o}{=}\PY{l+m+mf}{0.05}\PY{p}{)}\PY{p}{)} \PY{c+c1}{\PYZsh{} Beschriftung und Markierung der Funktion mit einem Pfeil}
\PY{n}{plt}\PY{o}{.}\PY{n}{annotate}\PY{p}{(}\PY{l+s+s1}{\PYZsq{}}\PY{l+s+s1}{Ausübungspreis (Basispreis) b=}\PY{l+s+si}{\PYZpc{}d}\PY{l+s+s1}{\PYZsq{}} \PY{o}{\PYZpc{}}\PY{k}{b}, xy = (45,\PYZhy{}30), xytext=(32.5,\PYZhy{}22.5), arrowprops=dict(facecolor=\PYZsq{}green\PYZsq{}, shrink=0.05)) \PYZsh{} Beschriftung und Markierung der Funktion mit einem Pfeil
\PY{n}{plt}\PY{o}{.}\PY{n}{annotate}\PY{p}{(}\PY{l+s+s1}{\PYZsq{}}\PY{l+s+s1}{Ausübungspreis b + Optionsprämie c}\PY{l+s+s1}{\PYZsq{}}\PY{p}{,} \PY{n}{xy} \PY{o}{=} \PY{p}{(}\PY{l+m+mi}{45}\PY{o}{+}\PY{l+m+mf}{2.5}\PY{p}{,}\PY{o}{\PYZhy{}}\PY{l+m+mi}{30}\PY{p}{)}\PY{p}{,} \PY{n}{xytext}\PY{o}{=}\PY{p}{(}\PY{l+m+mi}{50}\PY{p}{,}\PY{o}{\PYZhy{}}\PY{l+m+mf}{22.5}\PY{p}{)}\PY{p}{,} \PY{n}{arrowprops}\PY{o}{=}\PY{n+nb}{dict}\PY{p}{(}\PY{n}{facecolor}\PY{o}{=}\PY{l+s+s1}{\PYZsq{}}\PY{l+s+s1}{red}\PY{l+s+s1}{\PYZsq{}}\PY{p}{,} \PY{n}{shrink}\PY{o}{=}\PY{l+m+mf}{0.05}\PY{p}{)}\PY{p}{)} \PY{c+c1}{\PYZsh{} Beschriftung und Markierung der Funktion mit einem Pfeil}
\PY{n}{plt}\PY{o}{.}\PY{n}{show}\PY{p}{(}\PY{p}{)} \PY{c+c1}{\PYZsh{} Funktion zum anzeigen der Grafik}
\end{Verbatim}
\end{tcolorbox}

    \begin{Verbatim}[commandchars=\\\{\}]
\#-------------------------------------------------------------------------------
----------------------------------------------\#
|  Die Netto-Payoffs für s=[30 35 40 45 50 55 60 65] und b=45 betragen: [-2.5
-2.5 -2.5 -2.5  2.5  7.5 12.5 17.5] (Long-Call) |
|                          Bzw. [  2.5   2.5   2.5   2.5  -2.5  -7.5 -12.5
-17.5] für den Short-Call                          |
\#-------------------------------------------------------------------------------
----------------------------------------------\#
    \end{Verbatim}

    \begin{center}
    \adjustimage{max size={0.9\linewidth}{0.9\paperheight}}{output_13_1.png}
    \end{center}
    { \hspace*{\fill} \\}
    
    \hypertarget{graphische-veranschaulichung-der-gewinn-verlustfunktion-eines-kuxe4ufersverkuxe4uferes-bei-einer-put-option}{%
\subsubsection{1.5) Graphische Veranschaulichung der
Gewinn-/Verlustfunktion eines Käufers/Verkäuferes bei einer
Put-Option}\label{graphische-veranschaulichung-der-gewinn-verlustfunktion-eines-kuxe4ufersverkuxe4uferes-bei-einer-put-option}}

Analog lässt sich dies auch für eine Put-Option veranschaulichen.

    \begin{tcolorbox}[breakable, size=fbox, boxrule=1pt, pad at break*=1mm,colback=cellbackground, colframe=cellborder]
\prompt{In}{incolor}{8}{\boxspacing}
\begin{Verbatim}[commandchars=\\\{\}]
\PY{n}{s} \PY{o}{=} \PY{n}{np}\PY{o}{.}\PY{n}{arange}\PY{p}{(}\PY{l+m+mi}{30}\PY{p}{,}\PY{l+m+mi}{70}\PY{p}{,}\PY{l+m+mi}{5}\PY{p}{)} \PY{c+c1}{\PYZsh{} Array mit möglichen Preisen einer Aktie}
\PY{n}{b} \PY{o}{=} \PY{l+m+mi}{45} \PY{c+c1}{\PYZsh{} Ausübungspreis}
\PY{n}{p} \PY{o}{=} \PY{l+m+mf}{2.5} \PY{c+c1}{\PYZsh{} Options\PYZhy{}Prämie für einen Put}

\PY{n}{y} \PY{o}{=} \PY{p}{(}\PY{n}{b} \PY{o}{\PYZhy{}} \PY{n}{s} \PY{o}{+} \PY{n+nb}{abs}\PY{p}{(}\PY{n}{b} \PY{o}{\PYZhy{}} \PY{n}{s}\PY{p}{)}\PY{p}{)} \PY{o}{/} \PY{l+m+mi}{2} \PY{o}{\PYZhy{}} \PY{n}{p} \PY{c+c1}{\PYZsh{} Bestimmung der Gewinn\PYZhy{}/Verlustfunktion für jedes Element im Array}

\PY{n}{plt}\PY{o}{.}\PY{n}{ylim}\PY{p}{(}\PY{o}{\PYZhy{}}\PY{l+m+mi}{30}\PY{p}{,}\PY{l+m+mi}{35}\PY{p}{)} \PY{c+c1}{\PYZsh{} Grenzen der Y\PYZhy{}Achse in der Grafik}
\PY{n}{plt}\PY{o}{.}\PY{n}{plot}\PY{p}{(}\PY{n}{s}\PY{p}{,}\PY{n}{y}\PY{p}{)} \PY{c+c1}{\PYZsh{} Plotten der Gewinn\PYZhy{}/Verlustfunktion für den Käufer der Put\PYZhy{}Option}
\PY{n}{plt}\PY{o}{.}\PY{n}{axhline}\PY{p}{(}\PY{l+m+mi}{0}\PY{p}{,} \PY{n}{color}\PY{o}{=}\PY{l+s+s1}{\PYZsq{}}\PY{l+s+s1}{black}\PY{l+s+s1}{\PYZsq{}}\PY{p}{)} \PY{c+c1}{\PYZsh{} Plotten der X\PYZhy{}Achse}
\PY{n}{plt}\PY{o}{.}\PY{n}{plot}\PY{p}{(}\PY{n}{s}\PY{p}{,}\PY{o}{\PYZhy{}}\PY{n}{y}\PY{p}{)} \PY{c+c1}{\PYZsh{} Plotten der Gewinn\PYZhy{}/Verlustfunktion für den Verkäufer der Put\PYZhy{}Option}
\PY{n}{plt}\PY{o}{.}\PY{n}{plot}\PY{p}{(}\PY{p}{[}\PY{n}{b}\PY{p}{,}\PY{n}{b}\PY{p}{]}\PY{p}{,}\PY{p}{[}\PY{o}{\PYZhy{}}\PY{l+m+mi}{30}\PY{p}{,}\PY{l+m+mi}{3}\PY{p}{]}\PY{p}{,} \PY{l+s+s1}{\PYZsq{}}\PY{l+s+s1}{:}\PY{l+s+s1}{\PYZsq{}}\PY{p}{)} \PY{c+c1}{\PYZsh{} Plotten der Geraden zur Markierung des Ausübungspreises}
\PY{n}{plt}\PY{o}{.}\PY{n}{plot}\PY{p}{(}\PY{p}{[}\PY{n}{b}\PY{o}{\PYZhy{}}\PY{n}{p}\PY{p}{,}\PY{n}{b}\PY{o}{\PYZhy{}}\PY{n}{p}\PY{p}{]}\PY{p}{,}\PY{p}{[}\PY{o}{\PYZhy{}}\PY{l+m+mi}{30}\PY{p}{,}\PY{l+m+mi}{0}\PY{p}{]}\PY{p}{,} \PY{l+s+s1}{\PYZsq{}}\PY{l+s+s1}{:}\PY{l+s+s1}{\PYZsq{}}\PY{p}{)}
\PY{n}{plt}\PY{o}{.}\PY{n}{grid}\PY{p}{(}\PY{p}{)} \PY{c+c1}{\PYZsh{} Gitternetz}
\PY{n}{plt}\PY{o}{.}\PY{n}{title}\PY{p}{(}\PY{l+s+s1}{\PYZsq{}}\PY{l+s+s1}{Gewinn\PYZhy{}/Verlustfunktion für eine Put\PYZhy{}Option}\PY{l+s+s1}{\PYZsq{}}\PY{p}{)} \PY{c+c1}{\PYZsh{} Titel der Grafik }
\PY{n}{plt}\PY{o}{.}\PY{n}{xlabel}\PY{p}{(}\PY{l+s+s1}{\PYZsq{}}\PY{l+s+s1}{Preis der Aktie}\PY{l+s+s1}{\PYZsq{}}\PY{p}{)} \PY{c+c1}{\PYZsh{} Bezeichung der X\PYZhy{}Achse}
\PY{n}{plt}\PY{o}{.}\PY{n}{ylabel}\PY{p}{(}\PY{l+s+s1}{\PYZsq{}}\PY{l+s+s1}{Gewinn/Verlust}\PY{l+s+s1}{\PYZsq{}}\PY{p}{)}  \PY{c+c1}{\PYZsh{} Bezeichnung der Y\PYZhy{}Achse}
\PY{n}{plt}\PY{o}{.}\PY{n}{annotate}\PY{p}{(}\PY{l+s+s1}{\PYZsq{}}\PY{l+s+s1}{Käufer Put\PYZhy{}Option}\PY{l+s+s1}{\PYZsq{}}\PY{p}{,} \PY{n}{xy} \PY{o}{=} \PY{p}{(}\PY{l+m+mf}{32.5}\PY{p}{,}\PY{l+m+mi}{12}\PY{p}{)}\PY{p}{,} \PY{n}{xytext}\PY{o}{=}\PY{p}{(}\PY{l+m+mf}{32.5}\PY{p}{,}\PY{l+m+mf}{20.5}\PY{p}{)}\PY{p}{,} \PY{n}{arrowprops}\PY{o}{=}\PY{n+nb}{dict}\PY{p}{(}\PY{n}{facecolor}\PY{o}{=}\PY{l+s+s1}{\PYZsq{}}\PY{l+s+s1}{blue}\PY{l+s+s1}{\PYZsq{}}\PY{p}{,} \PY{n}{shrink}\PY{o}{=}\PY{l+m+mf}{0.05}\PY{p}{)}\PY{p}{)} \PY{c+c1}{\PYZsh{} Beschriftung und Markierung der Funktion mit einem Pfeil}
\PY{n}{plt}\PY{o}{.}\PY{n}{annotate}\PY{p}{(}\PY{l+s+s1}{\PYZsq{}}\PY{l+s+s1}{Verkäufer }\PY{l+s+se}{\PYZbs{}n}\PY{l+s+s1}{Put\PYZhy{}Option}\PY{l+s+s1}{\PYZsq{}}\PY{p}{,} \PY{n}{xy} \PY{o}{=} \PY{p}{(}\PY{l+m+mf}{32.5}\PY{p}{,}\PY{o}{\PYZhy{}}\PY{l+m+mi}{12}\PY{p}{)}\PY{p}{,} \PY{n}{xytext}\PY{o}{=}\PY{p}{(}\PY{l+m+mf}{32.5}\PY{p}{,}\PY{o}{\PYZhy{}}\PY{l+m+mi}{25}\PY{p}{)}\PY{p}{,} \PY{n}{arrowprops}\PY{o}{=}\PY{n+nb}{dict}\PY{p}{(}\PY{n}{facecolor}\PY{o}{=}\PY{l+s+s1}{\PYZsq{}}\PY{l+s+s1}{orange}\PY{l+s+s1}{\PYZsq{}}\PY{p}{,} \PY{n}{shrink}\PY{o}{=}\PY{l+m+mf}{0.05}\PY{p}{)}\PY{p}{)} \PY{c+c1}{\PYZsh{} Beschriftung und Markierung der Funktion mit einem Pfeil }
\PY{n}{plt}\PY{o}{.}\PY{n}{annotate}\PY{p}{(}\PY{l+s+s1}{\PYZsq{}}\PY{l+s+s1}{Ausübungspreis (Basispreis) b=}\PY{l+s+si}{\PYZpc{}d}\PY{l+s+s1}{\PYZsq{}} \PY{o}{\PYZpc{}}\PY{k}{b}, xy = (45,\PYZhy{}30), xytext=(47.5,\PYZhy{}22.5), arrowprops=dict(facecolor=\PYZsq{}green\PYZsq{}, shrink=0.05)) \PYZsh{} Beschriftung und Markierung der Funktion mit einem Pfeil
\PY{n}{plt}\PY{o}{.}\PY{n}{annotate}\PY{p}{(}\PY{l+s+s1}{\PYZsq{}}\PY{l+s+s1}{Ausübungspreis b \PYZhy{} Optionsprämie p}\PY{l+s+s1}{\PYZsq{}}\PY{p}{,} \PY{n}{xy} \PY{o}{=} \PY{p}{(}\PY{l+m+mi}{45}\PY{o}{\PYZhy{}}\PY{l+m+mf}{2.5}\PY{p}{,}\PY{o}{\PYZhy{}}\PY{l+m+mi}{20}\PY{p}{)}\PY{p}{,} \PY{n}{xytext}\PY{o}{=}\PY{p}{(}\PY{l+m+mf}{47.5}\PY{p}{,}\PY{o}{\PYZhy{}}\PY{l+m+mi}{15}\PY{p}{)}\PY{p}{,} \PY{n}{arrowprops}\PY{o}{=}\PY{n+nb}{dict}\PY{p}{(}\PY{n}{facecolor}\PY{o}{=}\PY{l+s+s1}{\PYZsq{}}\PY{l+s+s1}{red}\PY{l+s+s1}{\PYZsq{}}\PY{p}{,} \PY{n}{shrink}\PY{o}{=}\PY{l+m+mf}{0.05}\PY{p}{)}\PY{p}{)}
\PY{n}{plt}\PY{o}{.}\PY{n}{show}\PY{p}{(}\PY{p}{)} \PY{c+c1}{\PYZsh{} Funktion zum anzeigen der Grafik}
\end{Verbatim}
\end{tcolorbox}

    \begin{center}
    \adjustimage{max size={0.9\linewidth}{0.9\paperheight}}{output_15_0.png}
    \end{center}
    { \hspace*{\fill} \\}
    
    \hypertarget{aufagbe-2---ein-uxfcberblick-uxfcber-verschiedene-tradingstrategien}{%
\subsection{Aufagbe 2 - Ein Überblick über Verschiedene
Tradingstrategien}\label{aufagbe-2---ein-uxfcberblick-uxfcber-verschiedene-tradingstrategien}}

    Variablenverzeichnis: - c = Optionsprämie für einen Call - s = Preis der
Aktie zum Fälligkeitsdatum - tau = Aktueller Preis der Aktie - b =
Ausübungspreis (Basispreis) - p = Options-Prämie für einen Put

    \hypertarget{covered-call---eine-kombination-aus-einer-aktie-in-der-long-position-und-der-short-position-eines-calls}{%
\subsubsection{2.1) Covered Call - Eine Kombination aus einer Aktie in
der Long-Position und der Short-Position eines
Calls}\label{covered-call---eine-kombination-aus-einer-aktie-in-der-long-position-und-der-short-position-eines-calls}}

In dieser Grafik sind die Payoff-Funktionen eines Aktienkaufs, eines
covered Calls und eines Calls zu sehen. Sofern der Preis der Aktie unter
17 GE bleibt (\texttt{b}=15 + \texttt{c}=2), ist der gedeckte Call dem
Aktienkauf vorzuziehen, da ein niedrigerer Preis des Underlyings
ausreicht um denselben Payoff zu erzielen. Darüber hinaus ist der
alleinige Aktienkauf von Vorteil.

    \begin{tcolorbox}[breakable, size=fbox, boxrule=1pt, pad at break*=1mm,colback=cellbackground, colframe=cellborder]
\prompt{In}{incolor}{9}{\boxspacing}
\begin{Verbatim}[commandchars=\\\{\}]
\PY{n}{s} \PY{o}{=} \PY{n}{np}\PY{o}{.}\PY{n}{arange}\PY{p}{(}\PY{l+m+mi}{0}\PY{p}{,}\PY{l+m+mi}{40}\PY{p}{,}\PY{l+m+mi}{5}\PY{p}{)} \PY{c+c1}{\PYZsh{} Array mit möglichen Preisen einer Aktie}
\PY{n}{b} \PY{o}{=} \PY{l+m+mi}{15} \PY{c+c1}{\PYZsh{} Ausübungspreis (Basispreis)}
\PY{n}{tau} \PY{o}{=} \PY{l+m+mi}{10} \PY{c+c1}{\PYZsh{} Aktueller Preis der Aktie}
\PY{n}{c} \PY{o}{=} \PY{l+m+mi}{2} \PY{c+c1}{\PYZsh{} Optionsprämie für einen Call}

\PY{n}{y1} \PY{o}{=} \PY{n}{s} \PY{o}{\PYZhy{}} \PY{n}{tau} \PY{c+c1}{\PYZsh{} Nur Aktienkauf}
\PY{n}{y2} \PY{o}{=} \PY{p}{(}\PY{n}{s} \PY{o}{\PYZhy{}} \PY{n}{b} \PY{o}{+} \PY{n+nb}{abs}\PY{p}{(}\PY{n}{s} \PY{o}{\PYZhy{}} \PY{n}{b}\PY{p}{)}\PY{p}{)} \PY{o}{/} \PY{l+m+mi}{2} \PY{o}{\PYZhy{}} \PY{n}{c} \PY{c+c1}{\PYZsh{} Call}
\PY{n}{y3} \PY{o}{=} \PY{n}{y1} \PY{o}{\PYZhy{}} \PY{n}{y2} \PY{c+c1}{\PYZsh{} Covered Call}

\PY{n}{plt}\PY{o}{.}\PY{n}{ylim}\PY{p}{(}\PY{o}{\PYZhy{}}\PY{l+m+mi}{10}\PY{p}{,}\PY{l+m+mi}{30}\PY{p}{)} \PY{c+c1}{\PYZsh{} Grenzen der Y\PYZhy{}Achse in der Grafik}
\PY{n}{plt}\PY{o}{.}\PY{n}{plot}\PY{p}{(}\PY{n}{s}\PY{p}{,}\PY{n}{y1}\PY{p}{)} \PY{c+c1}{\PYZsh{} Plotten der Funktion für den alleinigen Aktienkauf}
\PY{n}{plt}\PY{o}{.}\PY{n}{plot}\PY{p}{(}\PY{n}{s}\PY{p}{,}\PY{n}{y2}\PY{p}{)} \PY{c+c1}{\PYZsh{}  Plotten der Funktion für den Call}
\PY{n}{plt}\PY{o}{.}\PY{n}{plot}\PY{p}{(}\PY{n}{s}\PY{p}{,}\PY{n}{y3}\PY{p}{,} \PY{n}{color}\PY{o}{=}\PY{l+s+s1}{\PYZsq{}}\PY{l+s+s1}{red}\PY{l+s+s1}{\PYZsq{}}\PY{p}{)} \PY{c+c1}{\PYZsh{} Plotten der Funktion für den gedeckten Call}
\PY{n}{plt}\PY{o}{.}\PY{n}{axhline}\PY{p}{(}\PY{l+m+mi}{0}\PY{p}{,} \PY{n}{color}\PY{o}{=}\PY{l+s+s1}{\PYZsq{}}\PY{l+s+s1}{black}\PY{l+s+s1}{\PYZsq{}}\PY{p}{)} \PY{c+c1}{\PYZsh{} Plotten der X\PYZhy{}Achse}
\PY{n}{plt}\PY{o}{.}\PY{n}{plot}\PY{p}{(}\PY{p}{[}\PY{n}{b}\PY{p}{,}\PY{n}{b}\PY{p}{]}\PY{p}{,}\PY{p}{[}\PY{o}{\PYZhy{}}\PY{l+m+mi}{10}\PY{p}{,}\PY{l+m+mf}{7.5}\PY{p}{]}\PY{p}{,} \PY{l+s+s1}{\PYZsq{}}\PY{l+s+s1}{:}\PY{l+s+s1}{\PYZsq{}}\PY{p}{,} \PY{n}{color}\PY{o}{=}\PY{l+s+s1}{\PYZsq{}}\PY{l+s+s1}{green}\PY{l+s+s1}{\PYZsq{}}\PY{p}{)} \PY{c+c1}{\PYZsh{} Plotten der Geraden zur Markierung des Ausübungspreises}
\PY{n}{plt}\PY{o}{.}\PY{n}{title}\PY{p}{(}\PY{l+s+s1}{\PYZsq{}}\PY{l+s+s1}{Covered Call (Long\PYZhy{}Position and short one call)}\PY{l+s+s1}{\PYZsq{}}\PY{p}{)} \PY{c+c1}{\PYZsh{} Titel der Grafik }
\PY{n}{plt}\PY{o}{.}\PY{n}{xlabel}\PY{p}{(}\PY{l+s+s1}{\PYZsq{}}\PY{l+s+s1}{Preis der Aktie}\PY{l+s+s1}{\PYZsq{}}\PY{p}{)} \PY{c+c1}{\PYZsh{} Bezeichung der X\PYZhy{}Achse}
\PY{n}{plt}\PY{o}{.}\PY{n}{ylabel}\PY{p}{(}\PY{l+s+s1}{\PYZsq{}}\PY{l+s+s1}{Gewinn/Verlust}\PY{l+s+s1}{\PYZsq{}}\PY{p}{)} \PY{c+c1}{\PYZsh{} Bezeichnung der Y\PYZhy{}Achse}
\PY{n}{plt}\PY{o}{.}\PY{n}{annotate}\PY{p}{(}\PY{l+s+s1}{\PYZsq{}}\PY{l+s+s1}{Nur Aktienkauf}\PY{l+s+s1}{\PYZsq{}}\PY{p}{,} \PY{n}{xy} \PY{o}{=} \PY{p}{(}\PY{l+m+mi}{24}\PY{p}{,}\PY{l+m+mi}{15}\PY{p}{)}\PY{p}{,} \PY{n}{xytext}\PY{o}{=}\PY{p}{(}\PY{l+m+mi}{18}\PY{p}{,}\PY{l+m+mi}{20}\PY{p}{)}\PY{p}{,} \PY{n}{arrowprops}\PY{o}{=}\PY{n+nb}{dict}\PY{p}{(}\PY{n}{facecolor}\PY{o}{=}\PY{l+s+s1}{\PYZsq{}}\PY{l+s+s1}{blue}\PY{l+s+s1}{\PYZsq{}}\PY{p}{,} \PY{n}{shrink}\PY{o}{=}\PY{l+m+mf}{0.05}\PY{p}{)}\PY{p}{)} \PY{c+c1}{\PYZsh{} Beschriftung und Markierung der Funktion mit einem Pfeil}
\PY{n}{plt}\PY{o}{.}\PY{n}{annotate}\PY{p}{(}\PY{l+s+s1}{\PYZsq{}}\PY{l+s+s1}{Covered Call}\PY{l+s+s1}{\PYZsq{}}\PY{p}{,} \PY{n}{xy} \PY{o}{=} \PY{p}{(}\PY{l+m+mi}{10}\PY{p}{,}\PY{l+m+mi}{2}\PY{p}{)}\PY{p}{,} \PY{n}{xytext}\PY{o}{=}\PY{p}{(}\PY{l+m+mi}{9}\PY{p}{,}\PY{l+m+mi}{8}\PY{p}{)}\PY{p}{,} \PY{n}{arrowprops}\PY{o}{=}\PY{n+nb}{dict}\PY{p}{(}\PY{n}{facecolor}\PY{o}{=}\PY{l+s+s1}{\PYZsq{}}\PY{l+s+s1}{red}\PY{l+s+s1}{\PYZsq{}}\PY{p}{,} \PY{n}{shrink}\PY{o}{=}\PY{l+m+mf}{0.05}\PY{p}{)}\PY{p}{)} \PY{c+c1}{\PYZsh{} Beschriftung und Markierung der Funktion mit einem Pfeil}
\PY{n}{plt}\PY{o}{.}\PY{n}{annotate}\PY{p}{(}\PY{l+s+s1}{\PYZsq{}}\PY{l+s+s1}{Call}\PY{l+s+s1}{\PYZsq{}}\PY{p}{,} \PY{n}{xy} \PY{o}{=} \PY{p}{(}\PY{l+m+mi}{20}\PY{p}{,}\PY{l+m+mi}{2}\PY{p}{)}\PY{p}{,} \PY{n}{xytext}\PY{o}{=}\PY{p}{(}\PY{l+m+mi}{19}\PY{p}{,}\PY{o}{\PYZhy{}}\PY{l+m+mi}{5}\PY{p}{)}\PY{p}{,} \PY{n}{arrowprops}\PY{o}{=}\PY{n+nb}{dict}\PY{p}{(}\PY{n}{facecolor}\PY{o}{=}\PY{l+s+s1}{\PYZsq{}}\PY{l+s+s1}{orange}\PY{l+s+s1}{\PYZsq{}}\PY{p}{,} \PY{n}{shrink}\PY{o}{=}\PY{l+m+mf}{0.05}\PY{p}{)}\PY{p}{)} \PY{c+c1}{\PYZsh{} Beschriftung und Markierung der Funktion mit einem Pfeil}
\PY{n}{plt}\PY{o}{.}\PY{n}{annotate}\PY{p}{(}\PY{l+s+s1}{\PYZsq{}}\PY{l+s+s1}{Ausübungspreis (Basispreis) b=}\PY{l+s+si}{\PYZpc{}d}\PY{l+s+s1}{\PYZsq{}} \PY{o}{\PYZpc{}}\PY{k}{b}, xy = (b + 0.2,\PYZhy{}10 + 0.5), xytext=(17.5,\PYZhy{}8.5), arrowprops=dict(facecolor=\PYZsq{}green\PYZsq{}, shrink=0.05)) \PYZsh{} Beschriftung und Markierung der Funktion mit einem Pfeil
\PY{n}{plt}\PY{o}{.}\PY{n}{grid}\PY{p}{(}\PY{p}{)} \PY{c+c1}{\PYZsh{} Gitternetz}
\PY{n}{plt}\PY{o}{.}\PY{n}{show}\PY{p}{(}\PY{p}{)} \PY{c+c1}{\PYZsh{} Funktion zum anzeigen der Grafik}
\end{Verbatim}
\end{tcolorbox}

    \begin{center}
    \adjustimage{max size={0.9\linewidth}{0.9\paperheight}}{output_19_0.png}
    \end{center}
    { \hspace*{\fill} \\}
    
    \hypertarget{straddle---kauf-eines-calls-und-eines-puts-mit-demselben-ausuxfcbungspreis}{%
\subsubsection{2.2) Straddle - Kauf eines Calls und eines Puts mit
demselben
Ausübungspreis}\label{straddle---kauf-eines-calls-und-eines-puts-mit-demselben-ausuxfcbungspreis}}

Bei einem Straddle erfolgt der Kauf eines Calls und eines Puts mit
demselben Ausübungspreis. Anwendung findet diese Optionsstrategie bspw.,
wenn nicht abgeschätzt werden kann welche Auswirkungen (positiv oder
negativ) ein zukünftiges Unternehmensereignis hat. Sofern sich der Preis
der Aktie in eine beliebige Richtung bewegt, profitiert der Halter der
beiden Positionen. Der Break-Even-Point wird erreicht, sobald die
Auszahlung die Kosten der Optionsprämie \texttt{c} und \texttt{p} deckt.

    \begin{tcolorbox}[breakable, size=fbox, boxrule=1pt, pad at break*=1mm,colback=cellbackground, colframe=cellborder]
\prompt{In}{incolor}{10}{\boxspacing}
\begin{Verbatim}[commandchars=\\\{\}]
\PY{n}{s} \PY{o}{=} \PY{n}{np}\PY{o}{.}\PY{n}{arange}\PY{p}{(}\PY{l+m+mi}{30}\PY{p}{,}\PY{l+m+mi}{80}\PY{p}{,}\PY{l+m+mi}{5}\PY{p}{)} \PY{c+c1}{\PYZsh{} Array mit möglichen Preisen einer Aktie}
\PY{n}{b} \PY{o}{=} \PY{l+m+mi}{50} \PY{c+c1}{\PYZsh{} Ausübungspreis}
\PY{n}{c} \PY{o}{=} \PY{l+m+mi}{2} \PY{c+c1}{\PYZsh{} Optionsprämie für einen Call}
\PY{n}{p} \PY{o}{=} \PY{l+m+mi}{1} \PY{c+c1}{\PYZsh{} Optionsprämie für einen Put }

\PY{n}{call} \PY{o}{=} \PY{p}{(}\PY{n}{s} \PY{o}{\PYZhy{}} \PY{n}{b} \PY{o}{+} \PY{n+nb}{abs}\PY{p}{(}\PY{n}{s} \PY{o}{\PYZhy{}} \PY{n}{b}\PY{p}{)}\PY{p}{)} \PY{o}{/} \PY{l+m+mi}{2} \PY{o}{\PYZhy{}} \PY{n}{c} \PY{c+c1}{\PYZsh{} Call}
\PY{n}{put} \PY{o}{=} \PY{p}{(}\PY{n}{b} \PY{o}{\PYZhy{}} \PY{n}{s} \PY{o}{+} \PY{n+nb}{abs}\PY{p}{(}\PY{n}{b} \PY{o}{\PYZhy{}} \PY{n}{s}\PY{p}{)}\PY{p}{)} \PY{o}{/} \PY{l+m+mi}{2} \PY{o}{\PYZhy{}} \PY{n}{p} \PY{c+c1}{\PYZsh{} Put}
\PY{n}{straddle} \PY{o}{=} \PY{n}{call} \PY{o}{+} \PY{n}{put} \PY{c+c1}{\PYZsh{} Straddle}

\PY{n}{plt}\PY{o}{.}\PY{n}{xlim}\PY{p}{(}\PY{l+m+mi}{40}\PY{p}{,}\PY{l+m+mi}{70}\PY{p}{)} \PY{c+c1}{\PYZsh{} Grenzen der X\PYZhy{}Achse in der Grafik}
\PY{n}{plt}\PY{o}{.}\PY{n}{ylim}\PY{p}{(}\PY{o}{\PYZhy{}}\PY{l+m+mi}{6}\PY{p}{,}\PY{l+m+mi}{20}\PY{p}{)} \PY{c+c1}{\PYZsh{} Grenzen der Y\PYZhy{}Achse in der Grafik}
\PY{n}{plt}\PY{o}{.}\PY{n}{axhline}\PY{p}{(}\PY{l+m+mi}{0}\PY{p}{,} \PY{n}{color}\PY{o}{=}\PY{l+s+s1}{\PYZsq{}}\PY{l+s+s1}{black}\PY{l+s+s1}{\PYZsq{}}\PY{p}{)} \PY{c+c1}{\PYZsh{} Plotten der X\PYZhy{}Achse}
\PY{n}{plt}\PY{o}{.}\PY{n}{plot}\PY{p}{(}\PY{n}{s}\PY{p}{,}\PY{n}{straddle}\PY{p}{,}\PY{l+s+s1}{\PYZsq{}}\PY{l+s+s1}{r}\PY{l+s+s1}{\PYZsq{}}\PY{p}{)} \PY{c+c1}{\PYZsh{} Plotten der Funktion für den Straddle}
\PY{n}{plt}\PY{o}{.}\PY{n}{plot}\PY{p}{(}\PY{p}{[}\PY{n}{b}\PY{p}{,}\PY{n}{b}\PY{p}{]}\PY{p}{,}\PY{p}{[}\PY{o}{\PYZhy{}}\PY{l+m+mi}{6}\PY{p}{,}\PY{l+m+mf}{0.5}\PY{p}{]}\PY{p}{,} \PY{l+s+s1}{\PYZsq{}}\PY{l+s+s1}{:}\PY{l+s+s1}{\PYZsq{}}\PY{p}{,} \PY{n}{color}\PY{o}{=}\PY{l+s+s1}{\PYZsq{}}\PY{l+s+s1}{green}\PY{l+s+s1}{\PYZsq{}}\PY{p}{)} \PY{c+c1}{\PYZsh{} Plotten der Geraden zur Markierung des Ausübungspreises}
\PY{n}{plt}\PY{o}{.}\PY{n}{title}\PY{p}{(}\PY{l+s+s1}{\PYZsq{}}\PY{l+s+s1}{Gewinn/Verlust\PYZhy{}Funktion für einen Straddle}\PY{l+s+s1}{\PYZsq{}}\PY{p}{)} \PY{c+c1}{\PYZsh{} Titel der Grafik }
\PY{n}{plt}\PY{o}{.}\PY{n}{xlabel}\PY{p}{(}\PY{l+s+s1}{\PYZsq{}}\PY{l+s+s1}{Preis der Aktie}\PY{l+s+s1}{\PYZsq{}}\PY{p}{)} \PY{c+c1}{\PYZsh{} Bezeichung der X\PYZhy{}Achse}
\PY{n}{plt}\PY{o}{.}\PY{n}{ylabel}\PY{p}{(}\PY{l+s+s1}{\PYZsq{}}\PY{l+s+s1}{Gewinn/Verlust}\PY{l+s+s1}{\PYZsq{}}\PY{p}{)} \PY{c+c1}{\PYZsh{} Bezeichnung der Y\PYZhy{}Achse}
\PY{n}{plt}\PY{o}{.}\PY{n}{annotate}\PY{p}{(}\PY{l+s+s1}{\PYZsq{}}\PY{l+s+s1}{Punkt 1=}\PY{l+s+s1}{\PYZsq{}} \PY{o}{+} \PY{n+nb}{str}\PY{p}{(}\PY{n}{b}\PY{o}{\PYZhy{}}\PY{n}{c}\PY{o}{\PYZhy{}}\PY{n}{p}\PY{p}{)}\PY{p}{,} \PY{n}{xy} \PY{o}{=} \PY{p}{(}\PY{n}{b}\PY{o}{\PYZhy{}}\PY{n}{p}\PY{o}{\PYZhy{}}\PY{n}{c}\PY{p}{,}\PY{l+m+mi}{0}\PY{p}{)}\PY{p}{,} \PY{n}{xytext}\PY{o}{=}\PY{p}{(}\PY{n}{b}\PY{o}{\PYZhy{}}\PY{n}{p}\PY{o}{\PYZhy{}}\PY{n}{c}\PY{p}{,}\PY{l+m+mi}{3}\PY{p}{)}\PY{p}{,} \PY{n}{arrowprops}\PY{o}{=}\PY{n+nb}{dict}\PY{p}{(}\PY{n}{facecolor}\PY{o}{=}\PY{l+s+s1}{\PYZsq{}}\PY{l+s+s1}{red}\PY{l+s+s1}{\PYZsq{}}\PY{p}{,} \PY{n}{shrink}\PY{o}{=}\PY{l+m+mf}{0.05}\PY{p}{)}\PY{p}{)} \PY{c+c1}{\PYZsh{} Beschriftung und Markierung der Funktion mit einem Pfeil}
\PY{n}{plt}\PY{o}{.}\PY{n}{annotate}\PY{p}{(}\PY{l+s+s1}{\PYZsq{}}\PY{l+s+s1}{Punkt 2=}\PY{l+s+s1}{\PYZsq{}} \PY{o}{+} \PY{n+nb}{str}\PY{p}{(}\PY{n}{b}\PY{o}{+}\PY{n}{c}\PY{o}{+}\PY{n}{p}\PY{p}{)}\PY{p}{,} \PY{n}{xy} \PY{o}{=} \PY{p}{(}\PY{n}{b}\PY{o}{+}\PY{n}{c}\PY{o}{+}\PY{n}{p}\PY{p}{,}\PY{l+m+mi}{0}\PY{p}{)}\PY{p}{,} \PY{n}{xytext}\PY{o}{=}\PY{p}{(}\PY{n}{b}\PY{o}{+}\PY{n}{c}\PY{o}{+}\PY{n}{p}\PY{p}{,}\PY{l+m+mf}{4.5}\PY{p}{)}\PY{p}{,} \PY{n}{arrowprops}\PY{o}{=}\PY{n+nb}{dict}\PY{p}{(}\PY{n}{facecolor}\PY{o}{=}\PY{l+s+s1}{\PYZsq{}}\PY{l+s+s1}{blue}\PY{l+s+s1}{\PYZsq{}}\PY{p}{,} \PY{n}{shrink}\PY{o}{=}\PY{l+m+mf}{0.05}\PY{p}{)}\PY{p}{)} \PY{c+c1}{\PYZsh{} Beschriftung und Markierung der Funktion mit einem Pfeil}
\PY{n}{plt}\PY{o}{.}\PY{n}{annotate}\PY{p}{(}\PY{l+s+s1}{\PYZsq{}}\PY{l+s+s1}{Ausübungspreis (Basispreis)= }\PY{l+s+si}{\PYZpc{}d}\PY{l+s+s1}{\PYZsq{}} \PY{o}{\PYZpc{}}\PY{k}{b}, xy = (50,\PYZhy{}5), xytext=(51.5,\PYZhy{}5), arrowprops=dict(facecolor=\PYZsq{}green\PYZsq{}, shrink=0.05)) \PYZsh{} Beschriftung und Markierung der Funktion mit einem Pfeil
\PY{n}{plt}\PY{o}{.}\PY{n}{grid}\PY{p}{(}\PY{p}{)} \PY{c+c1}{\PYZsh{} Gitternetz}
\PY{n}{plt}\PY{o}{.}\PY{n}{show}\PY{p}{(}\PY{p}{)} \PY{c+c1}{\PYZsh{} Funktion zum anzeigen der Grafik}
\end{Verbatim}
\end{tcolorbox}

    \begin{center}
    \adjustimage{max size={0.9\linewidth}{0.9\paperheight}}{output_21_0.png}
    \end{center}
    { \hspace*{\fill} \\}
    
    \hypertarget{butterfly-mit-calls}{%
\subsubsection{2.3) Butterfly mit Calls}\label{butterfly-mit-calls}}

Der Kauf zweier Calls mit den Ausübungspreisen \texttt{b1}=50 GE und
\texttt{b3}=60 GE und der Verkauf eines Calls mit dem Ausübungspreis
\texttt{b2}=(\texttt{b1}+\texttt{b3}/2)=55 GE mit derselben Fälligkeit
für dieselbe Aktie wird als Butterfly bezeichnet. Die jeweiligen Prämien
der Calls betragen: \texttt{c1}=10 GE, \texttt{c2}=7 GE, \texttt{c3}=5
GE. Die Gewinn-/Verlustfunktion dieser Optionsstratiegie lässt sich wie
folgt darstellen:

    \begin{tcolorbox}[breakable, size=fbox, boxrule=1pt, pad at break*=1mm,colback=cellbackground, colframe=cellborder]
\prompt{In}{incolor}{11}{\boxspacing}
\begin{Verbatim}[commandchars=\\\{\}]
\PY{n}{s} \PY{o}{=} \PY{n}{np}\PY{o}{.}\PY{n}{arange}\PY{p}{(}\PY{l+m+mi}{30}\PY{p}{,}\PY{l+m+mi}{80}\PY{p}{,}\PY{l+m+mi}{5}\PY{p}{)} \PY{c+c1}{\PYZsh{} Array mit möglichen Preisen einer Aktie}
\PY{n}{b1} \PY{o}{=} \PY{l+m+mi}{50}\PY{p}{;} \PY{n}{c1} \PY{o}{=} \PY{l+m+mi}{10} \PY{c+c1}{\PYZsh{} Ausübungspreis 1, Optionsprämie für einen Call 1}
\PY{n}{b2} \PY{o}{=} \PY{l+m+mi}{55}\PY{p}{;} \PY{n}{c2} \PY{o}{=} \PY{l+m+mi}{7} \PY{c+c1}{\PYZsh{} Ausübungspreis 2, Optionsprämie für einen Call 2}
\PY{n}{b3} \PY{o}{=} \PY{l+m+mi}{60}\PY{p}{;} \PY{n}{c3} \PY{o}{=} \PY{l+m+mi}{5} \PY{c+c1}{\PYZsh{} Ausübungspreis 3, Optionsprämie für einen Call 3}

\PY{n}{y1} \PY{o}{=} \PY{p}{(}\PY{n}{s} \PY{o}{\PYZhy{}} \PY{n}{b1} \PY{o}{+} \PY{n+nb}{abs}\PY{p}{(}\PY{n}{s} \PY{o}{\PYZhy{}} \PY{n}{b1}\PY{p}{)}\PY{p}{)} \PY{o}{/} \PY{l+m+mi}{2} \PY{o}{\PYZhy{}} \PY{n}{c1} \PY{c+c1}{\PYZsh{} Call blau (Long)}
\PY{n}{y2} \PY{o}{=} \PY{p}{(}\PY{n}{s} \PY{o}{\PYZhy{}} \PY{n}{b2} \PY{o}{+} \PY{n+nb}{abs}\PY{p}{(}\PY{n}{s} \PY{o}{\PYZhy{}} \PY{n}{b2}\PY{p}{)}\PY{p}{)} \PY{o}{/} \PY{l+m+mi}{2} \PY{o}{\PYZhy{}} \PY{n}{c2} \PY{c+c1}{\PYZsh{} Call orange (Short)}
\PY{n}{y3} \PY{o}{=} \PY{p}{(}\PY{n}{s} \PY{o}{\PYZhy{}} \PY{n}{b3} \PY{o}{+} \PY{n+nb}{abs}\PY{p}{(}\PY{n}{s} \PY{o}{\PYZhy{}} \PY{n}{b3}\PY{p}{)}\PY{p}{)} \PY{o}{/} \PY{l+m+mi}{2} \PY{o}{\PYZhy{}} \PY{n}{c3} \PY{c+c1}{\PYZsh{} Call grün (Long)}
\PY{n}{butterfly} \PY{o}{=} \PY{n}{y1} \PY{o}{+} \PY{n}{y3} \PY{o}{\PYZhy{}} \PY{l+m+mi}{2} \PY{o}{*} \PY{n}{y2} \PY{c+c1}{\PYZsh{} Butterfly}

\PY{n}{plt}\PY{o}{.}\PY{n}{xlim}\PY{p}{(}\PY{l+m+mi}{40}\PY{p}{,}\PY{l+m+mi}{70}\PY{p}{)} \PY{c+c1}{\PYZsh{} Grenzen der X\PYZhy{}Achse in der Grafik}
\PY{n}{plt}\PY{o}{.}\PY{n}{ylim}\PY{p}{(}\PY{o}{\PYZhy{}}\PY{l+m+mi}{20}\PY{p}{,}\PY{l+m+mi}{20}\PY{p}{)} \PY{c+c1}{\PYZsh{} Grenzen der Y\PYZhy{}Achse in der Grafik}
\PY{n}{plt}\PY{o}{.}\PY{n}{axhline}\PY{p}{(}\PY{l+m+mi}{0}\PY{p}{,} \PY{n}{color}\PY{o}{=}\PY{l+s+s1}{\PYZsq{}}\PY{l+s+s1}{black}\PY{l+s+s1}{\PYZsq{}}\PY{p}{)} \PY{c+c1}{\PYZsh{} Plotten der X\PYZhy{}Achse}
\PY{n}{plt}\PY{o}{.}\PY{n}{plot}\PY{p}{(}\PY{n}{s}\PY{p}{,}\PY{n}{y1}\PY{p}{)} \PY{c+c1}{\PYZsh{} Plotten der Funktion für blauen Call}
\PY{n}{plt}\PY{o}{.}\PY{n}{plot}\PY{p}{(}\PY{n}{s}\PY{p}{,}\PY{o}{\PYZhy{}}\PY{n}{y2}\PY{p}{)} \PY{c+c1}{\PYZsh{} Plotten der Funktion für den orangenen Call}
\PY{n}{plt}\PY{o}{.}\PY{n}{plot}\PY{p}{(}\PY{n}{s}\PY{p}{,}\PY{n}{y3}\PY{p}{)} \PY{c+c1}{\PYZsh{} Plotten der Funktion für den grünen Call }
\PY{n}{plt}\PY{o}{.}\PY{n}{plot}\PY{p}{(}\PY{n}{s}\PY{p}{,}\PY{n}{butterfly}\PY{p}{,}\PY{l+s+s1}{\PYZsq{}}\PY{l+s+s1}{r}\PY{l+s+s1}{\PYZsq{}}\PY{p}{)} \PY{c+c1}{\PYZsh{} Plotten der Funktion für den Butterlfy}
\PY{n}{plt}\PY{o}{.}\PY{n}{title}\PY{p}{(}\PY{l+s+s1}{\PYZsq{}}\PY{l+s+s1}{Gewinn\PYZhy{}Verlust für einen Butterfly}\PY{l+s+s1}{\PYZsq{}}\PY{p}{)} \PY{c+c1}{\PYZsh{} Titel der Grafik }
\PY{n}{plt}\PY{o}{.}\PY{n}{xlabel}\PY{p}{(}\PY{l+s+s1}{\PYZsq{}}\PY{l+s+s1}{Preis der Aktie}\PY{l+s+s1}{\PYZsq{}}\PY{p}{)} \PY{c+c1}{\PYZsh{} Bezeichung der X\PYZhy{}Achse}
\PY{n}{plt}\PY{o}{.}\PY{n}{ylabel}\PY{p}{(}\PY{l+s+s1}{\PYZsq{}}\PY{l+s+s1}{Gewinn/Verlust}\PY{l+s+s1}{\PYZsq{}}\PY{p}{)} \PY{c+c1}{\PYZsh{} Bezeichnung der Y\PYZhy{}Achse}
\PY{n}{plt}\PY{o}{.}\PY{n}{annotate}\PY{p}{(}\PY{l+s+s1}{\PYZsq{}}\PY{l+s+s1}{Butterfly}\PY{l+s+s1}{\PYZsq{}}\PY{p}{,} \PY{n}{xy} \PY{o}{=} \PY{p}{(}\PY{l+m+mi}{56}\PY{p}{,}\PY{l+m+mi}{3}\PY{p}{)}\PY{p}{,} \PY{n}{xytext}\PY{o}{=}\PY{p}{(}\PY{l+m+mi}{60}\PY{p}{,}\PY{l+m+mi}{8}\PY{p}{)}\PY{p}{,} \PY{n}{arrowprops}\PY{o}{=}\PY{n+nb}{dict}\PY{p}{(}\PY{n}{facecolor}\PY{o}{=}\PY{l+s+s1}{\PYZsq{}}\PY{l+s+s1}{red}\PY{l+s+s1}{\PYZsq{}}\PY{p}{,} \PY{n}{shrink}\PY{o}{=}\PY{l+m+mf}{0.05}\PY{p}{)}\PY{p}{)} \PY{c+c1}{\PYZsh{} Beschriftung und Markierung der Funktion mit einem Pfeil}
\PY{n}{plt}\PY{o}{.}\PY{n}{grid}\PY{p}{(}\PY{p}{)} \PY{c+c1}{\PYZsh{} Gitternetz}
\PY{n}{plt}\PY{o}{.}\PY{n}{show}\PY{p}{(}\PY{p}{)} \PY{c+c1}{\PYZsh{} Funktion zum anzeigen der Grafik}
\end{Verbatim}
\end{tcolorbox}

    \begin{center}
    \adjustimage{max size={0.9\linewidth}{0.9\paperheight}}{output_23_0.png}
    \end{center}
    { \hspace*{\fill} \\}
    
    \hypertarget{aufgabe-3---put-call-parituxe4t-und-deren-graphische-repruxe4sentation}{%
\subsection{Aufgabe 3 - Put-Call-Parität und deren graphische
Repräsentation}\label{aufgabe-3---put-call-parituxe4t-und-deren-graphische-repruxe4sentation}}

Variablenverzeichnis: - b = Ausübungspreis (Basispreis) - roh =
risikoloser Zinssatz - t = Fälligkeit in Jahren - s = Preis der Aktie
zum Fälligkeitsdatum

\hypertarget{beruxfccksichtigung-des-present-value}{%
\subsubsection{3.1) Berücksichtigung des Present
Value}\label{beruxfccksichtigung-des-present-value}}

Annahme: Es wird ein Call mit einem Ausübungspreis i.H.v. 20 GE und
einer Fälligkeit von drei Monaten betrachtet. Der risikolose Zinssatz
beträgt 5\%, sodass sich der folgende Barwert ergbit:

    \begin{tcolorbox}[breakable, size=fbox, boxrule=1pt, pad at break*=1mm,colback=cellbackground, colframe=cellborder]
\prompt{In}{incolor}{12}{\boxspacing}
\begin{Verbatim}[commandchars=\\\{\}]
\PY{n}{b} \PY{o}{=} \PY{l+m+mi}{20} \PY{c+c1}{\PYZsh{} Ausübungspreis (Basispreis)}
\PY{n}{t} \PY{o}{=} \PY{l+m+mi}{3} \PY{o}{/} \PY{l+m+mi}{12} \PY{c+c1}{\PYZsh{} Fälligkeit }
\PY{n}{roh} \PY{o}{=} \PY{l+m+mf}{0.05} \PY{c+c1}{\PYZsh{} Risikoloser Zinssatz}

\PY{n}{b1} \PY{o}{=} \PY{n}{b} \PY{o}{*} \PY{n}{math}\PY{o}{.}\PY{n}{exp}\PY{p}{(}\PY{o}{\PYZhy{}} \PY{n}{roh} \PY{o}{*} \PY{n}{t}\PY{p}{)} \PY{c+c1}{\PYZsh{} Diskontierter Ausübungspreis}
\PY{n}{b1} \PY{o}{=} \PY{n+nb}{round}\PY{p}{(}\PY{n}{b1}\PY{p}{,}\PY{l+m+mi}{2}\PY{p}{)} \PY{c+c1}{\PYZsh{} Gerundeter diskontierter Ausübungspreis}

\PY{n+nb}{print}\PY{p}{(}\PY{l+s+s1}{\PYZsq{}}\PY{l+s+s1}{\PYZsh{}}\PY{l+s+s1}{\PYZsq{}} \PY{o}{+} \PY{n}{SCREEN\PYZus{}WIDTH} \PY{o}{*} \PY{l+s+s1}{\PYZsq{}}\PY{l+s+s1}{\PYZhy{}}\PY{l+s+s1}{\PYZsq{}} \PY{o}{+} \PY{l+s+s1}{\PYZsq{}}\PY{l+s+s1}{\PYZsh{}}\PY{l+s+s1}{\PYZsq{}}\PY{p}{)}
\PY{n+nb}{print}\PY{p}{(}\PY{l+s+s1}{\PYZsq{}}\PY{l+s+s1}{|}\PY{l+s+s1}{\PYZsq{}} \PY{o}{+} \PY{n}{centered}\PY{p}{(}\PY{l+s+s1}{\PYZsq{}}\PY{l+s+s1}{Der Present Value des Ausübungspreises i.H.v. }\PY{l+s+si}{\PYZpc{}d}\PY{l+s+s1}{\PYZsq{}} \PY{o}{\PYZpc{}}\PY{k}{b} + \PYZsq{} GE beträgt nach \PYZpc{}f\PYZsq{} \PYZpc{}t + \PYZsq{} Jahren : \PYZsq{} + str(b1) + \PYZsq{} GE\PYZsq{}) + \PYZsq{}| \PYZsq{})
\PY{n+nb}{print}\PY{p}{(}\PY{l+s+s1}{\PYZsq{}}\PY{l+s+s1}{\PYZsh{}}\PY{l+s+s1}{\PYZsq{}} \PY{o}{+} \PY{n}{SCREEN\PYZus{}WIDTH} \PY{o}{*} \PY{l+s+s1}{\PYZsq{}}\PY{l+s+s1}{\PYZhy{}}\PY{l+s+s1}{\PYZsq{}} \PY{o}{+} \PY{l+s+s1}{\PYZsq{}}\PY{l+s+s1}{\PYZsh{}}\PY{l+s+s1}{\PYZsq{}}\PY{p}{)}
\end{Verbatim}
\end{tcolorbox}

    \begin{Verbatim}[commandchars=\\\{\}]
\#-------------------------------------------------------------------------------
----------------------------------------------\#
|                 Der Present Value des Ausübungspreises i.H.v. 20 GE beträgt
nach 0.250000 Jahren : 19.75 GE                 |
\#-------------------------------------------------------------------------------
----------------------------------------------\#
    \end{Verbatim}

    \hypertarget{graphische-veranschaulichung-der-put-call-parituxe4t}{%
\subsubsection{3.2) Graphische Veranschaulichung der Put-Call
Parität}\label{graphische-veranschaulichung-der-put-call-parituxe4t}}

In der folgenden Grafik wird veranschaulicht, dass zwischen Put- und
Call-Optionen eine feste Beziehung besteht. Sofern der Put und der Call
sich auf denselben Basiswert beziehen, den selben Strike haben und eine
identische Laufzeit ausweisen, entspricht der Call + der Barwert des
abgezinsten Strikes (Cash) dem Put + dem Kassapreis des Barwertes
(Aktie). Beide Stragegien führen somit zum selben Ergebnis. 1. Fall: Bei
einem Aktienpreis von unter 20 GE wird der Call auf diese Aktie nicht
ausgeübt, das vorhandene Cash wird behalten; bei einem Aktienpreis von
über 20 GE wird der Call auf diese Aktie unter Verwendung des Cashs
ausgeübt -\textgreater{} Es wird das Maximum aus Aktienpreis und Cash
ausgezahlt 2. Fall: Bei einem Aktienpreis von unter 20 GE wird der Put
auf diese Aktie ausgeübt, es werden 20 GE ausgezahlt; bei einem
Aktienpreis von über 20 GE wird diese behalten -\textgreater{} Es wird
das Maximum aus dem Payoff des Puts und dem Aktienpreis ausgezahlt

    \begin{tcolorbox}[breakable, size=fbox, boxrule=1pt, pad at break*=1mm,colback=cellbackground, colframe=cellborder]
\prompt{In}{incolor}{13}{\boxspacing}
\begin{Verbatim}[commandchars=\\\{\}]
\PY{n}{b} \PY{o}{=} \PY{l+m+mi}{10} \PY{c+c1}{\PYZsh{} Ausübungspreis (Basispreis)}
\PY{n}{s} \PY{o}{=} \PY{n}{np}\PY{o}{.}\PY{n}{arange}\PY{p}{(}\PY{l+m+mi}{0}\PY{p}{,}\PY{l+m+mi}{30}\PY{p}{,}\PY{l+m+mi}{5}\PY{p}{)} \PY{c+c1}{\PYZsh{} Array mit möglichen Preisen einer Aktie}

\PY{n}{call\PYZus{}payoff} \PY{o}{=} \PY{p}{(}\PY{n}{s} \PY{o}{\PYZhy{}} \PY{n}{b} \PY{o}{+} \PY{n+nb}{abs}\PY{p}{(}\PY{n}{s} \PY{o}{\PYZhy{}} \PY{n}{b}\PY{p}{)}\PY{p}{)} \PY{o}{/} \PY{l+m+mi}{2} \PY{c+c1}{\PYZsh{} Payoff des Calls}
\PY{n}{put\PYZus{}payoff} \PY{o}{=} \PY{p}{(}\PY{n}{b} \PY{o}{\PYZhy{}} \PY{n}{s} \PY{o}{+} \PY{n+nb}{abs}\PY{p}{(}\PY{n}{b} \PY{o}{\PYZhy{}} \PY{n}{s}\PY{p}{)}\PY{p}{)} \PY{o}{/} \PY{l+m+mi}{2}  \PY{c+c1}{\PYZsh{} Payoff des Puts}
\PY{n}{cash} \PY{o}{=} \PY{n}{np}\PY{o}{.}\PY{n}{zeros}\PY{p}{(}\PY{n+nb}{len}\PY{p}{(}\PY{n}{s}\PY{p}{)}\PY{p}{)} \PY{o}{+} \PY{n}{b} \PY{c+c1}{\PYZsh{} Cash}

\PY{c+c1}{\PYZsh{} Funktion, welche die Platzierrung und Formatierung der Grafik vereinfacht}
\PY{k}{def} \PY{n+nf}{graph}\PY{p}{(}\PY{n}{text}\PY{p}{,} \PY{n}{text2}\PY{o}{=}\PY{l+s+s1}{\PYZsq{}}\PY{l+s+s1}{\PYZsq{}}\PY{p}{)}\PY{p}{:}
    \PY{n}{pl}\PY{o}{.}\PY{n}{xticks}\PY{p}{(}\PY{p}{(}\PY{p}{)}\PY{p}{)} \PY{c+c1}{\PYZsh{} Blendet die Beschriftung der X\PYZhy{}Achse aus}
    \PY{n}{pl}\PY{o}{.}\PY{n}{yticks}\PY{p}{(}\PY{p}{(}\PY{p}{)}\PY{p}{)} \PY{c+c1}{\PYZsh{} Blendet die Beschriftung der Y\PYZhy{}Achse aus}
    \PY{n}{pl}\PY{o}{.}\PY{n}{xlim}\PY{p}{(}\PY{l+m+mi}{0}\PY{p}{,}\PY{l+m+mi}{30}\PY{p}{)} \PY{c+c1}{\PYZsh{} Grenzen der X\PYZhy{}Achse in der Grafik}
    \PY{n}{pl}\PY{o}{.}\PY{n}{ylim}\PY{p}{(}\PY{l+m+mi}{0}\PY{p}{,}\PY{l+m+mi}{20}\PY{p}{)} \PY{c+c1}{\PYZsh{} Grenzen der Y\PYZhy{}Achse in der Grafik}
    \PY{n}{pl}\PY{o}{.}\PY{n}{plot}\PY{p}{(}\PY{p}{[}\PY{n}{b}\PY{p}{,}\PY{n}{b}\PY{p}{]}\PY{p}{,}\PY{p}{[}\PY{l+m+mi}{0}\PY{p}{,}\PY{l+m+mi}{3}\PY{p}{]}\PY{p}{)} \PY{c+c1}{\PYZsh{} Plottet die senkrechte, blaue Gerade}
    \PY{n}{pl}\PY{o}{.}\PY{n}{text}\PY{p}{(}\PY{n}{b}\PY{p}{,}\PY{o}{\PYZhy{}}\PY{l+m+mi}{2}\PY{p}{,}\PY{l+s+s1}{\PYZsq{}}\PY{l+s+s1}{b}\PY{l+s+s1}{\PYZsq{}}\PY{p}{)} \PY{c+c1}{\PYZsh{} Plottet das \PYZdq{}X\PYZdq{} auf der Y\PYZhy{}Achse}
    \PY{n}{pl}\PY{o}{.}\PY{n}{text}\PY{p}{(}\PY{l+m+mi}{0}\PY{p}{,}\PY{n}{b}\PY{p}{,}\PY{l+s+s1}{\PYZsq{}}\PY{l+s+s1}{b}\PY{l+s+s1}{\PYZsq{}}\PY{p}{)}  \PY{c+c1}{\PYZsh{} Plottet das \PYZdq{}X\PYZdq{} auf der X\PYZhy{}Achse}
    \PY{n}{pl}\PY{o}{.}\PY{n}{text}\PY{p}{(}\PY{n}{b}\PY{p}{,}\PY{n}{b} \PY{o}{*} \PY{l+m+mf}{1.7}\PY{p}{,} \PY{n}{text}\PY{p}{,} \PY{n}{ha}\PY{o}{=}\PY{l+s+s1}{\PYZsq{}}\PY{l+s+s1}{center}\PY{l+s+s1}{\PYZsq{}}\PY{p}{,} \PY{n}{va}\PY{o}{=}\PY{l+s+s1}{\PYZsq{}}\PY{l+s+s1}{center}\PY{l+s+s1}{\PYZsq{}}\PY{p}{,} \PY{n}{size}\PY{o}{=}\PY{l+m+mi}{15}\PY{p}{,} \PY{n}{alpha}\PY{o}{=}\PY{l+m+mf}{0.5}\PY{p}{)} \PY{c+c1}{\PYZsh{} Plottet und Positioniert die Beschriftung der jeweiligen Grafik}
    \PY{n}{pl}\PY{o}{.}\PY{n}{text}\PY{p}{(}\PY{o}{\PYZhy{}}\PY{l+m+mi}{5}\PY{p}{,}\PY{l+m+mi}{10}\PY{p}{,}\PY{n}{text2}\PY{p}{,}\PY{n}{size}\PY{o}{=}\PY{l+m+mi}{25}\PY{p}{)} \PY{c+c1}{\PYZsh{} Plottet die Plus\PYZhy{} und die Minuszeichen}

\PY{n}{pl}\PY{o}{.}\PY{n}{subplot}\PY{p}{(}\PY{l+m+mi}{2}\PY{p}{,}\PY{l+m+mi}{3}\PY{p}{,}\PY{l+m+mi}{1}\PY{p}{)}\PY{p}{;} \PY{n}{graph}\PY{p}{(}\PY{l+s+s1}{\PYZsq{}}\PY{l+s+s1}{Payoff des Calls}\PY{l+s+s1}{\PYZsq{}}\PY{p}{)}\PY{p}{;} \PY{n}{pl}\PY{o}{.}\PY{n}{plot}\PY{p}{(}\PY{n}{s}\PY{p}{,} \PY{n}{call\PYZus{}payoff}\PY{p}{)} \PY{c+c1}{\PYZsh{} Plottet die Grafik mit dem Payoff des Calls (Oben)}
\PY{n}{pl}\PY{o}{.}\PY{n}{subplot}\PY{p}{(}\PY{l+m+mi}{2}\PY{p}{,}\PY{l+m+mi}{3}\PY{p}{,}\PY{l+m+mi}{2}\PY{p}{)}\PY{p}{;} \PY{n}{graph}\PY{p}{(}\PY{l+s+s1}{\PYZsq{}}\PY{l+s+s1}{Cash}\PY{l+s+s1}{\PYZsq{}}\PY{p}{,} \PY{l+s+s1}{\PYZsq{}}\PY{l+s+s1}{+}\PY{l+s+s1}{\PYZsq{}}\PY{p}{)}\PY{p}{;} \PY{n}{pl}\PY{o}{.}\PY{n}{plot}\PY{p}{(}\PY{n}{s}\PY{p}{,} \PY{n}{cash}\PY{p}{)} \PY{c+c1}{\PYZsh{} Plottet die Grafik mit dem Cash (Oben)}
\PY{n}{pl}\PY{o}{.}\PY{n}{subplot}\PY{p}{(}\PY{l+m+mi}{2}\PY{p}{,}\PY{l+m+mi}{3}\PY{p}{,}\PY{l+m+mi}{3}\PY{p}{)}\PY{p}{;} \PY{n}{graph}\PY{p}{(}\PY{l+s+s1}{\PYZsq{}}\PY{l+s+s1}{Portfolio A }\PY{l+s+s1}{\PYZsq{}}\PY{p}{,} \PY{l+s+s1}{\PYZsq{}}\PY{l+s+s1}{=}\PY{l+s+s1}{\PYZsq{}}\PY{p}{)}\PY{p}{;} \PY{n}{pl}\PY{o}{.}\PY{n}{plot}\PY{p}{(}\PY{n}{s}\PY{p}{,} \PY{n}{cash} \PY{o}{+} \PY{n}{call\PYZus{}payoff}\PY{p}{)} \PY{c+c1}{\PYZsh{} Plottet die Grafik mit der Kombination aus Call und Cash (Oben)}
\PY{n}{pl}\PY{o}{.}\PY{n}{subplot}\PY{p}{(}\PY{l+m+mi}{2}\PY{p}{,}\PY{l+m+mi}{3}\PY{p}{,}\PY{l+m+mi}{4}\PY{p}{)}\PY{p}{;} \PY{n}{graph}\PY{p}{(}\PY{l+s+s1}{\PYZsq{}}\PY{l+s+s1}{Payoff des Puts}\PY{l+s+s1}{\PYZsq{}}\PY{p}{)}\PY{p}{;} \PY{n}{pl}\PY{o}{.}\PY{n}{plot}\PY{p}{(}\PY{n}{s}\PY{p}{,} \PY{n}{put\PYZus{}payoff}\PY{p}{)}  \PY{c+c1}{\PYZsh{} Plottet die Grafik mit dem Payoff des Puts (Unten)}
\PY{n}{pl}\PY{o}{.}\PY{n}{subplot}\PY{p}{(}\PY{l+m+mi}{2}\PY{p}{,}\PY{l+m+mi}{3}\PY{p}{,}\PY{l+m+mi}{5}\PY{p}{)}\PY{p}{;} \PY{n}{graph}\PY{p}{(}\PY{l+s+s1}{\PYZsq{}}\PY{l+s+s1}{Aktie}\PY{l+s+s1}{\PYZsq{}}\PY{p}{,} \PY{l+s+s1}{\PYZsq{}}\PY{l+s+s1}{+}\PY{l+s+s1}{\PYZsq{}}\PY{p}{)}\PY{p}{;} \PY{n}{pl}\PY{o}{.}\PY{n}{plot}\PY{p}{(}\PY{n}{s}\PY{p}{,} \PY{n}{s}\PY{p}{)} \PY{c+c1}{\PYZsh{} Plottet die Grafik mit der Aktie (Unten)}
\PY{n}{pl}\PY{o}{.}\PY{n}{subplot}\PY{p}{(}\PY{l+m+mi}{2}\PY{p}{,}\PY{l+m+mi}{3}\PY{p}{,}\PY{l+m+mi}{6}\PY{p}{)}\PY{p}{;} \PY{n}{graph}\PY{p}{(}\PY{l+s+s1}{\PYZsq{}}\PY{l+s+s1}{Portfolio B}\PY{l+s+s1}{\PYZsq{}}\PY{p}{,} \PY{l+s+s1}{\PYZsq{}}\PY{l+s+s1}{=}\PY{l+s+s1}{\PYZsq{}}\PY{p}{)}\PY{p}{;} \PY{n}{pl}\PY{o}{.}\PY{n}{plot}\PY{p}{(}\PY{n}{s}\PY{p}{,} \PY{n}{s} \PY{o}{+} \PY{n}{put\PYZus{}payoff}\PY{p}{)}  \PY{c+c1}{\PYZsh{} Plottet die Grafik mit der Kombination aus Put und Aktie (Unten)}
\PY{n}{pl}\PY{o}{.}\PY{n}{show}\PY{p}{(}\PY{p}{)} \PY{c+c1}{\PYZsh{} Funktion zum anzeigen der Grafik}
\end{Verbatim}
\end{tcolorbox}

    \begin{center}
    \adjustimage{max size={0.9\linewidth}{0.9\paperheight}}{output_27_0.png}
    \end{center}
    { \hspace*{\fill} \\}
    
    \hypertarget{aufagbe-4---implizite-volatilituxe4t}{%
\subsection{Aufagbe 4 - Implizite
Volatilität}\label{aufagbe-4---implizite-volatilituxe4t}}

Variablenverzeichnis: - s = Preis der Aktie zum Fälligkeitsdatum - b =
Ausübungspreis (Basispreis) - t = Fälligkeit in Jahren - roh =
risikofreier Zinssatz - sigma = Volatilität der Aktie, (annualisierte
Standardabweichung der Renditen) - c = Optionsprämie für einen Call - p
= Options-Prämie für einen Put

Ein Parameter kann aus der Black-Scholes-Formel nicht direkt beobachtet
werden: die Volatilität des Aktienkurses. In der Realität wird mit
impliziten Volatilitäten gearbeitet. Diese sind Volatilitäten, die in
den am Markt beobachteten Optionespreisen enthalten sind. Implizite
Volatilitäten werden verwendet, um die Marktmeinung über die Volatilität
einer bestimmten Aktie zu beobachten. Während historische Volatilitäten
rückwirkend ermittelt werden, blicken implizite Volatilitäten in die
Zukunft.

\hypertarget{schuxe4tzung-des-preises-einer-europuxe4ischen-call-option}{%
\subsubsection{4.1) Schätzung des Preises einer europäischen
Call-Option}\label{schuxe4tzung-des-preises-einer-europuxe4ischen-call-option}}

Basierend auf dem Black-Scholes-Merton-Modell lässt sich der Preis einer
europäischen Call-Option in Python wie folgt schätzen: Annahme: s=40;
b=40; t=0.5; roh=0.05; sigma=0.25

    \begin{tcolorbox}[breakable, size=fbox, boxrule=1pt, pad at break*=1mm,colback=cellbackground, colframe=cellborder]
\prompt{In}{incolor}{14}{\boxspacing}
\begin{Verbatim}[commandchars=\\\{\}]
\PY{c+c1}{\PYZsh{} Definition der Funktion \PYZdq{}bs\PYZus{}call\PYZdq{}, welche die zu entrichtende Prämie einer Call\PYZhy{}Option betimmt}
\PY{k}{def} \PY{n+nf}{bs\PYZus{}call}\PY{p}{(}\PY{n}{s}\PY{p}{,}\PY{n}{b}\PY{p}{,}\PY{n}{t}\PY{p}{,}\PY{n}{roh}\PY{p}{,}\PY{n}{sigma}\PY{p}{)}\PY{p}{:}
    \PY{n}{d1} \PY{o}{=} \PY{p}{(}\PY{n}{math}\PY{o}{.}\PY{n}{log}\PY{p}{(}\PY{n}{s} \PY{o}{/} \PY{n}{b}\PY{p}{)} \PY{o}{+} \PY{p}{(}\PY{n}{roh} \PY{o}{+} \PY{n}{sigma} \PY{o}{*} \PY{n}{sigma} \PY{o}{/} \PY{l+m+mi}{2}\PY{p}{)} \PY{o}{*} \PY{n}{t} \PY{p}{)} \PY{o}{/} \PY{p}{(}\PY{n}{sigma} \PY{o}{*} \PY{n}{math}\PY{o}{.}\PY{n}{sqrt}\PY{p}{(}\PY{n}{t}\PY{p}{)}\PY{p}{)} \PY{c+c1}{\PYZsh{} Berechnung d1}
    \PY{n}{d2} \PY{o}{=} \PY{n}{d1} \PY{o}{\PYZhy{}} \PY{n}{sigma} \PY{o}{*} \PY{n}{math}\PY{o}{.}\PY{n}{sqrt}\PY{p}{(}\PY{n}{t}\PY{p}{)} \PY{c+c1}{\PYZsh{} Berechnung d2}
    \PY{k}{return} \PY{n}{s} \PY{o}{*} \PY{n}{stats}\PY{o}{.}\PY{n}{norm}\PY{o}{.}\PY{n}{cdf}\PY{p}{(}\PY{n}{d1}\PY{p}{)} \PY{o}{\PYZhy{}} \PY{n}{b} \PY{o}{*} \PY{n}{math}\PY{o}{.}\PY{n}{exp}\PY{p}{(}\PY{o}{\PYZhy{}}\PY{n}{roh} \PY{o}{*} \PY{n}{t}\PY{p}{)} \PY{o}{*} \PY{n}{stats}\PY{o}{.}\PY{n}{norm}\PY{o}{.}\PY{n}{cdf}\PY{p}{(}\PY{n}{d2}\PY{p}{)} \PY{c+c1}{\PYZsh{} Rückgabe des in dieser Zeile berechneten Wertes}

\PY{c+c1}{\PYZsh{} Ausführung der Funktion}
\PY{n}{price} \PY{o}{=} \PY{n}{bs\PYZus{}call}\PY{p}{(}\PY{l+m+mi}{40}\PY{p}{,}\PY{l+m+mi}{40}\PY{p}{,}\PY{l+m+mf}{0.5}\PY{p}{,}\PY{l+m+mf}{0.05}\PY{p}{,}\PY{l+m+mf}{0.25}\PY{p}{)}

\PY{c+c1}{\PYZsh{} Ausgabe des Ergebnises}
\PY{n+nb}{print}\PY{p}{(}\PY{l+s+s1}{\PYZsq{}}\PY{l+s+s1}{\PYZsh{}}\PY{l+s+s1}{\PYZsq{}} \PY{o}{+} \PY{n}{SCREEN\PYZus{}WIDTH} \PY{o}{*} \PY{l+s+s1}{\PYZsq{}}\PY{l+s+s1}{\PYZhy{}}\PY{l+s+s1}{\PYZsq{}} \PY{o}{+} \PY{l+s+s1}{\PYZsq{}}\PY{l+s+s1}{\PYZsh{}}\PY{l+s+s1}{\PYZsq{}}\PY{p}{)}
\PY{n+nb}{print}\PY{p}{(}\PY{l+s+s1}{\PYZsq{}}\PY{l+s+s1}{|}\PY{l+s+s1}{\PYZsq{}} \PY{o}{+} \PY{n}{centered}\PY{p}{(}\PY{l+s+s1}{\PYZsq{}}\PY{l+s+s1}{Der Preis der europäischen Call\PYZhy{}Option beträgt }\PY{l+s+si}{\PYZpc{}f}\PY{l+s+s1}{\PYZsq{}} \PY{o}{\PYZpc{}}\PY{k}{price} + \PYZsq{} GE\PYZsq{}) + \PYZsq{}| \PYZsq{})
\PY{n+nb}{print}\PY{p}{(}\PY{l+s+s1}{\PYZsq{}}\PY{l+s+s1}{\PYZsh{}}\PY{l+s+s1}{\PYZsq{}} \PY{o}{+} \PY{n}{SCREEN\PYZus{}WIDTH} \PY{o}{*} \PY{l+s+s1}{\PYZsq{}}\PY{l+s+s1}{\PYZhy{}}\PY{l+s+s1}{\PYZsq{}} \PY{o}{+} \PY{l+s+s1}{\PYZsq{}}\PY{l+s+s1}{\PYZsh{}}\PY{l+s+s1}{\PYZsq{}}\PY{p}{)}
\end{Verbatim}
\end{tcolorbox}

    \begin{Verbatim}[commandchars=\\\{\}]
\#-------------------------------------------------------------------------------
----------------------------------------------\#
|                                  Der Preis der europäischen Call-Option
beträgt 3.304006 GE                                 |
\#-------------------------------------------------------------------------------
----------------------------------------------\#
    \end{Verbatim}

    \hypertarget{schuxe4tzung-der-impliziten-volatilituxe4t-einer-europuxe4ischen-option-bei-sonst-gegebenen-parametern}{%
\subsubsection{4.2) Schätzung der impliziten Volatilität einer
europäischen Option bei sonst gegebenen
Parametern}\label{schuxe4tzung-der-impliziten-volatilituxe4t-einer-europuxe4ischen-option-bei-sonst-gegebenen-parametern}}

Annahme: Gegeben sind: s=40, b=40, t=0.5, roh=0.05, und c=3.30. Gesucht
ist sigma, welches nach obiger Rechnung zufolge den Wert 0.25 annehmen
sollte.

\hypertarget{betrachtung-einer-call-option}{%
\paragraph{Betrachtung einer Call-Option
:}\label{betrachtung-einer-call-option}}

Der Funktion ``implied\_vol\_calls()'' werden die bekannten Parameter
s,b,t,roh und c übergeben. In jeder der 100 Iteration wird systematisch
ein sigma bestimmt auf Basis dessen die Prämie des Calls berechnet wird.
Die Berechnung wird gestoppt, sobald die absolute Differenz zwischen dem
so berechneten und dem vorgegebenen Call-Preis einen kritischen Wert
unterschreitet. In diesem Fall wurder dieser kritische Wert auf 0.01 GE
festgelegt.

    \begin{tcolorbox}[breakable, size=fbox, boxrule=1pt, pad at break*=1mm,colback=cellbackground, colframe=cellborder]
\prompt{In}{incolor}{15}{\boxspacing}
\begin{Verbatim}[commandchars=\\\{\}]
\PY{c+c1}{\PYZsh{} Definition der Funktion \PYZdq{}implied\PYZus{}vol\PYZus{}calls\PYZdq{}, welche die implizite Volatilität einer Call\PYZhy{}Option betimmt}
\PY{k}{def} \PY{n+nf}{implied\PYZus{}vol\PYZus{}calls}\PY{p}{(}\PY{n}{s}\PY{p}{,}\PY{n}{x}\PY{p}{,}\PY{n}{t}\PY{p}{,}\PY{n}{roh}\PY{p}{,}\PY{n}{c}\PY{p}{)}\PY{p}{:}
    \PY{c+c1}{\PYZsh{} Führe die folgende Berechnung 100 mal durch}
    \PY{k}{for} \PY{n}{i} \PY{o+ow}{in} \PY{n+nb}{range}\PY{p}{(}\PY{l+m+mi}{100}\PY{p}{)}\PY{p}{:}
        \PY{n}{sigma} \PY{o}{=} \PY{l+m+mf}{0.005} \PY{o}{*} \PY{p}{(}\PY{n}{i} \PY{o}{+} \PY{l+m+mi}{1}\PY{p}{)} \PY{c+c1}{\PYZsh{} Bestimmung sigma}
        \PY{n}{d1} \PY{o}{=} \PY{p}{(}\PY{n}{math}\PY{o}{.}\PY{n}{log}\PY{p}{(}\PY{n}{s} \PY{o}{/} \PY{n}{x}\PY{p}{)} \PY{o}{+} \PY{p}{(}\PY{n}{roh} \PY{o}{+} \PY{n}{sigma} \PY{o}{*} \PY{n}{sigma} \PY{o}{/} \PY{l+m+mi}{2}\PY{p}{)} \PY{o}{*} \PY{n}{t} \PY{p}{)} \PY{o}{/} \PY{p}{(}\PY{n}{sigma} \PY{o}{*} \PY{n}{math}\PY{o}{.}\PY{n}{sqrt}\PY{p}{(}\PY{n}{t}\PY{p}{)}\PY{p}{)} \PY{c+c1}{\PYZsh{} Berechnung d1}
        \PY{n}{d2} \PY{o}{=} \PY{n}{d1} \PY{o}{\PYZhy{}} \PY{n}{sigma} \PY{o}{*} \PY{n}{math}\PY{o}{.}\PY{n}{sqrt}\PY{p}{(}\PY{n}{t}\PY{p}{)} \PY{c+c1}{\PYZsh{} Berechnung d2}
        \PY{n}{diff} \PY{o}{=} \PY{n}{c} \PY{o}{\PYZhy{}} \PY{p}{(}\PY{n}{s} \PY{o}{*} \PY{n}{stats}\PY{o}{.}\PY{n}{norm}\PY{o}{.}\PY{n}{cdf}\PY{p}{(}\PY{n}{d1}\PY{p}{)} \PY{o}{\PYZhy{}} \PY{n}{x} \PY{o}{*} \PY{n}{math}\PY{o}{.}\PY{n}{exp}\PY{p}{(}\PY{o}{\PYZhy{}}\PY{n}{roh} \PY{o}{*} \PY{n}{t}\PY{p}{)} \PY{o}{*} \PY{n}{stats}\PY{o}{.}\PY{n}{norm}\PY{o}{.}\PY{n}{cdf}\PY{p}{(}\PY{n}{d2}\PY{p}{)}\PY{p}{)} \PY{c+c1}{\PYZsh{} Berechnung Differenz zwischen gegebenen Call\PYZhy{}Preis (3.30) und bestimmten Call\PYZhy{}Preis}
        \PY{c+c1}{\PYZsh{} Abbruchkriterium}
        \PY{k}{if} \PY{n+nb}{abs}\PY{p}{(}\PY{n}{diff}\PY{p}{)} \PY{o}{\PYZlt{}}\PY{o}{=} \PY{l+m+mf}{0.01}\PY{p}{:}
            \PY{k}{return} \PY{p}{(}\PY{n}{i}\PY{p}{,} \PY{n}{sigma}\PY{p}{,} \PY{n}{diff}\PY{p}{)} \PY{c+c1}{\PYZsh{} Rückgabe des in dieser Zeile berechneten Wertes}

\PY{c+c1}{\PYZsh{} Ausführung der Funktion}
\PY{n}{i}\PY{p}{,} \PY{n}{sigma}\PY{p}{,} \PY{n}{diff} \PY{o}{=} \PY{n}{implied\PYZus{}vol\PYZus{}calls}\PY{p}{(}\PY{l+m+mi}{40}\PY{p}{,}\PY{l+m+mi}{40}\PY{p}{,}\PY{l+m+mf}{0.5}\PY{p}{,}\PY{l+m+mf}{0.05}\PY{p}{,}\PY{l+m+mf}{3.3}\PY{p}{)} 

\PY{c+c1}{\PYZsh{} Ausgabe des Ergebnises}
\PY{n+nb}{print}\PY{p}{(}\PY{l+s+s1}{\PYZsq{}}\PY{l+s+s1}{\PYZsh{}}\PY{l+s+s1}{\PYZsq{}} \PY{o}{+} \PY{n}{SCREEN\PYZus{}WIDTH} \PY{o}{*} \PY{l+s+s1}{\PYZsq{}}\PY{l+s+s1}{\PYZhy{}}\PY{l+s+s1}{\PYZsq{}} \PY{o}{+} \PY{l+s+s1}{\PYZsq{}}\PY{l+s+s1}{\PYZsh{}}\PY{l+s+s1}{\PYZsq{}}\PY{p}{)}
\PY{n+nb}{print}\PY{p}{(}\PY{l+s+s1}{\PYZsq{}}\PY{l+s+s1}{|}\PY{l+s+s1}{\PYZsq{}} \PY{o}{+} \PY{n}{centered}\PY{p}{(}\PY{l+s+s1}{\PYZsq{}}\PY{l+s+s1}{Iteration=}\PY{l+s+si}{\PYZpc{}d}\PY{l+s+s1}{\PYZsq{}} \PY{o}{\PYZpc{}}\PY{k}{i} + \PYZsq{}, Sigma=\PYZpc{}f\PYZsq{} \PYZpc{}sigma + \PYZsq{}, Differenz geschätzter Call\PYZhy{}Preis zum gegeben=\PYZpc{}f\PYZsq{} \PYZpc{}diff) + \PYZsq{}| \PYZsq{})
\PY{n+nb}{print}\PY{p}{(}\PY{l+s+s1}{\PYZsq{}}\PY{l+s+s1}{\PYZsh{}}\PY{l+s+s1}{\PYZsq{}} \PY{o}{+} \PY{n}{SCREEN\PYZus{}WIDTH} \PY{o}{*} \PY{l+s+s1}{\PYZsq{}}\PY{l+s+s1}{\PYZhy{}}\PY{l+s+s1}{\PYZsq{}} \PY{o}{+} \PY{l+s+s1}{\PYZsq{}}\PY{l+s+s1}{\PYZsh{}}\PY{l+s+s1}{\PYZsq{}}\PY{p}{)}
\end{Verbatim}
\end{tcolorbox}

    \begin{Verbatim}[commandchars=\\\{\}]
\#-------------------------------------------------------------------------------
----------------------------------------------\#
|                     Iteration=49, Sigma=0.250000, Differenz geschätzter Call-
Preis zum gegeben=-0.004006                    |
\#-------------------------------------------------------------------------------
----------------------------------------------\#
    \end{Verbatim}

    \hypertarget{betrachtung-einer-put-option}{%
\paragraph{Betrachtung einer
Put-Option:}\label{betrachtung-einer-put-option}}

Im folgenden wird eine Put-Option betrachtet, sodass hierbei die
Put-Prämie \texttt{p}=1.501 anstelle einer Call-Prämie als letzte
Input-Variable an die Funktion übergeben wird. Innerhalb der Funktion
werden zunächst Ausgangswerte initialisiert, welche in den sich
anschließenden Rechenoperation überschrieben werden. Sobald das
Abbruchkriterium erreicht wird, werden die berechneten Werte
zurückgebenen.

    \begin{tcolorbox}[breakable, size=fbox, boxrule=1pt, pad at break*=1mm,colback=cellbackground, colframe=cellborder]
\prompt{In}{incolor}{16}{\boxspacing}
\begin{Verbatim}[commandchars=\\\{\}]
\PY{k}{def} \PY{n+nf}{implied\PYZus{}vol\PYZus{}put\PYZus{}min}\PY{p}{(}\PY{n}{s}\PY{p}{,}\PY{n}{b}\PY{p}{,}\PY{n}{t}\PY{p}{,}\PY{n}{roh}\PY{p}{,}\PY{n}{p}\PY{p}{)}\PY{p}{:}
    \PY{n}{implied\PYZus{}vol} \PY{o}{=} \PY{l+m+mf}{1.0} \PY{c+c1}{\PYZsh{} Initialisierung der impliziten Volatilität }
    \PY{n}{min\PYZus{}value} \PY{o}{=} \PY{l+m+mf}{100.0} \PY{c+c1}{\PYZsh{} Initialisierung des Ausgangswertes für das Abbruchkriteriums}
    
    \PY{c+c1}{\PYZsh{} Führe die folgende Berechnung 10000 mal durch}
    \PY{k}{for} \PY{n}{i} \PY{o+ow}{in} \PY{n+nb}{range}\PY{p}{(}\PY{l+m+mi}{1}\PY{p}{,}\PY{l+m+mi}{10000}\PY{p}{)}\PY{p}{:}
        \PY{n}{sigma} \PY{o}{=} \PY{l+m+mf}{0.0001} \PY{o}{*} \PY{p}{(}\PY{n}{i} \PY{o}{+} \PY{l+m+mi}{1}\PY{p}{)} \PY{c+c1}{\PYZsh{} Bestimmung sigma }
        \PY{n}{d1} \PY{o}{=} \PY{p}{(}\PY{n}{math}\PY{o}{.}\PY{n}{log}\PY{p}{(}\PY{n}{s} \PY{o}{/} \PY{n}{b}\PY{p}{)} \PY{o}{+} \PY{p}{(}\PY{n}{roh} \PY{o}{+} \PY{n}{sigma} \PY{o}{*} \PY{n}{sigma} \PY{o}{/} \PY{l+m+mi}{2}\PY{p}{)} \PY{o}{*} \PY{n}{t} \PY{p}{)} \PY{o}{/} \PY{p}{(}\PY{n}{sigma} \PY{o}{*} \PY{n}{math}\PY{o}{.}\PY{n}{sqrt}\PY{p}{(}\PY{n}{t}\PY{p}{)}\PY{p}{)} \PY{c+c1}{\PYZsh{} Berechnung d1}
        \PY{n}{d2} \PY{o}{=} \PY{n}{d1} \PY{o}{\PYZhy{}} \PY{n}{sigma} \PY{o}{*} \PY{n}{math}\PY{o}{.}\PY{n}{sqrt}\PY{p}{(}\PY{n}{t}\PY{p}{)} \PY{c+c1}{\PYZsh{} Berechnung d2}
        \PY{n}{put} \PY{o}{=} \PY{n}{b} \PY{o}{*} \PY{n}{math}\PY{o}{.}\PY{n}{exp}\PY{p}{(}\PY{o}{\PYZhy{}}\PY{n}{roh} \PY{o}{*} \PY{n}{t}\PY{p}{)} \PY{o}{*} \PY{n}{stats}\PY{o}{.}\PY{n}{norm}\PY{o}{.}\PY{n}{cdf}\PY{p}{(}\PY{o}{\PYZhy{}}\PY{n}{d2}\PY{p}{)} \PY{o}{\PYZhy{}} \PY{n}{s} \PY{o}{*} \PY{n}{stats}\PY{o}{.}\PY{n}{norm}\PY{o}{.}\PY{n}{cdf}\PY{p}{(}\PY{o}{\PYZhy{}}\PY{n}{d1}\PY{p}{)} \PY{c+c1}{\PYZsh{} Berechnung der Prämie des Puts}
        \PY{n}{abs\PYZus{}diff} \PY{o}{=} \PY{n+nb}{abs}\PY{p}{(}\PY{n}{put} \PY{o}{\PYZhy{}} \PY{n}{p}\PY{p}{)} \PY{c+c1}{\PYZsh{} Berechnung der absoluten Diffnerenz}
        \PY{c+c1}{\PYZsh{} Abbruchkriterium}
        \PY{k}{if} \PY{n}{abs\PYZus{}diff} \PY{o}{\PYZlt{}} \PY{n}{min\PYZus{}value}\PY{p}{:}
            \PY{n}{min\PYZus{}value} \PY{o}{=} \PY{n}{abs\PYZus{}diff} \PY{c+c1}{\PYZsh{} Bestimmung des neuen Wertes für das Abbruchkriteriums}
            \PY{n}{implied\PYZus{}vol} \PY{o}{=} \PY{n}{sigma} \PY{c+c1}{\PYZsh{} Bestimmung der impliziten Volatilität }
            \PY{n}{k} \PY{o}{=} \PY{n}{i} \PY{c+c1}{\PYZsh{} Bestimmung der Iteration}
        \PY{n}{put\PYZus{}out} \PY{o}{=} \PY{n}{put} \PY{c+c1}{\PYZsh{} Bestimmung des Puts}
    \PY{k}{return}\PY{p}{(}\PY{n}{k}\PY{p}{,} \PY{n}{implied\PYZus{}vol}\PY{p}{,} \PY{n}{put\PYZus{}out}\PY{p}{,} \PY{n}{min\PYZus{}value}\PY{p}{)} \PY{c+c1}{\PYZsh{} Rückgabe der berechneten Wertes}

\PY{c+c1}{\PYZsh{} Ausführung der Funktion}
\PY{n}{k}\PY{p}{,} \PY{n}{implied\PYZus{}vol}\PY{p}{,} \PY{n}{put\PYZus{}out}\PY{p}{,} \PY{n}{min\PYZus{}value} \PY{o}{=} \PY{n}{implied\PYZus{}vol\PYZus{}put\PYZus{}min}\PY{p}{(}\PY{l+m+mi}{40}\PY{p}{,}\PY{l+m+mi}{40}\PY{p}{,}\PY{l+m+mi}{1}\PY{p}{,}\PY{l+m+mf}{0.1}\PY{p}{,}\PY{l+m+mf}{1.501}\PY{p}{)}

\PY{c+c1}{\PYZsh{} Ausgabe des Ergebnises}
\PY{n+nb}{print}\PY{p}{(}\PY{l+s+s1}{\PYZsq{}}\PY{l+s+s1}{\PYZsh{}}\PY{l+s+s1}{\PYZsq{}} \PY{o}{+} \PY{n}{SCREEN\PYZus{}WIDTH} \PY{o}{*} \PY{l+s+s1}{\PYZsq{}}\PY{l+s+s1}{\PYZhy{}}\PY{l+s+s1}{\PYZsq{}} \PY{o}{+} \PY{l+s+s1}{\PYZsq{}}\PY{l+s+s1}{\PYZsh{}}\PY{l+s+s1}{\PYZsq{}}\PY{p}{)}
\PY{n+nb}{print}\PY{p}{(}\PY{l+s+s1}{\PYZsq{}}\PY{l+s+s1}{|}\PY{l+s+s1}{\PYZsq{}} \PY{o}{+} \PY{n}{centered}\PY{p}{(}\PY{l+s+s1}{\PYZsq{}}\PY{l+s+s1}{Iteration=}\PY{l+s+si}{\PYZpc{}d}\PY{l+s+s1}{\PYZsq{}} \PY{o}{\PYZpc{}}\PY{k}{k} + \PYZsq{}, Implizite Volatilität=\PYZpc{}f\PYZsq{} \PYZpc{}implied\PYZus{}vol + \PYZsq{}, Payoff Put=\PYZpc{}f\PYZsq{} \PYZpc{}put\PYZus{}out + \PYZsq{}, Absolute Differenz=\PYZpc{}f\PYZsq{} \PYZpc{}min\PYZus{}value) + \PYZsq{}| \PYZsq{})
\PY{n+nb}{print}\PY{p}{(}\PY{l+s+s1}{\PYZsq{}}\PY{l+s+s1}{\PYZsh{}}\PY{l+s+s1}{\PYZsq{}} \PY{o}{+} \PY{n}{SCREEN\PYZus{}WIDTH} \PY{o}{*} \PY{l+s+s1}{\PYZsq{}}\PY{l+s+s1}{\PYZhy{}}\PY{l+s+s1}{\PYZsq{}} \PY{o}{+} \PY{l+s+s1}{\PYZsq{}}\PY{l+s+s1}{\PYZsh{}}\PY{l+s+s1}{\PYZsq{}}\PY{p}{)}
\end{Verbatim}
\end{tcolorbox}

    \begin{Verbatim}[commandchars=\\\{\}]
\#-------------------------------------------------------------------------------
----------------------------------------------\#
|              Iteration=1999, Implizite Volatilität=0.200000, Payoff
Put=12.751880, Absolute Differenz=0.000367              |
\#-------------------------------------------------------------------------------
----------------------------------------------\#
    \end{Verbatim}

    \hypertarget{aufagbe-5---volatility-smile-and-skewness}{%
\subsection{Aufagbe 5 - Volatility smile and
skewness}\label{aufagbe-5---volatility-smile-and-skewness}}

Zwischen den Aktienkursen und der Volatilität besteht eine negative
Korrelation. Wenn sich die Preise nach unten (nach oben) bewegen, neigen
Volatilitäten dazu, sich nach oben (nach unten) zu bewegen. Die
implizite Volatilität ist relativ gering für at-the-money Optionen, wird
aber größer für Optionen die in-the-money oder out-of-the-money sind.
Das Volatility-Smile zeigt diesen Zusammenhang auf und ist in der
folgenden Grafik veranschaulicht.

Des Weiteren lässt sich mitunter auch das sogenannte Volatility-Skew
beobachten, welches durch das Verhältnis von Angebot und Nachfrage
bestimmter Optionen beeinflusst wird, und darüber Aufschluss gibt, ob
Fondsmanager vorwiegend Calls oder Puts zeichnen.

    \begin{tcolorbox}[breakable, size=fbox, boxrule=1pt, pad at break*=1mm,colback=cellbackground, colframe=cellborder]
\prompt{In}{incolor}{17}{\boxspacing}
\begin{Verbatim}[commandchars=\\\{\}]
\PY{c+c1}{\PYZsh{} Dictonary in welchem die Ticker\PYZhy{}Kürzel hinterlegt sind}
\PY{n}{AG\PYZus{}Dict} \PY{o}{=} \PY{p}{\PYZob{}}\PY{l+s+s2}{\PYZdq{}}\PY{l+s+s2}{Apple}\PY{l+s+s2}{\PYZdq{}}\PY{p}{:} \PY{l+s+s1}{\PYZsq{}}\PY{l+s+s1}{AAPL}\PY{l+s+s1}{\PYZsq{}}\PY{p}{,}
           \PY{l+s+s2}{\PYZdq{}}\PY{l+s+s2}{Amazon}\PY{l+s+s2}{\PYZdq{}}\PY{p}{:} \PY{l+s+s1}{\PYZsq{}}\PY{l+s+s1}{AMZN}\PY{l+s+s1}{\PYZsq{}}\PY{p}{,}
           \PY{l+s+s2}{\PYZdq{}}\PY{l+s+s2}{VW}\PY{l+s+s2}{\PYZdq{}}\PY{p}{:} \PY{l+s+s1}{\PYZsq{}}\PY{l+s+s1}{VOW.DE}\PY{l+s+s1}{\PYZsq{}}\PY{p}{,}
           \PY{l+s+s2}{\PYZdq{}}\PY{l+s+s2}{BASF}\PY{l+s+s2}{\PYZdq{}}\PY{p}{:} \PY{l+s+s1}{\PYZsq{}}\PY{l+s+s1}{BAS.DE}\PY{l+s+s1}{\PYZsq{}}\PY{p}{,}
           \PY{l+s+s2}{\PYZdq{}}\PY{l+s+s2}{Microsoft}\PY{l+s+s2}{\PYZdq{}}\PY{p}{:} \PY{l+s+s1}{\PYZsq{}}\PY{l+s+s1}{MSFT}\PY{l+s+s1}{\PYZsq{}}\PY{p}{,}
           \PY{l+s+s2}{\PYZdq{}}\PY{l+s+s2}{Tesla}\PY{l+s+s2}{\PYZdq{}}\PY{p}{:} \PY{l+s+s1}{\PYZsq{}}\PY{l+s+s1}{TSLA}\PY{l+s+s1}{\PYZsq{}}\PY{p}{\PYZcb{}}

\PY{c+c1}{\PYZsh{}\PYZhy{}\PYZhy{}\PYZhy{}\PYZhy{}\PYZhy{}\PYZhy{}\PYZhy{}\PYZhy{}\PYZhy{}\PYZhy{}\PYZhy{}\PYZhy{}\PYZhy{}\PYZhy{}\PYZhy{}\PYZhy{}\PYZhy{}\PYZhy{}\PYZhy{}\PYZhy{}\PYZhy{}\PYZhy{}\PYZhy{}\PYZhy{}\PYZhy{}\PYZhy{}\PYZhy{}\PYZhy{}\PYZhy{}\PYZhy{}\PYZhy{}\PYZhy{}\PYZhy{}\PYZhy{}\PYZhy{}\PYZhy{}\PYZhy{}\PYZhy{}\PYZhy{}\PYZhy{}\PYZhy{}\PYZhy{}\PYZhy{}\PYZhy{}\PYZhy{}\PYZhy{}\PYZhy{}\PYZhy{}\PYZhy{}\PYZhy{}\PYZhy{}\PYZhy{}\PYZhy{}\PYZhy{}\PYZhy{}\PYZhy{}\PYZhy{}\PYZhy{}\PYZhy{}\PYZhy{}}
\PY{c+c1}{\PYZsh{} Abfrage zur Auswahl des Unternehmens}

\PY{n+nb}{print}\PY{p}{(}\PY{l+s+s1}{\PYZsq{}}\PY{l+s+s1}{\PYZsh{}}\PY{l+s+s1}{\PYZsq{}} \PY{o}{+} \PY{n}{SCREEN\PYZus{}WIDTH} \PY{o}{*} \PY{l+s+s1}{\PYZsq{}}\PY{l+s+s1}{\PYZhy{}}\PY{l+s+s1}{\PYZsq{}} \PY{o}{+} \PY{l+s+s1}{\PYZsq{}}\PY{l+s+s1}{\PYZsh{}}\PY{l+s+s1}{\PYZsq{}}\PY{p}{)}
\PY{n+nb}{print}\PY{p}{(}\PY{l+s+s1}{\PYZsq{}}\PY{l+s+s1}{|}\PY{l+s+s1}{\PYZsq{}} \PY{o}{+} \PY{n}{centered}\PY{p}{(}\PY{l+s+s1}{\PYZsq{}}\PY{l+s+s1}{Es sind die folgenden Unternehmen betrachtbar:}\PY{l+s+s1}{\PYZsq{}}\PY{p}{)} \PY{o}{+} \PY{l+s+s1}{\PYZsq{}}\PY{l+s+s1}{| }\PY{l+s+s1}{\PYZsq{}}\PY{p}{)}
\PY{n+nb}{print}\PY{p}{(}\PY{l+s+s1}{\PYZsq{}}\PY{l+s+s1}{\PYZsh{}}\PY{l+s+s1}{\PYZsq{}} \PY{o}{+} \PY{n}{SCREEN\PYZus{}WIDTH} \PY{o}{*} \PY{l+s+s1}{\PYZsq{}}\PY{l+s+s1}{\PYZhy{}}\PY{l+s+s1}{\PYZsq{}} \PY{o}{+} \PY{l+s+s1}{\PYZsq{}}\PY{l+s+s1}{\PYZsh{}}\PY{l+s+s1}{\PYZsq{}}\PY{p}{)}
\PY{n}{counter} \PY{o}{=} \PY{l+m+mi}{0}
\PY{k}{for} \PY{n}{key}\PY{p}{,} \PY{n}{value} \PY{o+ow}{in} \PY{n}{AG\PYZus{}Dict}\PY{o}{.}\PY{n}{items}\PY{p}{(}\PY{p}{)} \PY{p}{:}
    \PY{n+nb}{print}\PY{p}{(}\PY{n+nb}{str}\PY{p}{(}\PY{n}{counter}\PY{p}{)} \PY{o}{+} \PY{l+s+s1}{\PYZsq{}}\PY{l+s+s1}{) }\PY{l+s+s1}{\PYZsq{}} \PY{o}{+} \PY{n+nb}{str}\PY{p}{(}\PY{n}{key}\PY{p}{)}\PY{p}{)}
    \PY{n}{counter} \PY{o}{+}\PY{o}{=} \PY{l+m+mi}{1}
\PY{n+nb}{print}\PY{p}{(}\PY{l+s+s1}{\PYZsq{}}\PY{l+s+s1}{\PYZsh{}}\PY{l+s+s1}{\PYZsq{}} \PY{o}{+} \PY{n}{SCREEN\PYZus{}WIDTH} \PY{o}{*} \PY{l+s+s1}{\PYZsq{}}\PY{l+s+s1}{\PYZhy{}}\PY{l+s+s1}{\PYZsq{}} \PY{o}{+} \PY{l+s+s1}{\PYZsq{}}\PY{l+s+s1}{\PYZsh{}}\PY{l+s+s1}{\PYZsq{}}\PY{p}{)}
\PY{n}{My\PYZus{}AG} \PY{o}{=} \PY{n+nb}{input}\PY{p}{(}\PY{l+s+s1}{\PYZsq{}}\PY{l+s+s1}{|}\PY{l+s+s1}{\PYZsq{}} \PY{o}{+} \PY{n}{centered}\PY{p}{(}\PY{l+s+s1}{\PYZsq{}}\PY{l+s+s1}{Bitte geben Sie namentlich ein Unternehmen aus der Liste ein:}\PY{l+s+s1}{\PYZsq{}}\PY{p}{)} \PY{o}{+} \PY{l+s+s1}{\PYZsq{}}\PY{l+s+s1}{| }\PY{l+s+s1}{\PYZsq{}}\PY{p}{)}

\PY{c+c1}{\PYZsh{}\PYZhy{}\PYZhy{}\PYZhy{}\PYZhy{}\PYZhy{}\PYZhy{}\PYZhy{}\PYZhy{}\PYZhy{}\PYZhy{}\PYZhy{}\PYZhy{}\PYZhy{}\PYZhy{}\PYZhy{}\PYZhy{}\PYZhy{}\PYZhy{}\PYZhy{}\PYZhy{}\PYZhy{}\PYZhy{}\PYZhy{}\PYZhy{}\PYZhy{}\PYZhy{}\PYZhy{}\PYZhy{}\PYZhy{}\PYZhy{}\PYZhy{}\PYZhy{}\PYZhy{}\PYZhy{}\PYZhy{}\PYZhy{}\PYZhy{}\PYZhy{}\PYZhy{}\PYZhy{}\PYZhy{}\PYZhy{}\PYZhy{}\PYZhy{}\PYZhy{}\PYZhy{}\PYZhy{}\PYZhy{}\PYZhy{}\PYZhy{}\PYZhy{}\PYZhy{}\PYZhy{}\PYZhy{}\PYZhy{}\PYZhy{}\PYZhy{}\PYZhy{}\PYZhy{}\PYZhy{}}
\PY{c+c1}{\PYZsh{} Ausgabe verfügbare Fälligkeiten}

\PY{n}{AG} \PY{o}{=} \PY{n}{yf}\PY{o}{.}\PY{n}{Ticker}\PY{p}{(}\PY{n}{AG\PYZus{}Dict}\PY{p}{[}\PY{n}{My\PYZus{}AG}\PY{p}{]}\PY{p}{)}
\PY{n+nb}{print}\PY{p}{(}\PY{l+s+s1}{\PYZsq{}}\PY{l+s+s1}{\PYZsh{}}\PY{l+s+s1}{\PYZsq{}} \PY{o}{+} \PY{n}{SCREEN\PYZus{}WIDTH} \PY{o}{*} \PY{l+s+s1}{\PYZsq{}}\PY{l+s+s1}{\PYZhy{}}\PY{l+s+s1}{\PYZsq{}} \PY{o}{+} \PY{l+s+s1}{\PYZsq{}}\PY{l+s+s1}{\PYZsh{}}\PY{l+s+s1}{\PYZsq{}}\PY{p}{)}
\PY{n+nb}{print}\PY{p}{(}\PY{l+s+s1}{\PYZsq{}}\PY{l+s+s1}{|}\PY{l+s+s1}{\PYZsq{}} \PY{o}{+} \PY{n}{centered}\PY{p}{(}\PY{l+s+s1}{\PYZsq{}}\PY{l+s+s1}{Für }\PY{l+s+s1}{\PYZsq{}} \PY{o}{+} \PY{n+nb}{str}\PY{p}{(}\PY{n}{My\PYZus{}AG}\PY{p}{)} \PY{o}{+} \PY{l+s+s1}{\PYZsq{}}\PY{l+s+s1}{ sind Optionen mit den folgenden Fälligkeiten verfügbar:}\PY{l+s+s1}{\PYZsq{}}\PY{p}{)} \PY{o}{+} \PY{l+s+s1}{\PYZsq{}}\PY{l+s+s1}{| }\PY{l+s+s1}{\PYZsq{}}\PY{p}{)}
\PY{n+nb}{print}\PY{p}{(}\PY{l+s+s1}{\PYZsq{}}\PY{l+s+s1}{\PYZsh{}}\PY{l+s+s1}{\PYZsq{}} \PY{o}{+} \PY{n}{SCREEN\PYZus{}WIDTH} \PY{o}{*} \PY{l+s+s1}{\PYZsq{}}\PY{l+s+s1}{\PYZhy{}}\PY{l+s+s1}{\PYZsq{}} \PY{o}{+} \PY{l+s+s1}{\PYZsq{}}\PY{l+s+s1}{\PYZsh{}}\PY{l+s+s1}{\PYZsq{}}\PY{p}{)}
\PY{n}{maturitys} \PY{o}{=} \PY{n}{AG}\PY{o}{.}\PY{n}{options}
\PY{n}{counter} \PY{o}{=} \PY{l+m+mi}{0}
\PY{k}{for} \PY{n}{maturity} \PY{o+ow}{in} \PY{n}{maturitys}\PY{p}{:}
    \PY{n+nb}{print}\PY{p}{(}\PY{n+nb}{str}\PY{p}{(}\PY{n}{counter}\PY{p}{)} \PY{o}{+} \PY{l+s+s1}{\PYZsq{}}\PY{l+s+s1}{) }\PY{l+s+s1}{\PYZsq{}} \PY{o}{+} \PY{n+nb}{str}\PY{p}{(}\PY{n}{maturity}\PY{p}{)}\PY{p}{)}
    \PY{n}{counter} \PY{o}{+}\PY{o}{=} \PY{l+m+mi}{1}
\PY{n+nb}{print}\PY{p}{(}\PY{l+s+s1}{\PYZsq{}}\PY{l+s+s1}{\PYZsh{}}\PY{l+s+s1}{\PYZsq{}} \PY{o}{+} \PY{n}{SCREEN\PYZus{}WIDTH} \PY{o}{*} \PY{l+s+s1}{\PYZsq{}}\PY{l+s+s1}{\PYZhy{}}\PY{l+s+s1}{\PYZsq{}} \PY{o}{+} \PY{l+s+s1}{\PYZsq{}}\PY{l+s+s1}{\PYZsh{}}\PY{l+s+s1}{\PYZsq{}}\PY{p}{)}

\PY{c+c1}{\PYZsh{}\PYZhy{}\PYZhy{}\PYZhy{}\PYZhy{}\PYZhy{}\PYZhy{}\PYZhy{}\PYZhy{}\PYZhy{}\PYZhy{}\PYZhy{}\PYZhy{}\PYZhy{}\PYZhy{}\PYZhy{}\PYZhy{}\PYZhy{}\PYZhy{}\PYZhy{}\PYZhy{}\PYZhy{}\PYZhy{}\PYZhy{}\PYZhy{}\PYZhy{}\PYZhy{}\PYZhy{}\PYZhy{}\PYZhy{}\PYZhy{}\PYZhy{}\PYZhy{}\PYZhy{}\PYZhy{}\PYZhy{}\PYZhy{}\PYZhy{}\PYZhy{}\PYZhy{}\PYZhy{}\PYZhy{}\PYZhy{}\PYZhy{}\PYZhy{}\PYZhy{}\PYZhy{}\PYZhy{}\PYZhy{}\PYZhy{}\PYZhy{}\PYZhy{}\PYZhy{}\PYZhy{}\PYZhy{}\PYZhy{}\PYZhy{}\PYZhy{}\PYZhy{}\PYZhy{}\PYZhy{}}
\PY{c+c1}{\PYZsh{} Wahl eines Fälligkeitstermins}

\PY{n}{My\PYZus{}Date} \PY{o}{=} \PY{n+nb}{int}\PY{p}{(}\PY{n+nb}{input}\PY{p}{(}\PY{l+s+s1}{\PYZsq{}}\PY{l+s+s1}{|}\PY{l+s+s1}{\PYZsq{}} \PY{o}{+} \PY{n}{centered}\PY{p}{(}\PY{l+s+s1}{\PYZsq{}}\PY{l+s+s1}{Bitte geben Sie die Nummer der gewünschten Fälligkeit ein:}\PY{l+s+s1}{\PYZsq{}}\PY{p}{)} \PY{o}{+} \PY{l+s+s1}{\PYZsq{}}\PY{l+s+s1}{| }\PY{l+s+s1}{\PYZsq{}}\PY{p}{)}\PY{p}{)}
\PY{n+nb}{print}\PY{p}{(}\PY{l+s+s1}{\PYZsq{}}\PY{l+s+s1}{\PYZsh{}}\PY{l+s+s1}{\PYZsq{}} \PY{o}{+} \PY{n}{SCREEN\PYZus{}WIDTH} \PY{o}{*} \PY{l+s+s1}{\PYZsq{}}\PY{l+s+s1}{\PYZhy{}}\PY{l+s+s1}{\PYZsq{}} \PY{o}{+} \PY{l+s+s1}{\PYZsq{}}\PY{l+s+s1}{\PYZsh{}}\PY{l+s+s1}{\PYZsq{}}\PY{p}{)}
\PY{n}{AG\PYZus{}options} \PY{o}{=} \PY{n}{AG}\PY{o}{.}\PY{n}{option\PYZus{}chain}\PY{p}{(}\PY{n}{AG}\PY{o}{.}\PY{n}{options}\PY{p}{[}\PY{n}{My\PYZus{}Date}\PY{p}{]}\PY{p}{)}
\PY{n}{data} \PY{o}{=} \PY{n}{AG\PYZus{}options}\PY{o}{.}\PY{n}{calls}
\PY{n}{data} \PY{o}{=} \PY{n}{AG\PYZus{}options}\PY{o}{.}\PY{n}{puts}

\PY{c+c1}{\PYZsh{}\PYZhy{}\PYZhy{}\PYZhy{}\PYZhy{}\PYZhy{}\PYZhy{}\PYZhy{}\PYZhy{}\PYZhy{}\PYZhy{}\PYZhy{}\PYZhy{}\PYZhy{}\PYZhy{}\PYZhy{}\PYZhy{}\PYZhy{}\PYZhy{}\PYZhy{}\PYZhy{}\PYZhy{}\PYZhy{}\PYZhy{}\PYZhy{}\PYZhy{}\PYZhy{}\PYZhy{}\PYZhy{}\PYZhy{}\PYZhy{}\PYZhy{}\PYZhy{}\PYZhy{}\PYZhy{}\PYZhy{}\PYZhy{}\PYZhy{}\PYZhy{}\PYZhy{}\PYZhy{}\PYZhy{}\PYZhy{}\PYZhy{}\PYZhy{}\PYZhy{}\PYZhy{}\PYZhy{}\PYZhy{}\PYZhy{}\PYZhy{}\PYZhy{}\PYZhy{}\PYZhy{}\PYZhy{}\PYZhy{}\PYZhy{}\PYZhy{}\PYZhy{}\PYZhy{}\PYZhy{}}
\PY{c+c1}{\PYZsh{} Vereinfachter DataFrame mit Informationen zu den Kontrakten}

\PY{n}{stock\PYZus{}overview} \PY{o}{=} \PY{n}{pd}\PY{o}{.}\PY{n}{DataFrame}\PY{p}{(}\PY{p}{\PYZob{}}\PY{l+s+s1}{\PYZsq{}}\PY{l+s+s1}{Bezeichnung}\PY{l+s+s1}{\PYZsq{}}\PY{p}{:}\PY{n}{data}\PY{p}{[}\PY{l+s+s1}{\PYZsq{}}\PY{l+s+s1}{contractSymbol}\PY{l+s+s1}{\PYZsq{}}\PY{p}{]}\PY{p}{,} \PY{l+s+s1}{\PYZsq{}}\PY{l+s+s1}{Strike}\PY{l+s+s1}{\PYZsq{}}\PY{p}{:}\PY{n}{data}\PY{p}{[}\PY{l+s+s1}{\PYZsq{}}\PY{l+s+s1}{strike}\PY{l+s+s1}{\PYZsq{}}\PY{p}{]}\PY{p}{,} \PY{l+s+s1}{\PYZsq{}}\PY{l+s+s1}{Bid}\PY{l+s+s1}{\PYZsq{}}\PY{p}{:}\PY{n}{data}\PY{p}{[}\PY{l+s+s1}{\PYZsq{}}\PY{l+s+s1}{bid}\PY{l+s+s1}{\PYZsq{}}\PY{p}{]}\PY{p}{,} \PY{l+s+s1}{\PYZsq{}}\PY{l+s+s1}{Ask}\PY{l+s+s1}{\PYZsq{}}\PY{p}{:}\PY{n}{data}\PY{p}{[}\PY{l+s+s1}{\PYZsq{}}\PY{l+s+s1}{ask}\PY{l+s+s1}{\PYZsq{}}\PY{p}{]}\PY{p}{,} \PY{l+s+s1}{\PYZsq{}}\PY{l+s+s1}{Volatilität}\PY{l+s+s1}{\PYZsq{}}\PY{p}{:}\PY{n}{data}\PY{p}{[}\PY{l+s+s1}{\PYZsq{}}\PY{l+s+s1}{impliedVolatility}\PY{l+s+s1}{\PYZsq{}}\PY{p}{]}\PY{p}{,} \PY{l+s+s1}{\PYZsq{}}\PY{l+s+s1}{InTheMoney}\PY{l+s+s1}{\PYZsq{}}\PY{p}{:}\PY{n}{data}\PY{p}{[}\PY{l+s+s1}{\PYZsq{}}\PY{l+s+s1}{inTheMoney}\PY{l+s+s1}{\PYZsq{}}\PY{p}{]}\PY{p}{\PYZcb{}}\PY{p}{)}
\PY{n+nb}{print}\PY{p}{(}\PY{l+s+s1}{\PYZsq{}}\PY{l+s+s1}{|}\PY{l+s+s1}{\PYZsq{}} \PY{o}{+} \PY{n}{centered}\PY{p}{(}\PY{l+s+s1}{\PYZsq{}}\PY{l+s+s1}{Ausgewählt wurde }\PY{l+s+s1}{\PYZsq{}} \PY{o}{+} \PY{n+nb}{str}\PY{p}{(}\PY{n}{My\PYZus{}AG}\PY{p}{)} \PY{o}{+} \PY{l+s+s1}{\PYZsq{}}\PY{l+s+s1}{ zur Fälligkeit }\PY{l+s+s1}{\PYZsq{}} \PY{o}{+} \PY{n+nb}{str}\PY{p}{(}\PY{n}{AG}\PY{o}{.}\PY{n}{options}\PY{p}{[}\PY{n}{My\PYZus{}Date}\PY{p}{]}\PY{p}{)} \PY{o}{+} \PY{l+s+s1}{\PYZsq{}}\PY{l+s+s1}{.}\PY{l+s+s1}{\PYZsq{}}\PY{p}{)} \PY{o}{+} \PY{l+s+s1}{\PYZsq{}}\PY{l+s+s1}{| }\PY{l+s+s1}{\PYZsq{}}\PY{p}{)}
\PY{n+nb}{print}\PY{p}{(}\PY{l+s+s1}{\PYZsq{}}\PY{l+s+s1}{\PYZsh{}}\PY{l+s+s1}{\PYZsq{}} \PY{o}{+} \PY{n}{SCREEN\PYZus{}WIDTH} \PY{o}{*} \PY{l+s+s1}{\PYZsq{}}\PY{l+s+s1}{\PYZhy{}}\PY{l+s+s1}{\PYZsq{}} \PY{o}{+} \PY{l+s+s1}{\PYZsq{}}\PY{l+s+s1}{\PYZsh{}}\PY{l+s+s1}{\PYZsq{}}\PY{p}{)}
\PY{n+nb}{print}\PY{p}{(}\PY{n}{stock\PYZus{}overview}\PY{p}{)}
\PY{n+nb}{print}\PY{p}{(}\PY{l+s+s1}{\PYZsq{}}\PY{l+s+s1}{\PYZsh{}}\PY{l+s+s1}{\PYZsq{}} \PY{o}{+} \PY{n}{SCREEN\PYZus{}WIDTH} \PY{o}{*} \PY{l+s+s1}{\PYZsq{}}\PY{l+s+s1}{\PYZhy{}}\PY{l+s+s1}{\PYZsq{}} \PY{o}{+} \PY{l+s+s1}{\PYZsq{}}\PY{l+s+s1}{\PYZsh{}}\PY{l+s+s1}{\PYZsq{}}\PY{p}{)}

\PY{c+c1}{\PYZsh{}\PYZhy{}\PYZhy{}\PYZhy{}\PYZhy{}\PYZhy{}\PYZhy{}\PYZhy{}\PYZhy{}\PYZhy{}\PYZhy{}\PYZhy{}\PYZhy{}\PYZhy{}\PYZhy{}\PYZhy{}\PYZhy{}\PYZhy{}\PYZhy{}\PYZhy{}\PYZhy{}\PYZhy{}\PYZhy{}\PYZhy{}\PYZhy{}\PYZhy{}\PYZhy{}\PYZhy{}\PYZhy{}\PYZhy{}\PYZhy{}\PYZhy{}\PYZhy{}\PYZhy{}\PYZhy{}\PYZhy{}\PYZhy{}\PYZhy{}\PYZhy{}\PYZhy{}\PYZhy{}\PYZhy{}\PYZhy{}\PYZhy{}\PYZhy{}\PYZhy{}\PYZhy{}\PYZhy{}\PYZhy{}\PYZhy{}\PYZhy{}\PYZhy{}\PYZhy{}\PYZhy{}\PYZhy{}\PYZhy{}\PYZhy{}\PYZhy{}\PYZhy{}\PYZhy{}\PYZhy{}}
\PY{c+c1}{\PYZsh{} Generierung der Grafik für das Volatility Smile }

\PY{n}{x} \PY{o}{=} \PY{n}{data}\PY{p}{[}\PY{l+s+s1}{\PYZsq{}}\PY{l+s+s1}{strike}\PY{l+s+s1}{\PYZsq{}}\PY{p}{]} \PY{c+c1}{\PYZsh{} Speichere die eingelesenen Daten aus der Spalte \PYZsq{}Strike\PYZsq{} als die Variable x}
\PY{n}{y} \PY{o}{=} \PY{n+nb}{list}\PY{p}{(}\PY{n}{data}\PY{p}{[}\PY{l+s+s1}{\PYZsq{}}\PY{l+s+s1}{impliedVolatility}\PY{l+s+s1}{\PYZsq{}}\PY{p}{]}\PY{p}{)} \PY{c+c1}{\PYZsh{} Überführe die eingelesenen Daten aus der Spalte \PYZsq{}Implied Volatility\PYZsq{} in eine Liste}
\PY{n}{plt}\PY{o}{.}\PY{n}{title}\PY{p}{(}\PY{l+s+s1}{\PYZsq{}}\PY{l+s+s1}{Volatility smile}\PY{l+s+s1}{\PYZsq{}}\PY{p}{)} \PY{c+c1}{\PYZsh{} Titel der Grafik}
\PY{n}{plt}\PY{o}{.}\PY{n}{ylabel}\PY{p}{(}\PY{l+s+s1}{\PYZsq{}}\PY{l+s+s1}{Volatilität}\PY{l+s+s1}{\PYZsq{}}\PY{p}{)} \PY{c+c1}{\PYZsh{} Beschriftung Y\PYZhy{}Achse}
\PY{n}{plt}\PY{o}{.}\PY{n}{xlabel}\PY{p}{(}\PY{l+s+s1}{\PYZsq{}}\PY{l+s+s1}{Preis des Strikes}\PY{l+s+s1}{\PYZsq{}}\PY{p}{)} \PY{c+c1}{\PYZsh{} Beschriftung X\PYZhy{}Achse}
\PY{n}{plt}\PY{o}{.}\PY{n}{plot}\PY{p}{(}\PY{n}{x}\PY{p}{,}\PY{n}{y}\PY{p}{,}\PY{l+s+s1}{\PYZsq{}}\PY{l+s+s1}{o}\PY{l+s+s1}{\PYZsq{}}\PY{p}{)} \PY{c+c1}{\PYZsh{} Plotten der Datenpunkte}
\PY{n}{plt}\PY{o}{.}\PY{n}{grid}\PY{p}{(}\PY{p}{)} \PY{c+c1}{\PYZsh{} Gitternetz}
\PY{n}{plt}\PY{o}{.}\PY{n}{show}\PY{p}{(}\PY{p}{)} \PY{c+c1}{\PYZsh{} Funktion zum anzeigen der Grafik}

\PY{c+c1}{\PYZsh{}\PYZhy{}\PYZhy{}\PYZhy{}\PYZhy{}\PYZhy{}\PYZhy{}\PYZhy{}\PYZhy{}\PYZhy{}\PYZhy{}\PYZhy{}\PYZhy{}\PYZhy{}\PYZhy{}\PYZhy{}\PYZhy{}\PYZhy{}\PYZhy{}\PYZhy{}\PYZhy{}\PYZhy{}\PYZhy{}\PYZhy{}\PYZhy{}\PYZhy{}\PYZhy{}\PYZhy{}\PYZhy{}\PYZhy{}\PYZhy{}\PYZhy{}\PYZhy{}\PYZhy{}\PYZhy{}\PYZhy{}\PYZhy{}\PYZhy{}\PYZhy{}\PYZhy{}\PYZhy{}\PYZhy{}\PYZhy{}\PYZhy{}\PYZhy{}\PYZhy{}\PYZhy{}\PYZhy{}\PYZhy{}\PYZhy{}\PYZhy{}\PYZhy{}\PYZhy{}\PYZhy{}\PYZhy{}\PYZhy{}\PYZhy{}\PYZhy{}\PYZhy{}\PYZhy{}\PYZhy{}\PYZhy{}\PYZhy{}\PYZhy{}\PYZhy{}\PYZhy{}\PYZhy{}\PYZhy{}\PYZhy{}\PYZhy{}\PYZhy{}\PYZhy{}\PYZhy{}\PYZhy{}\PYZhy{}\PYZhy{}\PYZhy{}\PYZhy{}\PYZhy{}\PYZhy{}\PYZhy{}\PYZhy{}\PYZhy{}\PYZhy{}\PYZhy{}\PYZhy{}\PYZhy{}\PYZhy{}\PYZhy{}\PYZhy{}\PYZhy{}\PYZhy{}\PYZhy{}\PYZhy{}\PYZhy{}\PYZhy{}\PYZhy{}\PYZhy{}\PYZhy{}\PYZhy{}\PYZhy{}\PYZhy{}\PYZhy{}\PYZhy{}\PYZhy{}\PYZhy{}\PYZhy{}\PYZhy{}\PYZhy{}\PYZhy{}\PYZhy{}\PYZhy{}\PYZhy{}\PYZhy{}\PYZhy{}\PYZhy{}\PYZhy{}\PYZhy{}\PYZhy{}\PYZhy{}\PYZhy{}}
\PY{c+c1}{\PYZsh{}\PYZsh{}\PYZsh{}\PYZsh{}\PYZsh{}\PYZsh{}\PYZsh{}\PYZsh{}\PYZsh{}\PYZsh{}\PYZsh{}\PYZsh{}\PYZsh{}\PYZsh{}\PYZsh{}\PYZsh{}\PYZsh{}\PYZsh{}\PYZsh{}\PYZsh{}\PYZsh{}\PYZsh{}\PYZsh{}\PYZsh{}\PYZsh{}\PYZsh{}\PYZsh{}\PYZsh{}\PYZsh{}\PYZsh{}\PYZsh{}\PYZsh{}\PYZsh{}\PYZsh{}\PYZsh{}\PYZsh{}\PYZsh{}\PYZsh{}\PYZsh{}\PYZsh{}\PYZsh{}\PYZsh{}\PYZsh{}\PYZsh{}\PYZsh{}\PYZsh{}\PYZsh{}\PYZsh{}\PYZsh{}\PYZsh{}\PYZsh{}\PYZsh{}\PYZsh{}\PYZsh{}\PYZsh{}\PYZsh{}\PYZsh{}\PYZsh{}\PYZsh{}\PYZsh{}\PYZsh{}\PYZsh{}\PYZsh{}\PYZsh{}\PYZsh{}\PYZsh{}\PYZsh{}\PYZsh{}\PYZsh{}\PYZsh{}\PYZsh{}\PYZsh{}\PYZsh{}\PYZsh{}\PYZsh{}\PYZsh{}\PYZsh{}\PYZsh{}\PYZsh{}\PYZsh{}\PYZsh{}\PYZsh{}\PYZsh{}\PYZsh{}\PYZsh{}\PYZsh{}\PYZsh{}\PYZsh{}\PYZsh{}\PYZsh{}\PYZsh{}\PYZsh{}\PYZsh{}\PYZsh{}\PYZsh{}\PYZsh{}\PYZsh{}\PYZsh{}\PYZsh{}\PYZsh{}\PYZsh{}\PYZsh{}\PYZsh{}\PYZsh{}\PYZsh{}\PYZsh{}\PYZsh{}\PYZsh{}\PYZsh{}\PYZsh{}\PYZsh{}\PYZsh{}\PYZsh{}\PYZsh{}\PYZsh{}\PYZsh{}\PYZsh{}\PYZsh{}\PYZsh{}\PYZsh{}\PYZsh{}}
\PY{c+c1}{\PYZsh{}\PYZhy{}\PYZhy{}\PYZhy{}\PYZhy{}\PYZhy{}\PYZhy{}\PYZhy{}\PYZhy{}\PYZhy{}\PYZhy{}\PYZhy{}\PYZhy{}\PYZhy{}\PYZhy{}\PYZhy{}\PYZhy{}\PYZhy{}\PYZhy{}\PYZhy{}\PYZhy{}\PYZhy{}\PYZhy{}\PYZhy{}\PYZhy{}\PYZhy{}\PYZhy{}\PYZhy{}\PYZhy{}\PYZhy{}\PYZhy{}\PYZhy{}\PYZhy{}\PYZhy{}\PYZhy{}\PYZhy{}\PYZhy{}\PYZhy{}\PYZhy{}\PYZhy{}\PYZhy{}\PYZhy{}\PYZhy{}\PYZhy{}\PYZhy{}\PYZhy{}\PYZhy{}\PYZhy{}\PYZhy{}\PYZhy{}\PYZhy{}\PYZhy{}\PYZhy{}\PYZhy{}\PYZhy{}\PYZhy{}\PYZhy{}\PYZhy{}\PYZhy{}\PYZhy{}\PYZhy{}\PYZhy{}\PYZhy{}\PYZhy{}\PYZhy{}\PYZhy{}\PYZhy{}\PYZhy{}\PYZhy{}\PYZhy{}\PYZhy{}\PYZhy{}\PYZhy{}\PYZhy{}\PYZhy{}\PYZhy{}\PYZhy{}\PYZhy{}\PYZhy{}\PYZhy{}\PYZhy{}\PYZhy{}\PYZhy{}\PYZhy{}\PYZhy{}\PYZhy{}\PYZhy{}\PYZhy{}\PYZhy{}\PYZhy{}\PYZhy{}\PYZhy{}\PYZhy{}\PYZhy{}\PYZhy{}\PYZhy{}\PYZhy{}\PYZhy{}\PYZhy{}\PYZhy{}\PYZhy{}\PYZhy{}\PYZhy{}\PYZhy{}\PYZhy{}\PYZhy{}\PYZhy{}\PYZhy{}\PYZhy{}\PYZhy{}\PYZhy{}\PYZhy{}\PYZhy{}\PYZhy{}\PYZhy{}\PYZhy{}\PYZhy{}\PYZhy{}\PYZhy{}\PYZhy{}\PYZhy{}}
\PY{c+c1}{\PYZsh{} Referenzfall}

\PY{n+nb}{print}\PY{p}{(}\PY{l+s+s1}{\PYZsq{}}\PY{l+s+s1}{\PYZsh{}}\PY{l+s+s1}{\PYZsq{}} \PY{o}{+} \PY{n}{SCREEN\PYZus{}WIDTH} \PY{o}{*} \PY{l+s+s1}{\PYZsq{}}\PY{l+s+s1}{\PYZhy{}}\PY{l+s+s1}{\PYZsq{}} \PY{o}{+} \PY{l+s+s1}{\PYZsq{}}\PY{l+s+s1}{\PYZsh{}}\PY{l+s+s1}{\PYZsq{}}\PY{p}{)}
\PY{n}{asplt}\PY{p}{(}\PY{p}{)}
\end{Verbatim}
\end{tcolorbox}

    \begin{Verbatim}[commandchars=\\\{\}]
\#-------------------------------------------------------------------------------
----------------------------------------------\#
|                                        Es sind die folgenden Unternehmen
betrachtbar:                                       |
\#-------------------------------------------------------------------------------
----------------------------------------------\#
0) Apple
1) Amazon
2) VW
3) BASF
4) Microsoft
5) Tesla
\#-------------------------------------------------------------------------------
----------------------------------------------\#
|                                Bitte geben Sie namentlich ein Unternehmen aus
der Liste ein:                                | Apple
\#-------------------------------------------------------------------------------
----------------------------------------------\#
|                              Für Apple sind Optionen mit den folgenden
Fälligkeiten verfügbar:                              |
\#-------------------------------------------------------------------------------
----------------------------------------------\#
0) 2020-01-24
1) 2020-01-31
2) 2020-02-07
3) 2020-02-14
4) 2020-02-21
5) 2020-02-28
6) 2020-03-20
7) 2020-04-17
8) 2020-06-19
9) 2020-07-17
10) 2020-09-18
11) 2021-01-15
12) 2021-06-18
13) 2021-09-17
14) 2022-01-21
15) 2022-06-17
\#-------------------------------------------------------------------------------
----------------------------------------------\#
|                                  Bitte geben Sie die Nummer der gewünschten
Fälligkeit ein:                                 | 1
\#-------------------------------------------------------------------------------
----------------------------------------------\#
|                                      Ausgewählt wurde Apple zur Fälligkeit
2020-01-31.                                      |
\#-------------------------------------------------------------------------------
----------------------------------------------\#
            Bezeichnung  Strike     Bid     Ask  Volatilität  InTheMoney
0   AAPL200131P00190000   190.0    0.00    0.02     1.125004       False
1   AAPL200131P00195000   195.0    0.00    0.23     1.355472       False
2   AAPL200131P00200000   200.0    0.00    0.02     1.015630       False
3   AAPL200131P00205000   205.0    0.00    0.03     1.000005       False
4   AAPL200131P00210000   210.0    0.00    0.25     1.181645       False
..                  {\ldots}     {\ldots}     {\ldots}     {\ldots}          {\ldots}         {\ldots}
61  AAPL200131P00370000   370.0   52.35   52.65     0.761721        True
62  AAPL200131P00375000   375.0   57.30   57.65     0.808107        True
63  AAPL200131P00380000   380.0   61.50   62.05     0.756838        True
64  AAPL200131P00410000   410.0   91.15   92.70     1.029302        True
65  AAPL200131P00420000   420.0  102.45  102.65     1.209721        True

[66 rows x 6 columns]
\#-------------------------------------------------------------------------------
----------------------------------------------\#
    \end{Verbatim}

    \begin{center}
    \adjustimage{max size={0.9\linewidth}{0.9\paperheight}}{output_35_1.png}
    \end{center}
    { \hspace*{\fill} \\}
    
    \begin{Verbatim}[commandchars=\\\{\}]
\#-------------------------------------------------------------------------------
----------------------------------------------\#
    \end{Verbatim}

    \begin{center}
    \adjustimage{max size={0.9\linewidth}{0.9\paperheight}}{output_35_3.png}
    \end{center}
    { \hspace*{\fill} \\}
    

    % Add a bibliography block to the postdoc
    
    
    
\end{document}
